\begin{abstract}
  We consider the problem of a client querying an encrypted binary
  tree structure, outsourced to an untrusted server. While the server must not learn the contents of
  the binary tree, we also prevent the client
  from maliciously crafting a query that traverses the tree
  out-of-order. That is, the client should not be able to retrieve nodes outside one contiguous path
  from the root to a leaf.  Finally, the server should not learn which
  path the client accesses, but is guaranteed that the access
  corresponds to one valid path in the tree.  This is an extension of
  protocols such as structured encryption, where it is only
  guaranteed that the tree's encrypted data remains hidden from the
  server.

To this end, we initiate the study of Iterative Oblivious Pseudorandom
Functions (iOPRFs), new primitives providing two-sided, fully
malicious security for these types of applications.  We present a
first, efficient \mbox{iOPRF} construction secure against both
malicious clients and servers in the standard model, based on the DDH
assumption.  We demonstrate that iOPRFs are useful to implement
different interesting applications, including an RFID authentication
protocol and a protocol for private evaluation of outsourced decision
trees. Finally, we implement and evaluate our full iOPRF construction
and show that it is efficient in practice.
\end{abstract}
