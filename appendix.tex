\section{Decision Trees}
\label{app:dtrees}
As an alternative to their paper, we summarize here the changes
necessary to convert the semi-honest secure protocol from
\citeauthor{wu2016privately} to a fully-malicious-secure version using
$\ioprf$.  We will reference our modifications in contrast with their
protocol (Figure 1 in~\cite{wu2016privately}).

\begin{enumerate}
    \item In step 1, the client proves that their input encryptions
    are bits.  This also happens in the maliciously-secure version
    from the original paper.
    
    \item In step 2, the negation of the intended DGK
      comparisons~\cite{dgk} are also computed.  This way the client
      has a ``successful'' comparison one way or the other to use in
      their proof to the server that they are behaving correctly.
    
    \item In step 4, the server additionally encrypts each node of the
    tree with a symmetric key derived from an $\ioprf$.  The keys are
    chosen such that each node can be decrypted by an $\ioprf$
    evaluation that corresponds to that node's location in the binary
    tree, adjusted for the randomly flipped comparisons.  The goal
    here is to restrict the client to only being able to decrypt the
    nodes corresponding to the contiguous path in the tree resulting
    from its comparisons.

  \item In step 5, the client uses PIR~\cite{chang2004single} to
    retrieve the target leaf node instead of conditional OT.  The
    client additionally runs an $\ioprf$ protocol to retrieve the
    symmetric key necessary to decrypt their chosen leaf node.  In
    execution of this protocol, they also prove in ZK that
    the input to the $\ioprf$ corresponds to the correct results of
    the comparison protocol (see Appendix~\ref{app:zkp}).
\end{enumerate}

\subsection{Binding Homomorphic Comparisons to $\ioprf$ Input}
\label{app:zkp}
Since the client now executes two DGK comparisons per level of the
tree, the original intended one and its negation, they now always have
a ``successful'' comparison at every level, which tells them which
direction to go in the tree.  The main idea behind the proofs that
will bind the client to the correct path is that they can use the
encryption of zero that results from a successful comparison as
evidence to the server that they are going in the direction they are
supposed~to.

At each level of the tree $k\in[d]$, the client creates a ciphertext
$c \leftarrow \mathsf{Enc}_{pk}(0)$ and generates a commitment  $\com$ to
$x_k=1$ if the comparison at that node was true and $x_k=0$ if its
negation was true.  This corresponds to the direction their comparison
at level $k$ in the shuffled tree tells them to go on the next level.
They then must prove that there exists an $i$ such that $c$ (the
encryption of zero) is plaintext-equivalent to either
$\mathsf{ct}_{k,i}$ or $\mathsf{ct}'_{k,i}$ (the result of the negated
comparison), and that if it is $\mathsf{ct}_{k,i}$ then $\com$ is 1, or
if it is $\mathsf{ct}'_{k,i}$ then $\com$ is 0.  Then, $comm$ is used
as the commitment in the $\ioprf$ protocol.  This binds the output of
the comparison to the input of the $\ioprf$, completing the proof.

Let $a \equiv b$ signify that $a$ and $b$ are encryptions of the same
value and $a \equiv {0,1}$ signify that $a$ is an encryption of 0 or
1.  The statement being proven can then be writen as follows

\begin{gather*}
\Bigg [ (c \equiv \mathsf{ct}_{k,1} \lor  \ldots \lor c \equiv
\mathsf{ct}_{k,t}) \land \com \equiv 1 \Bigg ] \\\lor\\ \Bigg [ (c \equiv
\mathsf{ct}'_{k,1} \lor  \ldots \lor c \equiv \mathsf{ct}'_{k,t})
\land \com \equiv 0 \Bigg ] 
\end{gather*}
  
We do not produce a full ZK proof for this statement, as it can be
efficiently designed in the same way we design ZK proofs in
Section~\ref{sec:sec-analysis} (plaintext equivalence is equivalent to
a proof of DDH tuple, one-out-of two technique for the $\lor$,
parallel proofs for $\land$ etc.).  For more details on efficient
composition of ZK proofs, see also~\citet{camenisch1997proof}.

\begin{comment}
\begin{figure*}
\begin{framed}
\begin{flushleft} Let $(\mathsf{pk}, \mathsf{sk})$ be a public-secret
key-pair for an additively homomorphic encryption scheme over
$\Z_p$.  We assume the client holds the secret key.  Fix a
precision $t \leq \lfloor \log_2 p \rfloor$.
\end{flushleft}


\begin{itemize}
    \item {\bf Client input:} A feature vector $x \in \Z_p^n$
    where each $x_i$ is at most $t$ bits.  Let $x_{i,j}$ denote the
    $j^\text{th}$ bit of $x_i$.
    
    \item {\bf Server input:} \ul{An $\ioprf$ key $K = ((\alpha_1,
    \beta_1), \ldots, (\alpha_\ell, \beta_\ell))$}.  A complete, binary
    decision tree $\mathcal{T}$ with $m$ internal nodes.  Let $q_1,
    \ldots, q_\ell$ be the indices of the non-dummy nodes, and let
    $f_{q_k}(x) = 1\{x_{i_k} \leq t_k\}$, where $i_k \in [n]$ and $t_k
    \in \Z_p$.  For the dummy nodes $v$, set $f_v(x) = 0$.
    Let $z_0, \ldots, \in \{0,1\}^*$ be the values of the leaves of
    $\mathcal{T}$.
\end{itemize}

\begin{enumerate}
    \item {\bf Client:} For each $i \in [n]$ and $j \in [t]$, compute
    and send $\mathsf{Enc}_{\mathsf{pk}}(x_{i,j})$ to the server.
    \ul{The client proves to the server in zero-knowledge that
    $\forall i : x_i \in \{0,1\}$.}
    
    \setul{6pt}{.4pt}
    \item {\bf Server:} The server chooses $b \getr
    \{0,1\}^\ell$.  Then, for each $k \in [\ell]$, set $\gamma_k = 1-2
    \cdot b_k$.  For each $k \in [\ell]$ and $j \in [t]$, choose
    $r_{k,j}$,\ul{$r'_{k,j}$}$\getr \Z_p^{*}$ and
    homomorphically compute the ciphertext

    \vskip 2mm

    \hfil $\mathsf{ct}_{k,j} = \mathsf{Enc}_{\mathsf{pk}}  \Bigg
    [r_{k,j} \Bigg ( x_{i_k,w} - t_{k,w} + \gamma_k + 3 \cdot
    \sum_{w<j} (x_{i_k,w} \oplus t_{k,w}) \Bigg ) \Bigg ]$ \hfill

    \vskip 2mm

\ul{and also}
    \vskip 2mm
    \setul{12pt}{.4pt}
    \hfil \ul{$\mathsf{ct}'_{k,j} = \mathsf{Enc}_{\mathsf{pk}}  \Bigg
    [r'_{k,j} \Bigg ( x_{i_k,w} - t_{k,w} - \gamma_k + 3 \cdot
    \sum_{w<j} (x_{i_k,w} \oplus t_{k,w}) \Bigg ) \Bigg ]$} \hfill
    \resetul
    \vskip 2mm

    For each $k\in[\ell]$, the server sends the client the ciphertexts
    $(\mathsf{ct}_{k,1}, \ldots, \mathsf{ct}_{k,t})$ \ul{and
    $(\mathsf{ct'}_{k,1}, \ldots, \mathsf{ct'}_{k,t})$} in
    \emph{random} order.

    \item {\bf Client:} The client obtains a set of $\ell$ tuples of
    the form $(\widetilde{\mathsf{ct}}_{k,1}, \ldots,
    \widetilde{\mathsf{ct}}_{k,t})$ from the server.  For each $k \in
    [\ell]$, it sets $b'_k = 1$ if there exists $j \in [t]$ such that
    $\widetilde{\mathsf{ct}}_{k,j}$ is an encryption of 0.  Otherwise,
    it sets $b'_k = 0$.  The client replies with
    $\mathsf{Enc}_{pk}(b'_1), \ldots, \mathsf{Enc}_pk(b'_\ell)$.
    
    \item {\bf Server:} The server chooses $s \xleftarrow{R}
    \{0,1\}^m$ and constructs the permuted tree $\mathcal{T}' =
    \pi_s(\mathcal{T})$, where $\pi_s$ is the permutation associated
    with the bit-string $s$.  \ul{Let $\mu(m)={\mu_1, \ldots,
    \mu_\ell}$ be the index of a node $m$ such that $\mu_i = 0$ if at
    level $i$ of the tree the path to $m$ goes left and $\mu_i = 1$ if
    it goes right.  Each node of $\mathcal{T}'$ is symmetrically
    encrypted with the key $\ioprf_K(\mu(m) \oplus b)$, representing the
    comparisons necessary from the client to reach this node in the
    tree.} Initialize $\rho = 0^m$.  For $k \in [\ell]$, update
    $\sigma_{i_k} = b_k \oplus b'_k$.  Let $\mathcal{T}_s$ be the
    permutation on the node indices of $\mathcal{T}$ effected by
    $\pi_s$, and compute $\rho' \leftarrow \tau_s(\sigma \oplus s)$.
    The server homomorphically computes $\mathsf{Enc}_{pk}(\rho')$ and
    sends the result to the client.

    \item {\bf Client and Server}: The client decrypts the server's
    message to obtain $\rho'$ and then computes the index $i$ of the
    leaf node containing the response.  \ul{The client uses Private
    Information Retrieval to retrieve this leaf node from the server.
    The client then executes an $\ioprf$ protocol with the server with
    input $b'$ to obtain the key to decrypt this node.  For each $i
    \in [\ell]$, the client proves in zero knowledge to the server
    that $b'_i$ corresponds to the correct evaluation of the
    homomorphic comparison computed at that level (see Appendix~A.1).}
    %Some mystery of latex here, if I try to do \ref on the line above
    %it won't compile, so I "hard coded" the section
\end{enumerate}
\end{framed}
\caption{\label{fig:dectrees} \citeauthor{wu2016privately}'s protocol using an $\ioprf$}
\end{figure*}
\end{comment}
