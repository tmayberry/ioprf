\section{Applications}
\label{sec:applications}
\paragraph{OPRF applications} Before presenting applications specific to $\ioprf$s, we briefly
highlight that $\ioprf$s are maliciously-secure OPRFs and can consequently also be used to realize
maliciously-secure adaptive $k$-out-of-$n$ OT and oblivious keyword
search~\cite{adaptiveot,oks}. There, a sender encrypts each document
$i$ with a key $K_i$ that is derived from the document's keyword
$W_i$, i.e., $\kappa_i = \iprf_K(W_i)$. The sender sends resulting
ciphertexts to the receiver. Now, sender and receiver evaluate the
$\ioprf$ such that the receivers gets $\kappa=\ioprf_K(W)$ for a any
keyword $W$ the receiver is interested in. Using $\kappa$, the
receiver can decrypt all documents matching $W$. For more details, we
refer to \citet{oks}.  Note that if we ``structure'' keywords along
the paths of a binary tree, we can allow some party to derive, for
example, all keywords that start with the same prefix.

\paragraph{$\ioprf$ specific applications}
An immediate application specific to our $\ioprf$ (but not OPRFs!) is to force correct
compliance of clients in structured encryption by allowing them to
only query a contiguous path in the graph data.  This can be
accomplished by adding a layer of encryption inside of existing
structured encryption solutions such that each data element is also
encrypted with a key derived from one iteration of the $\ioprf$.
After the structured encryption protocol is complete, an $\ioprf$
protocol is executed which will allow for final decryption of the
results only if they are on a contiguous path.

To hide from the server which path is queried, the client can fetch
each node using Private Information Retrieval or maliciously secure
OT.
Also in scenarios with structured encryption, the $\ioprf$'s
delegation feature can be used to delegate control over well-specified
sub-trees of the original data to other parties. The delegate can then
act as a data owner on their sub-tree, serving requests from clients
with the same security property as the original data owner.

To understand the usefulness of $\ioprf$s, we now
consider a specific implementation of RFID tag authentication which
uses a limited form of structured encryption.


\subsection{RFID}
We support a total of $N$ tags in the system. Each tag uniquely
corresponds to a leaf of a height $\ell$ (binary) key tree. To
identify a tag, a reader (a simple device which can query the tag and
has Internet connectivity) can talk to a database which knows all keys
of the key tree.

The reader wants to know whether a tag is valid by interacting first
with the tag and then with the database. To protect tag privacy,
internal details of a supply/distribution chain etc, the database
should not learn which tag the reader is querying for. An adversary
observing tag-reader interaction or being able to query tags
themselves should not be able to identify or trace/follow tags or
even fabricate new tags, too.


The database knows a secret key $K$ and populates a binary key tree
$T$ as follows. First, nodes in this key tree are indexed by bit
strings following the intuitive notation that the left child (``0'')
of some node indexed by bit string $\gamma_1\ldots\gamma_i$ is index by
$\gamma_1\ldots\gamma_i0$, and the right child (``1'') is indexed by
$\gamma_1\ldots\gamma_i1$. By convention, the root is indexed by
empty bit string $\epsilon$.

Root node $\epsilon$ stores random key
$K_{\epsilon}\getr\{0,1\}^\lambda$.  The left child of the root stores
key $K_0=\ioprf_K(0)$, and the right child stores key
$K_1=\ioprf_K(1)$. For a node $\gamma_1\ldots\gamma_i$,
the left child stores key
$K_{\gamma_1\ldots\gamma_i0}=\ioprf_K(\gamma_1\ldots\gamma_i0)$, and its
right child stores key
$K_{\gamma_1\ldots\gamma_i1}=\ioprf_K(\gamma_1\ldots\gamma_i1)$.

During authentication of tag $x$, the database will run
$\ioprf_K(\cdot)$ as the sender, and the reader will be the receiver
with input bit strings $x=x_1\ldots{}x_\ell$.

Here are protocol details.
\begin{itemize}
  
\item During initialization of a new tag $x$, the database stores a
  sequence of $\ell+1$ keys $K$ on the tag: one for each node on the
  path from the root of the database's tree $T$ to leaf
  $x=x_1\ldots{}x_\ell$. The tag also stores its own ID, i.e., $x$.

\item Each tag now identifies itself to a reader using a variation of
  the Molnar (\fixme{Difference: we send the next bit by appending
    it}) protocol:

  \begin{itemize}
  \item Tag $x$ chooses a random $r$ and sends $r$ together with a
    hash of $r$ and each of their $\ell+1$ keys and the next bit,
    respectively:
    $\trace=(r,T_0=H(r,
    K_\myroot,x_1),\ldots,T_\ell=H(r,K_{x_1\ldots{}x_{\ell-1}},x_{\ell}),H(r,K_{x_1\ldots{}x_\ell}))$.

\item The reader uses our $\ioprf$ to identify the tag as follows (I am now sticking to our running example of $x=1011$):

 \begin{itemize}

 \item The database beings by sending $K_\epsilon$ to the reader.
   
  \item The reader checks whether either $H(r,K_\epsilon,0)$ or $H(r,K_\epsilon,1)$  matches
    $T_0$.

  \item If yes, the reader would continue and query either the left or
    right child of the root with the $\ioprf$, compute keys, check
    which matches etc.
\end{itemize}
  \end{itemize}
\end{itemize}

As you can see, the security we are aiming for asks only for a
(delegatable) OPRF. Our $\ioprf$ supports delegation, but can do more. We
could also ask as an additional security requirement that the reader
should only learn ``one path'', i.e., one tag per interaction with the
database. 


\ignore{\newpage
section{This is the old protocol:}

Here are protocol details.
\begin{itemize}

\item During initialization of a new tag $x$, the database stores a
  sequence of $\ell$ keys $K$ on the tag, one for each node on the path from
  the root of the database's tree to leaf $x$. We keep the intuitive $0$
  and $1$ notation also for keys $K$, so for example tag $x = 1011$ stores
  keys $K_1$, $K_{10}$, $K_{101}$, $K_{1011}$, a total of $\ell=4$ keys.

\item The database computes each key as follows:  
\begin{itemize}
\item for a key corresponding to node $i$, it computes $\seed =
  \ioprf_K(\mathsf{PARENT}(i))$. For example, for $K_{1011}$, it would
  compute $\seed = \ioprf_K(101)$.

\item the database then computes
  $K_{\mathsf{leftChild}}||K_{\mathsf{rightChild}} = \prg(\seed)$. In
  our example, it would compute $K_{1010}||K_{1011} = \prg(\seed)$.

\item the database stores one of the two keys on the tag, the one corresponding to the node. So, $K_{1011}$ in our case.
\end{itemize}

\item Each tag can now identify itself to a reader using the Molnar protocol:

  \begin{itemize}
\item The tag chooses a random $r$ and sends $r$ together with a hash (or a PRF) of $r$ and each of their $\ell$ keys. So, it would send $r, H(r||K_1), H(r||K_{10}), H(r||K_{101}), H(r||K_{1011})$ to the reader.

\item The reader uses our $\ioprf$ to identify the tag as follows (I am now sticking to our running example of $x=1011$):
  \begin{itemize}
    
\item The reader would query the database for $\seed =
  \ioprf_K(\myroot)$, for some special input symbol $\myroot$.

\item The reader would derive $K_0$ and $K_1$ and would then check which of them matches the tag's first hash evaluation. 

\item If yes, the reader would continue and query either the left or right child of the root with the $\ioprf$, compute keys, check which hash matches etc.
\end{itemize}
  \end{itemize}
\end{itemize}

As you can see, the security we are aiming for asks only for a
delegatable OPRF. Our $\ioprf$ supports delegation, but can do more. We
could also ask as an additional security requirement that the reader
should only learn ``one path'', i.e., one tag per interaction with the
database. This is not 100\% true, because we currently also leak the
sibling of each ``key node'' in the above protocol. I don't think this
is a major problem though.
}%ignore

\subsubsection{Delegation}
As $\ioprf$s are delegatable, we also support scenarios where a main
database delegates the information to identify tags of, e.g.,
different countries or regions to different sub-databases.  We abstain
from presenting lengthy details, but delegation~\cite{sherman2}
with $\ioprf$s would bring the advantage that if keys from one
regional sub-database are stolen and thus tags in that region can be
fabricated, tags and their identification in other sub-databases are
still secure.


\subsection{Private Decision Tree Evaluation}
Another application where we can apply an $\ioprf$ is in the area of
private evaluation of decision trees.  There, the goal is to allow a
client holding a feature vector to query an outsourced decision tree
held by a server, resulting in the client receiving the machine
learning classification of their feature vector without the owner of
the decision tree learning what their input was. We refer to
\citet{schneidertree} for an overview.

The protocol by \citet{wu2016privately} accomplishes this with two
main techniques:

\begin{enumerate}[leftmargin=*]
\item Each node of the decision tree stores one value which will be
  compared against one feature of the client's feature vector.  To
  enable this, the client encrypts their feature vector with
  additively homomorphic encryption using the client's public key and
  sends ciphertexts to the server.  For each node of the tree, the
  server computes homomorphic DGK~\cite{dgk} comparisons ``$<$'' of
  one of the client's encrypted features with the specific node's
  value and sends encrypted comparison outcomes back to the client.
  Therewith, the client can identify the path in the tree and the leaf
  node corresponding to their input.

\item Once the correct leaf node is identified, the client uses
  oblivious transfer to retrieve it and compute the final
  classification.
\end{enumerate}

This protocol works for semi-honest clients, but it does not prevent
a malicious client from retrieving leaf nodes which do not actually
correspond to the result of their classification.  This is because the
server is not able to verify that the client traverses a contiguous
path in the tree or that the OT they perform corresponds to that path
if they did.
Consequently, \citeauthor{wu2016privately} suggest an augmented
version of the protocol that can handle malicious clients using a new
\emph{conditional oblivious transfer}, but a maliciously-secure
version could also be constructed simply by replacing OT with our
$\ioprf$.

Each node in the tree could be encrypted using keys derived
from the $\ioprf$ evaluation of their index, meaning that the client
would have to traverse a path in the tree all the way to the leaf in
order to decrypt it.  The only necessary modification for this
approach to work is a small number of additional ZKPs to ``bind'' the
results of the homomorphic evaluation to the input of the $\ioprf$.
When constructed this way, the client can use much more efficient
(maliciously secure) private information
retrieval~\cite{chang2004single} instead of the expensive conditional
OT designed by \citet{wu2016privately}.  For space reasons, we list only the main technical modifications necessary (in Appendix~\ref{app:dtrees}).
