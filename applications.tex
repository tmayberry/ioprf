\section{Applications}
\label{sec:applications}
\paragraph{OPRF applications} Before presenting applications specific to $\ioprf$s, we briefly
highlight that $\ioprf$s are maliciously-secure OPRFs and can consequently also be used to realize
maliciously-secure adaptive $k$-out-of-$n$ OT and oblivious keyword
search~\cite{adaptiveot,oks}. There, a sender encrypts each document
$i$ with a key $K_i$ that is derived from the document's keyword
$W_i$, i.e., $\kappa_i = \iprf_K(W_i)$. The sender sends resulting
ciphertexts to the receiver. Now, sender and receiver evaluate the
$\ioprf$ such that the receivers gets $\kappa=\ioprf_K(W)$ for a any
keyword $W$ the receiver is interested in. Using $\kappa$, the
receiver can decrypt all documents matching $W$. For more details, we
refer to \citet{oks}.  Note that if we ``structure'' keywords along
the paths of a binary tree, we can allow some party to derive, for
example, all keywords that start with the same prefix.

\paragraph{$\ioprf$ specific applications}
An immediate application specific to our $\ioprf$ (but not OPRFs!) is to force correct
compliance of clients in structured encryption by allowing them to
only query a contiguous path in the graph data.  This can be
accomplished by adding a layer of encryption inside of existing
structured encryption solutions such that each data element is also
encrypted with a key derived from one iteration of the $\ioprf$.
After the structured encryption protocol is complete, an $\ioprf$
protocol is executed which will allow for final decryption of the
results only if they are on a contiguous path.

To hide from the server which path is queried, the client can fetch
each node using Private Information Retrieval or maliciously secure
OT.
Also in scenarios with structured encryption, the $\ioprf$'s
delegation feature can be used to delegate control over well-specified
sub-trees of the original data to other parties. The delegate can then
act as a data owner on their sub-tree, serving requests from clients
with the same security property as the original data owner.

To understand the usefulness of $\ioprf$s, we now
consider a specific implementation of RFID tag authentication which
uses a limited form of structured encryption.


\subsection{RFID}
Radio Frequency IDentification (RFID) applications comprise a large
quantity of RFID tags attached to precious goods and RFID readers
which are connected to a central backend database. The goal is that
readers can properly identify tags using wireless communication in the
presence of adversaries.

An adversary observing wireless tag-reader interaction or being able
to interact with tags themselves should not be able to identify or
trace tags or even fabricate new tags or clone tags to counterfeit
goods. The main technical challenge is that RFID tags are extremely
resource restricted and can merely compute a cryptographic hash
function. While readers can perform more powerful operations, they
typically feature low storage (no state), but have network
connectivity, e.g., to connect to a central database.  RFID security
has been a very active area of research, see \citet{juels} for an
overview.

In a typical scenario, the reader wants to know whether a tag and
therewith the good it is attached to is valid, by interacting first
with the tag and then with the database. Typically, the tag stores a
unique key, and the reader performs a challenge-response type of
authentication, using the database which knows all tags' keys.
However, previous work has assumed that database and readers are
within the same trust domain, as the database learns which tag the
reader is querying for.  This is an unnecessary strong and often
unrealistic requirement.  To protect tag privacy and internal details
of supply or distribution chains, the database should not learn which
tag the reader is querying for. For example, if several readers
successively query for the same tag, the database knows that a
specific tag has traveled between these readers. At the same time, the
database does not want to give unrestricted access to the reader or
allow queries which leak more information than necessary for the
identification of a single tag per query. If the reader would receive
more information, the danger would be that a reader fabricates tags.
To mitigate these problems, we show how we can extend a prominent RFID
security protocol from the literature, the one by \citet{molnar}, by a
simple application of our $\ioprf$.

\paragraph{High-Level Idea} In the original work by \citeauthor{molnar}, the
database prepares a binary key tree of height $\ell$ storing random
keys in nodes. Leaves in the tree are enumerated by their path from
the root to the leaf. For example, the left most leaf is represented
by the bit string of $\ell$ zero bits. A tag is uniquely identified by
its ID, a bit string $x=(x_1\ldots{}x_\ell)$. During initialization, a
tag with ID $x$ receives all keys from the root to the leaf
represented by $x$. During tag identification, the tag chooses a
random $r$, ``encrypts'' $r$ with each of their keys, and sends the
resulting sequence of ciphertexts to the reader. The reader can access
the database and query keys. The reader checks which path in the tree
decrypts and ends up with a specific ID (leaf). As you can see, this
protocol does not protect the tag from a prying database. A simple
solution of just sending the whole key tree to the reader might
overburden the reader's storage capabilities and also impose a
security risk: having access to all keys, the reader could fabricate 
an arbitrary number of tags.

Our modification to the \citeauthor{molnar} protocol simply consists
of exchanging the way keys in the tree are computed. In our case, the
keys are outputs of the $\ioprf$ which will allow the reader to query
the database for exactly one contiguous path.  As a result, we hide
from the database which tag the reader is querying for, and the
database knows that the reader only gets one path of secrets from the
tree and will be able to identify exactly one tag with it.

\subsubsection{Technical Details}
Let $N=2^\ell$ be the total number of tags in the system. Each tag
uniquely corresponds to a leaf of a height $\ell$ binary \emph{key
tree}. To identify a tag, a reader can communicate with the database
which stores all keys of the key tree.

\paragraph{Preliminaries}
The database knows a secret key $K$ and populates binary key tree
$T$ as follows. First, nodes in this key tree are indexed by bit
strings following the intuitive notation that the left child (``0'')
of some node indexed by bit string $\gamma_1\ldots\gamma_i$ is index by
$\gamma_1\ldots\gamma_i0$, and the right child (``1'') is indexed by
$\gamma_1\ldots\gamma_i1$. By convention, the root is indexed by
empty bit string $\epsilon$.

\paragraph{Database Initialization}
Root node $\epsilon$ stores random key
$K_{\epsilon}\getr\{0,1\}^\lambda$.  The left child of the root stores
key $K_0=\ioprf_K(0)$, and the right child stores key
$K_1=\ioprf_K(1)$. For a node $\gamma_1\ldots\gamma_i$,
the left child stores key
$K_{\gamma_1\ldots\gamma_i0}=\ioprf_K(\gamma_1\ldots\gamma_i0)$, and its
right child stores key
$K_{\gamma_1\ldots\gamma_i1}=\ioprf_K(\gamma_1\ldots\gamma_i1)$.

During authentication of tag $x$, the database will run
$\ioprf_K(\cdot)$ as the sender, and the reader will be the receiver
with input bit strings $x=(x_1\ldots{}x_\ell)$ as follows.

\paragraph{Tag Initialization}
During initialization of a new tag $x$, the database stores a sequence
of $(\ell+1)$ keys $K$ on the tag: one for each node on the path from
the root $K_\epsilon$ of tree $T$ to leaf
$K_x=K_{x_1\ldots{}x_\ell}$. The tag also stores its own ID $x$.

\paragraph{Secure Tag Identification}
 Each tag identifies itself to a reader using a variation of
  the \citeauthor{molnar} protocol:

  \begin{itemize}[leftmargin=*]
  \item Tag $x$ chooses
    $r\getr\{0,1\}^\lambda$ and sends $r$ together with a hash of $r$
    and each of their $(\ell+1)$ keys and the next bit,
    respectively. More formally, the tag sends 
$    \trace=(r,T_0=H(r,
    K_\epsilon,x_1),\ldots,T_\ell=H(r,K_{x_1\ldots{}x_{\ell-1}},x_{\ell}),
    H(r,K_{x_1\ldots{}x_\ell})).
    $

The difference to the original protocol is that we also include
next bit $x_i$ into each hash. This allows the reader to check which
node to query for during the next iteration. Otherwise, the reader
would have to retrieve both children of the current node, revealing 
``one more key'' per level of the tree to the reader.

\item The reader uses the $\ioprf$ as the receiver and the database as
  the sender to identify the tag as follows.

 \begin{itemize}
 \item The database begins by sending $K_\epsilon$ to the reader.
   
  \item The reader checks whether either $H(r,K_\epsilon,0)$ or $H(r,K_\epsilon,1)$  matches
    $T_0$.

  \item Depending on the outcome, the reader iteratively continues and
    queries either the left child ($H(r,K_\epsilon,0)$ matches) or the
    right child ($H(r,K_\epsilon,1)$ matches) of the root with the
    $\ioprf$, compute keys, checks which matches etc.
\end{itemize}
  \end{itemize}

As you can see, the security we are aiming for asks only for a
(delegatable) OPRF. Our $\ioprf$ supports delegation, but can do more. We
could also ask as an additional security requirement that the reader
should only learn ``one path'', i.e., one tag per interaction with the
database. 

 Due to space limitations,
we have moved the {\bf security analysis} to Appendix~\ref{sec:rf-proof}.

\subsubsection{Delegation}
As $\ioprf$s are delegatable, we also support scenarios where a main
database delegates the information to identify tags of, e.g.,
different countries or regions to different sub-databases.  We abstain
from presenting lengthy details, but delegation~\cite{sherman2}
with $\ioprf$s would bring the advantage that if keys from one
regional sub-database are stolen and thus tags in that region can be
fabricated, tags and their identification in other sub-databases are
still secure.


\subsection{Private Decision Tree Evaluation}
Another application where we can apply an $\ioprf$ is in the area of
private evaluation of decision trees.  There, the goal is to allow a
client holding a feature vector to query an outsourced decision tree
held by a server, resulting in the client receiving the machine
learning classification of their feature vector without the owner of
the decision tree learning what their input was. We refer to
\citet{schneidertree} for an overview.

The protocol by \citet{wu2016privately} accomplishes this with two
main techniques:

\begin{enumerate}[leftmargin=*]
\item Each node of the decision tree stores one value which will be
  compared against one feature of the client's feature vector.  To
  enable this, the client encrypts their feature vector with
  additively homomorphic encryption using the client's public key and
  sends ciphertexts to the server.  For each node of the tree, the
  server computes homomorphic DGK~\cite{dgk} comparisons ``$<$'' of
  one of the client's encrypted features with the specific node's
  value and sends encrypted comparison outcomes back to the client.
  Therewith, the client can identify the path in the tree and the leaf
  node corresponding to their input.

\item Once the correct leaf node is identified, the client uses
  oblivious transfer to retrieve it and compute the final
  classification.
\end{enumerate}

This protocol works for semi-honest clients, but it does not prevent
a malicious client from retrieving leaf nodes which do not actually
correspond to the result of their classification.  This is because the
server is not able to verify that the client traverses a contiguous
path in the tree or that the OT they perform corresponds to that path
if they did.
Consequently, \citeauthor{wu2016privately} suggest an augmented
version of the protocol that can handle malicious clients using a new
\emph{conditional oblivious transfer}, but a maliciously-secure
version could also be constructed simply by replacing OT with our
$\ioprf$.

Each node in the tree could be encrypted using keys derived
from the $\ioprf$ evaluation of their index, meaning that the client
would have to traverse a path in the tree all the way to the leaf in
order to decrypt it.  The only necessary modification for this
approach to work is a small number of additional ZKPs to ``bind'' the
results of the homomorphic evaluation to the input of the $\ioprf$.
When constructed this way, the client can use much more efficient
(maliciously secure) private information
retrieval~\cite{chang2004single} instead of the expensive conditional
OT designed by \citet{wu2016privately}.  For space reasons, we list only the main technical modifications necessary (in Appendix~\ref{app:dtrees}).
