\section{New $\iprf$ Construction}
We present our new constructions for both $\iprf$ and $\ioprf$ (Section~\ref{our-ioprf}).  To
ease readability, we omit an important technicality in the description
and proofs: our $\iprf$ and $\ioprf$ constructions do not output
sequences of pseudo-random bit strings of length $\lambda$, but
pseudo-random elements of DDH group $\myG$. Yet, converting elements
to bit strings follows from a standard application of the leftover
hash lemma~\cite{leftover}. As $|p|\geq\lambda$, we have
$|\myG|\ge{}2^\lambda$, and we silently assume in the following that
each party implicitly hashes the output of $\iprf$ and $\ioprf$ using
any pairwise independent family of hash functions.


%\subsection{$\iprf$ Construction}
\label{sec:newprf}
\begin{construction}[Our iPRF]
  \label{const:newprf}
  For any $\ell\in\N$, choose a random generator $g$ and a key
  $K=(K_1,\ldots,K_\ell)$ by sampling $\ell$ pairs of random scalars
  $K_i=({\alpha_{i}},{\beta_{i}})\getr{}(\Z_p)^2.$ For any $\ell$ bit
  input $x=x_1\ldots{}x_\ell$, we define function family
  $\iprf_K(x_1,\ldots,x_\ell)=(f^1_{(\alpha_1,\beta_1)}(x_1),\ldots,f^\ell_{(\alpha_1,\beta_1),\ldots,(\alpha_\ell,\beta_\ell)}(x_1,\ldots,x_\ell))$,
  where
$$  f^i_{(\alpha_1,\beta_1),\ldots,(\alpha_i,\beta_i)}(x_1\ldots{}x_\ell) \stackrel{\text{def}}{=}
  g^{\prod_{x_i=1}\alpha_i \prod_{x_i=0}\beta_i}
=g^{\prod_{j=1}^{i}{\alpha_j^{x_j}\cdot{}\beta_j^{1-x_j}}}.$$


We can rewrite expression
$g^{\prod_{j=1}^{i}{\alpha_j^{x_j}\cdot{}\beta_j^{1-x_j}}}$ as
$g^{\prod_{j=1}^{i}(\alpha_j{}x_j+\beta_j(1-x_j))}$. This
representation of $f^i$ will be very useful during the
presentation of our new techniques later.
 \end{construction}

%   \vskip 1eX\noindent{\bf $\iprf$ Analysis and Delegation:} Due to
% space constraints, we present the formal analysis of
% Construction~\ref{const:newprf} as well as its delegation properties
% in Appendix~\ref{sec:iprf-analysis}.

\subsection{$\iprf$ Analysis}\label{sec:iprf-analysis}
To show that Construction~\ref{const:newprf} is actually an $\iprf$ according to
Definition~\ref{defiprf}, it is sufficient to show that each $f^i$ is
still a pseudo-random function. Mutual independence follows directly from the construction and the random choice of each $(\alpha_i,\beta_i)$.

\begin{theorem}
\label{theorem:newprf}
If the DDH-Assumption holds, then for every $i\leq\ell$ and for every
PPT distinguisher $\mathcal{D}$, there exists a negligible function
$\epsilon$ such that for sufficiently large $\lambda$
$$|
Pr[\mathcal{D}^{f^i_{(\alpha_1,\beta_1),\ldots,(\alpha_i,\beta_i)}(\cdot)}=1]
- Pr[\mathcal{D}^{R^i(\cdot)} = 1]| =\epsilon(\lambda), $$ where the
$(\alpha_1,\ldots, \beta_1),\ldots,(\alpha_i,\beta_i)$ are chosen randomly as
in Construction~\ref{const:newprf}, and $R^i$ is a randomly chosen
function from the set of functions with domain $\{0,1\}^i$ and image
$\myG$.
\end{theorem}

\begin{proof}
This follows because $f^i$ is essentially taking the output from the
PRF in Construction~\ref{nrprf} and adding additionally adding extra
random exponents, which maintains its character as a PRF. We can show
this via reduction.

First, fix any $i\leq\ell$ and consider $f^i$.  We prove the claim by
reduction, showing that if $\mathcal{D}$ exists which can distinguish
between $f^i$ and a random function $R^i$, then we can build
$\mathcal{D}'$ which can distinguish between $F_K$ from
Construction~\ref{nrprf} (on $i$ bit inputs and $i$ element keys) and
a random function $R$ (on $i$ bit inputs). This would violate $F_K$'s
pseudo-random output property of Definition~\ref{def:pr}.

Assume that $\mathcal{D}$ exists that can violate the inequality from
Theorem~\ref{theorem:newprf}.  We create $\mathcal{D'}$ as follows.
First, $\mathcal{D}'$ creates and stores a uniformly random sequence
$(\beta_1,\ldots,\beta_\ell)$ as in Construction~\ref{const:newprf}.
Additionally, it queries its oracle for $g' = PRF(0)$ which is
$g^{\alpha_0}$ if it is interacting with the real instance.  This will
be given to $\mathcal{D}$ as the generator so that $\mathcal{D}'$ can
use results from its oracle, which will always include $\alpha_0$, to
satisfy queries from $\mathcal{D}$.

$\mathcal{D}'$ then runs $\mathcal{D}$ as a subroutine.  Each time
$\mathcal{D}$ queries the oracle for an evaluation on input
$y\in\{0,1\}^i$, $\mathcal{D}'$ does the following: 
\begin{enumerate}
\item Query their own oracle on input $y$ and receive back $z$.
\item Calculate $z' = z^{\prod_{y_i = 0} \beta_i}$.
\item Return $z'$ to $\mathcal{D}$.
\end{enumerate}

Eventually, $\mathcal{D}'$ outputs the same as $\mathcal{D}$.  If
$\mathcal{D}'$ is interacting with PRF $F_K$, then the $z'$ values
$\mathcal{D}'$ gives to $\mathcal{D}$ will be identical to function
$f^i$, due to $\mathcal{D}'$ being able to multiply in the extra
$\beta$ components.  If $\mathcal{D}'$ is interacting with a real
random function, then the responses they give to $\mathcal{D}$ will be
distributed identically to a random function, since $z$ is the result
of a random function and $\mathcal{D}'$ is raising it to fixed powers.
Therefore, if $\mathcal{D}$ has a distinguishing advantage, so will
$\mathcal{D}'$.  $\mathcal{D}'$ has the same advantage that
$\mathcal{D}$ does, rendering the reduction tight.
\end{proof}


\subsection{Delegation}
We achieve delegation for Construction~\ref{const:newprf} using the
following transformation algorithm $T$.
\begin{align*}
&\text{On input: }
(g,((\alpha_1,\beta_1),\ldots,(\alpha_\ell,\beta_\ell)),x^*_1\ldots{}x^*_i),
\\&T\text{ outputs: }
(g',((\alpha_{i+1},\beta_{i+1}),\ldots{},(\alpha_\ell,\beta_\ell))),
\text{where }{g'}=g^{\prod_{j=1}^{i}{\alpha_j^{x_j}\cdot\beta_j^{1-x_j}}}.
\end{align*}

Observe that $g'$ is effectively a precomputed partial-$\iprf$ for input
$(x^*_1\ldots{}x^*_i)$.  So, if party $P_1$ sends
$({g'},((\alpha_{i+1},\beta_{i+1}),\ldots{},(\alpha_\ell,\beta_\ell)))$
to $P_2$, $P_2$ can then compute $\iprf$ outputs
$(v_{i+1},\ldots,v_\ell)$ for any input string $x=(x_1\ldots{}x_\ell)$
which has $(x^*_1\cdots{}x^*_i)$ as a prefix by computing
$v_{k}={g'}^{{\prod_{j={i+1}}^k{\alpha_j^{x_j}\cdot\beta_j^{1-x_j}}}}$.

\begin{lemma}
Construction~\ref{const:newprf} with transformation $T$ is a
delegatable $\iprf$.
\end{lemma}

\begin{proof}
  We prove this by straightforward reduction.  Let $\iprf_K$ be
  Construction~\ref{const:newprf} for inputs $x$ of length $\ell+1$
  bits, and let $\widehat{\iprf}_{\widehat{K}}$ be
  Construction~\ref{const:newprf} for inputs $x$ of length $\ell$
  bits.  Let prefix $x^*$ be any length $\ell$ bit string, and $K$ and
  $\widehat{K}$ are randomly chosen keys.

  Assume there exists distinguisher $\D$ which can distinguish the
  first $\ell$ outputs from $\iprf_K$ with a prefix different from $x^*$
  with non-negligible probability from $\ell$ random bit strings.

  We build distinguisher $\D'$ who will be able to distinguish the
  $\ell$ outputs from $\widehat{\iprf}_{\widehat{K}}$ from $\ell$
  randomly chosen bit strings.

\begin{enumerate}
\item If $\D$ queries for delegation of length $\ell$ prefix $x^*$,
  $\D'$ will query their challenger for $x^*$ and will get back $z$
  which is either $(v_1,\ldots,v_\ell)=\iprf_K(x^*)$ or $\ell$ random
  bit strings $(r_1,\ldots,r_\ell)$.
\item $\D'$ generates a random pair
  $(\alpha_{\ell+1},\beta_{\ell+1})\getr(\Z_p)^2$.  It computes
  transformation $(g'=z, (\alpha_{\ell+1},\beta_{\ell+1}))$ and sends it
  to $\D$.
\item When $\D$ queries for $x$ with a different prefix than $x^*$, $\D$
forwards $x$ to their challenger, forwards the response to $\D$ and
outputs whatever $\D$ outputs.
  
\end{enumerate}  
If $\D'$ is receiving the output of a $\widehat{\iprf}_{\widehat{K}}$,
then the values it gives to $\D$ will be identifically distributed to
correct outputs of a delegated $\iprf$, with the effective key of $K$
concatenated with the random $(\alpha_{\ell+1}, \beta_{\ell+1})$.  If
$\D'$ is receiving random strings $(r_1,\ldots,r_\ell)$, then $\D$ is also
getting random strings.  Therefore, $\D$'s view is distributed
identically to its distinguishing game. If $\D$ has a non-negligible advantage in distinguishing, then
$\D'$ will have the same advantage in distinguishing $\iprf$ output from random strings.
\end{proof}



 
% \vskip 1eX\noindent{\bf A Straightforward $\ioprf$ with one-sided security}
%In Appendix~\ref{sec:ot-ioprf}, we show how a trick similar to \citet{oprf} con%verts the above
%$\iprf$ to an $\ioprf$. This first $\ioprf$ is OT-based and elegant,
%yet it only offers one-sided security~\cite{efficient2pc} against a
%malicious receiver and semi-honest sender.
\subsection{Warm-up: simple $\ioprf$ with One-Sided Security}
\label{sec:ot-ioprf}
Our $\iprf$ from Construction~\ref{const:newprf} can be computed as an
$\ioprf$ with only one-sided security, i.e., malicious receiver or
semi-honest (or malicious, but only focusing on violating
privacy~\cite{one-sided}) sender, using a similar approach as the OPRF
by \citeauthor{oprf} (Construction~\ref{ot-oprf}).  Let
$\ot(b, y_0, y_1)$ denote any $\binom{2}{1}$ oblivious transfer
protocol which is one-sided simulatable~\cite{one-sided} or even
maliciously secure~\cite{schneiderot,schollot}.  Sender $S$ holds
$y_0$ and $y_1$ from $\Z_p$, receiver $R$ holds $b\in\{0,1\}$, and $R$
obliviously retrieves $y_b$ from $S$. Let $x=(x_1,\ldots,x_\ell)$ be
$R$'s input.  Our first OT-based construction for a $\proto$ protocol
gives an $\ioprf$ with one-sided security and works as follows.

\begin{construction}[One-Sided Secure $\ioprf$]
\label{const:one-side}
  \begin{itemize}
\item $S$ generates $\ell$ random scalars $r_i\getr\Z_p$.
\item For each $1 \leq i \leq \ell$, $R$ and $S$ execute $\ot(x_i, r_i\beta_i,r_i\alpha_i)$,  $R$ stores the result as $z_i$.
\item $S$ sends to $R$ the sequence $C=(C_1,\ldots,C_\ell)$ where
  $C_i = g^\frac{1}{\prod_{j=1}^{i} r_j}.$
\item $R$ recovers $\iprf$ output sequence $(v_1,\ldots,v_\ell)$ by calculating $v_i
  = C_i^{\prod_{j=1}^{i} z_j}.$
\end{itemize}
\end{construction}

{\bf Correctness:}  For all $1 \leq i \leq \ell$, we have
\begin{equation}
\begin{aligned}
v_i &= C_i^{\prod_{j=1}^{i} z_i} 
= g^{\frac{1}{\prod_{j=1}^{i} r_j} \cdot \prod_{j=1}^{i} z_j} 
= g^{\frac{1}{\prod_{j=1}^{i} r_j} \cdot \prod_{j=1}^{i} (\alpha_jr_j)^{x_j}(\beta_jr_j)^{1-x_j} } 
\\&= g^{\prod_{j=1}^{i} \alpha_j^{x_i}\beta_j^{1-x_i}}.
\end{aligned}
\end{equation}


To prove security for Construction~\ref{const:one-side}, we could make
a similar argument as \citet{oprf}, but rely on a one-sided
simulatable OT. However, we refrain from presenting more details, as
this $\ioprf$ anyways provides only one-sided security and conversion
to malicious security would be difficult. One would need to prove
correct computation of the $C_i$ and expensive maliciously secure OT
with ZK proofs that the sender's input $(r_i\beta_i,r_i\alpha_i)$
matches previous commitments to $\alpha_i$ and $\beta_i$. This is very
different from standard committed or verifiable
OT~\cite{commit-ot,verifot,commit-ot2}.





 
