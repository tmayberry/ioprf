\section{New Constructions}
We present our new constructions for both $\iprf$ and $\ioprf$.  To
ease readability, we omit an important technicality in the description
and proofs: our $\iprf$ and $\ioprf$ constructions do not output
pseudo-random bit strings of length $\lambda$, but pseudo-random
elements of DDH group $\myG$. Yet, converting elements to bit strings
follows from a standard application of the leftover hash
lemma~\cite{leftover}. As $|p|\geq\lambda$, we have
$|\myG|\ge{}2^\lambda$, and we silently assume in the following that
each party implicitly hashes the output of $\iprf$ and $\ioprf$ using
any pairwise independent family of hash functions.


\subsection{$\iprf$ Construction}
\label{sec:newprf}
\begin{construction}[Our iPRF]
  \label{const:newprf}
For any $\ell\in\N$, choose a key $K=(\vec{{\alpha}},\vec{\beta})$ by
sampling $\ell$ pairs of random scalars
$K_i=({\alpha_{i}},{\beta_{i}})\getr{}(\Z_p)^2.$ For any $\ell$ bit
input $x=x_1\ldots{}x_\ell$, we define function family
$\iprf_K(x_1,\ldots,x_\ell)=(f^1_{(\alpha_1,\beta_1)}(x_1),\ldots,f^\ell_{(\alpha_1,\beta_1),\ldots,(\alpha_\ell,\beta_\ell)}(x_1,\ldots,x_\ell))$,
where
\begin{align*}
f^i_{(\alpha_1,\beta_1),\ldots,(\alpha_i,\beta_i)} &=
g^{\prod_{x_i=1}\alpha_i \prod_{x_i=0}\beta_i}\\&=
g^{\prod_{j=1}^{i} (\alpha_jx_j+\beta_j(1-x_j))}.
\end{align*}
 
\ignore{Define the function ensemble $F' = \{F'_n\}_{n\in N}$.  For every $n$, a key of a function $F'_n$ is a tuple, $\langle P,Q,g,\vec{\alpha},\vec{\beta}\rangle$, 
where $P$ is an $n$-bit prime, $Q$ a prime divisor of $P-1$, $g$ an element of order $Q$ in $\mathbb{Z}_{p}^*$ and $\vec{\alpha}=\langle 
\alpha_0,\alpha_1, \ldots , \alpha_n \rangle$ and $\vec{\beta}=\langle 
\beta_0,\beta_1, \ldots ,\beta_n \rangle$, uniformly random sequences of $n+1$ elements of $\mathbb{Z}_Q$.  For any $n$-bit input, $x=x_1 x_2 \ldots x_n$, the 
function $f'_{P,Q,g,\vec{a},\vec{b}}$ is defined by:
$$f'_{P,Q,g,\vec{\alpha},\vec{\beta}}(x) =(g^{\alpha_0})^{\prod_{x_i=1}\alpha_i \prod_{x_i=0}\beta_i}$$
}%ignore
 \end{construction}

\todo{Show that this is a delegatable PRF.}

To show that our new $\iprf$ is actually an $\iprf$ according to
Definition~\ref{defiprf}, it is sufficient to show that each $f^i$ is
still a pseudo-random function.

\begin{theorem}
\label{theorem:newprf}
If the DDH-Assumption holds, then for every $i\leq\ell$ and for every
PPT distinguisher $\mathcal{D}$, there exists a negligible function
$\epsilon$ such that for sufficiently large $\lambda$
$$|
Pr[\mathcal{D}^{f^i_{(\alpha_1,\beta_1),\ldots,(\alpha_i,\beta_i)}}=1]
- Pr[\mathcal{D}^{R^i} = 1]| =\epsilon(\lambda), $$ where the
$(\alpha_1,\ldots_1),\ldots,(\alpha_i,\beta_i)$ are chosen randomly as
in Construction~\ref{const:newprf}, and $R^i$ is a randomly chosen
function from the set of functions with domain $\{0,1\}^i$ and image
$\myG$.
\ignore{
where in the first probability, $f'_{P,Q,g,\vec{a},\vec{\beta}}$ is distributed according to $F'_n$, and in the second probability the distribution of $R_{P,Q,g}$ is uniformly chosen in the set of functions with the domain $\{0,1\}^n$ and range $\langle g\rangle$.
}%ignore

\ignore{$$| Pr[\mathcal{M}^{f'_{P,Q,g,\vec{\alpha},\vec{\beta}}}(P,Q,g)=1] - Pr[\mathcal{M}^{R_{P,Q,g}}(P,Q,g) = 1]| < \frac{1}{n^\gamma} $$

where in the first probability, $f'_{P,Q,g,\vec{a},\vec{\beta}}$ is distributed according to $F'_n$, and in the second probability the distribution of $R_{P,Q,g}$ is uniformly chosen in the set of functions with the domain $\{0,1\}^n$ and range $\langle g\rangle$.
}%ignore
\end{theorem}

\begin{proof}
Fix any $i\leq\ell$ and consider $f^i$.  We prove the claim by
reduction, showing that if $\mathcal{D}$ exists that can distinguish
between $f^i$ and a random function $R^i$, then we can build
$\mathcal{D}'$ that can distinguish between $F_K$ from
Construction~\ref{nrprf} (on $i$ bit inputs and $i$ element keys) and a
random function $R$ (on $i$ bit inputs).

First, assume that $\mathcal{D}$ exists that can violate the
inequality from Theorem~\ref{theorem:newprf}.  We create
$\mathcal{D'}$ as follows.  First, $\mathcal{D}'$ creates and stores a
uniformly random vector $\vec{\beta}$ as in
Construction~\ref{const:newprf}.  $\mathcal{D}'$ then runs
$\mathcal{D}$ as a subroutine.  Each time $\mathcal{D}$ queries the
oracle for an evaluation on input $y\in\{0,1\}^i$, $\mathcal{D}'$ does the
following:

\begin{enumerate}
\item Query their own oracle for $y$ and receive back $z$.
\item Calculate $z' = z^{\prod_{y_i = 0} \beta_i}$.
\item Return $z'$ to $\mathcal{D}$.
\end{enumerate}

Eventually, $\mathcal{D}'$ outputs the same as $\mathcal{D}$.  If
$\mathcal{D}'$ is interacting with PRF $F_K$, then the $z'$ values
they give to $\mathcal{D}$ will be identical to function $f^i$, due to
them being able to multiply in the extra $\beta$ components.  If
$\mathcal{D}'$ is interacting with a real random function, then the
responses they give will be distributed identically to a random
function. \fixme{Yes, but why? There is one sentence of explanation
  missing.}  Therefore, if $\mathcal{D}$ has a distinguishing
advantage, so will $\mathcal{D}'$.
\end{proof}



\refstepcounter{construction}\label{const:ioprf}
\section{Construction~\ref{const:ioprf}: DH-based $\ioprf$}
\label{our-ioprf}
We now present a new $\proto$ protocol which realizes the ideal $\ioprf$
functionality $\fioprf$ from Figure~\ref{idealioprf}.

\subsection{Preliminaries}
%\subsubsection{Preliminaries}
Let there be two generators
$g_1,g_2$ of prime order $p$ group $\myG$ where the DDH assumption
holds. Neither party should know the discrete log of one generator
$g_i$ to the basis of the other generator $g_{j\neq{}i}$, which is
true with high probability if they are chosen at random.
%\fk{I just want to note that in a prime-order group, if the discrete logarithm $\log_{g_1}(g_2)$ is known, then so is $\log_{g_2}(g_1)$.}

\paragraph{Elgamal Encryption}
We will use additive Elgamal encryption with private keys $sk\in\Z_p$
and public keys $pk=g_1^{sk}$. Ciphertext $c$ to encrypt $m\in\Z_p$ is
$c=(c[0],c[1])=(g_1^r,pk^r\cdot{}g_2^m)\leftarrow\enc_{pk}(m)$, where
$r\getr\Z_p$.

\paragraph{Pedersen Commitments}
A Pedersen commitment $\com(m)\in\myG$ to message $m\in\Z_p$ is
defined as $\com(m)=g_1^r\cdot{}g_2^m$, where $r\getr\Z_p$.  To open
$\com(m)$, reveal tuple $(m,r)$. Pedersen commitments are perfectly
hiding and computationally binding.

\subsection{High-Level Intuition}
In round $i$ of $\ell$ rounds, sender $S$ will receive
two ciphertexts $V_i$ and $D_i$ from receiver $R$.  During the course of the
protocol, one of these ciphertexts will contain the $\ioprf$ output and one acts
as a ``dummy'', to keep $S$ from learning input bits $x_i$ of $R$.  They are
interchanged between rounds depending on the input bits.

For each round, using the $i^{\text{th}}$ round's keys $(\alpha_i,\beta_i)$, $S$
will then ``apply'' $\alpha_i$ to $V_i$ and $\beta_i$ to $D_i$, and send the
results back to $R$. In preparation for the next round $(i+1)$, if $x_{i+1}\neq
x_{i}$, $R$ will swap $V_i$ and $D_i$ for the next round.  After $\ell$ rounds,
$V_\ell$ will have the keys applied which correspond to the input bits of $R$, and
$D_\ell$ will have the complementary combination of keys applied.  $V_0$ is initialized
as an encryption of 1, so $V_\ell$ will contain the correct $\ioprf$ output, whereas
$D_0$ is initialized as an encryption of 0 so it will not contain any information.

\subsection{Technical Details}
For some input string $x=(x_1\ldots{}x_\ell)$, we define the output of
$\proto$ for the receiver as $(v_1,\ldots,v_\ell)=\ioprf_{K}(x)$ with
$v_i=g_2^{\prod_{j=1}^{i}(\alpha_j{}x_j+\beta_j(1-x_j))}$ and $K=\{(\alpha_i,\beta_i)\}^\ell_{i=1}$. 
We now describe details of Construction~\ref{const:ioprf} by its formal
$\proto$ interface (Definition~\ref{def:ioprf}), i.e., first its
initialization and then its iterative processing.

\subsubsection{$\proto$ Initialization}
Sender $S$ randomly chooses secret key
$K=((\alpha_1,\beta_1),\ldots,(\alpha_\ell,\beta_\ell)),
(\alpha_i,\beta_i)\getr(\Z_p)^2$.

$S$ also commits to $K$ by computing $2\ell$ Pedersen commitments
  $(\com(\alpha_i),\com(\beta_i))$. $S$ sends them to $R$ and
  proves knowledge of plaintexts in ZK (see \S\ref{pokop}).

Receiver $R$ computes a random
Elgamal private key $sk\getr\Z_p$ and public key $pk=g_1^{sk}$, and
sends $pk$ to $S$. Receiver $R$ proves knowledge of $sk$ using
a standard Schnorr ZK proof of knowledge (see \S\ref{poe}).


Receiver $R$ computes $V_0 \leftarrow\enc_{pk}(1)$ and
$D_0\leftarrow\enc_{pk}(0)$, sends them to $S$ and proves that
these are encryptions of $1$ and $0$ (see \S\ref{poe} below). 

\subsubsection{$\proto$ Iterative Processing in $\ell$ Rounds}
In round $i\in\{1,\ldots,\ell\}$, for $S$' input bit $x_i$:
\begin{enumerate}
  
\item {\bf Receiver shuffles:}
\begin{enumerate}%[leftmargin=0.3cm]
\item For input bit $x_i$, $R$ computes Pedersen commitment
  $\com{}(x_i)$ and proves that $x_i\in\{0,1\}$ (see
  \S\ref{pobit}). Similarly, $R$ computes $\com{}(1-x_i)$
  and proves that $(1-x_i)\in\{0,1\}$ (see \S\ref{pobit}). Finally,
  $R$ proves that the sum of plaintexts behind
  $\com{}(x_i)$ and $\com{}(1-x_i)$ equals $1$ (see
  \S\ref{pkseo}).


\item  Receiver $R$ chooses $r,r',r'',r'''\getr\Z_p$ and computes Elgamal ciphertexts
  \begin{align*}
    c_i&=(g_1^r\cdot{}V_{i-1}[0]^{x_i},pk^{r}\cdot{}V_{i-1}[1]^{x_i})
    \\c'_i&=(g_1^{r'}\cdot{}V_{i-1}[0]^{1-x_i},pk^{r'}\cdot{}V_{i-1}[1]^{1-x_i})
    \\d_i&=(g_1^{r''}\cdot{}D_{i-1}[0]^{x_i},pk^{r''}\cdot{}D_{i-1}[1]^{x_i})
    \\d'_i&=(g_1^{r'''}\cdot{}D_{i-1}[0]^{1-x_i},pk^{r'''}\cdot{}D_{i-1}[1]^{1-x_i})
 \end{align*} 
  \ignore{
    \vskip 1eX
\NoIndent{\begin{tabular}{@{}l@{\hskip 0.3cm}l}
    $c_i=(g_1^r\cdot{}V_{i-1}[0]^{x_i},pk^{r}\cdot{}V_{i-1}[1]^{x_i})$
    &$c'_i=(g_1^{r'}\cdot{}V_{i-1}[0]^{1-x_i},pk^{r'}\cdot{}V_{i-1}[1]^{1-x_i})$
    \\$d_i=(g_1^{r''}\cdot{}D_{i-1}[0]^{x_i},pk^{r''}\cdot{}D_{i-1}[1]^{x_i})$
    &$d'_i=(g_1^{r'''}\cdot{}D_{i-1}[0]^{1-x_i},pk^{r'''}\cdot{}D_{i-1}[1]^{1-x_i})$%\text{ and}
          \end{tabular}}
        }%ignore
  and sends $(c_i,c'_i,d_i,d'_i)$ to $S$.
\item Receiver $R$ proves correctness of the above computations in
  ZK. Specifically, $(c_i,c'_i,d_i,d'_i)$ result from correct
  exponentiation with $x_i$ (or $1-x_i$) from $\com{}(x_i)$ (or
  $\com{}(1-x_i)$), and multiplication with a random power of
  $g_1$ and $pk$, i.e., re-randomization (homomorphic addition of
  encryption of $0$).  See \S\ref{pexr} below for details.
   Both parties compute
\begin{align*}
   T_i&=(c_i[0]\cdot{}d'_i[0],c_i[1]\cdot{}d'_i[1])
    \\U_i&=(c'_i[0]\cdot{}d_i[0],c'_i[1]\cdot{}d_i[1]).
\end{align*}   
  \end{enumerate}
In the first round, after this step, $T_1$ is an encryption of $1$ and $U_1$ is an encryption of $0$ if $x_1 = 1$.
If $x_1 = 0$, then $T_1$ is an encryption of $0$ and $U_1$ is an encryption of $1$.
However,  sender $S$ does not know which of the two is the case.

\item {\bf Sender computes PRF:} For $r,r'\getr\Z_p$, $S$ computes the two Elgamal ciphertexts
  \begin{align*}
    X_i&=(g^r_1\cdot{}T_i[0]^{\alpha_i},pk^r\cdot{}T_i[1]^{\alpha_i})
    \\Y_i&=(g^{r'}_1\cdot{}U_i[0]^{\beta_i},pk^{r'}\cdot{}U_i[1]^{\beta_i}),
    \end{align*}
  sends $(X_i,Y_i)$ to $R$, and proves correct exponentiation
  (scalar multiplication of plaintexts) with $\alpha_i$ and $\beta_i$
  coming from previous commitments $\com{}(\alpha_i),\com{}(\beta_i)$
  \emph{and} re-randomization of ciphertexts (see \S\ref{pexr}).

\item {\bf Receiver shuffles back:}
  For $r,r',r'',r'''\getr\Z_p$, $R$ computes%\vskip 2eX
  \begin{align*}
    P_i&=(g_1^r\cdot{}X_i[0]^{x_i},pk^r\cdot{}X_i[1]^{x_i})
    \\P'_i&=(g_1^{r'}\cdot{}X_i[0]^{1-x_i},pk^{r'}\cdot{}X_i[1]^{1-x_i})
   \\Q_i&=(g_1^{r''}\cdot{}Y_i[0]^{x_i},pk^{r''}\cdot{}Y_i[1]^{x_i})
   \\Q'_i&=(g_1^{r'''}\cdot{}Y_i[0]^{1-x_i},pk^{r'''}\cdot{}Y_i[1]^{1-x_i})
\end{align*}
   \ignore{
     \begin{centering}
    \begin{tabular}{l@{\hskip 0.5cm}l}
    $P_i=(g_1^r\cdot{}X_i[0]^{x_i},pk^r\cdot{}X_i[1]^{x_i})$
    &$P'_i=(g_1^{r'}\cdot{}X_i[0]^{1-x_i},pk^{r'}\cdot{}X_i[1]^{1-x_i})$
   \\$Q_i=(g_1^{r''}\cdot{}Y_i[0]^{x_i},pk^{r''}\cdot{}Y_i[1]^{x_i})$
   &$Q'_i=(g_1^{r'''}\cdot{}Y_i[0]^{1-x_i},pk^{r'''}\cdot{}Y_i[1]^{1-x_i})$
  \end{tabular}
  \end{centering}
  \vskip 2eX
  }%ignore
  and sends $(P_i,P'_i,Q_i,Q'_i)$ together with ZK proofs of correct
  computation (see \S\ref{pexr}) to $S$.

  Both $S$ and $R$ compute
  $V_i=(P_i[0]\cdot{}Q'_i[0],P_i[1]\cdot{}Q'_i[1])$ and
  $D_i=(P'_i[0]\cdot{}Q_i[0],P'_i[1]\cdot{}Q_i[1])$.
  
In round $i$, after this step, $V_i$ is an encryption of $\iprf_{K}(x_1,\ldots,x_i)$, and $U_i$ is an encryption of $0$.
When computing $T_{i+1}$ and $U_{i+1}$, these values will be used instead of the encryptions of $0$ and $1$ and the iterative computation of the PRF continues.
Since both parties compute $V_i$ and $U_i$, $R$ cannot cheat and substitute for a value of his choice.

\item Receiver $R$ computes and outputs one $\iprf$ value
  $v_i=\frac{V_i[1]}{V_{i}[0]^{sk}}$.
\end{enumerate}

\paragraph{Discussion}
Observe that, in the last step, $R$ can never decrypt
additively homomorphic Elgamal ciphertext $(V_i[0],V_i[1])$ and thus
compute an $\alpha_i$ or $\beta_i$. As $\alpha_i$ or $\beta_i$ are in
the exponent and due to the hardness DLOG, $R$ can only
compute $v_i=g_2^{\ldots\alpha_i\ldots}$ or
$v_i=g_2^{\ldots\beta_i\ldots}$.
If $R$ wants to run several execution of
Construction~\ref{const:ioprf} and wants that $S$ uses the same key,
then $R$ will verify that commitments sent by $S$ during initialization do not change between executions. This
leads to {verifiability}.
Also note that communication complexity and computational complexity
are both in $O(\ell)$ per query, i.e.,  asymptotically
optimal.

%\vskip 1eX\noindent{\bf Security Analysis:} Due to space constraints, we defer %our full security analysis, including formal proofs, to Appendix~\ref{sec:sec-analysis}.
\section{Security Analysis}
%\subsection{Security Analysis}
\label{sec:sec-analysis}
We prove security of Construction~\ref{const:ioprf} using simulation
in the standard model. The simulation uses several efficient
Zero-Knowledge Proofs of Knowledge hybrids introduced first.  To ease
readability, we actually present Honest-Verifier Zero-Knowledge (HVZK)
versions of the proofs, but one can convert these to maliciously
verifier Zero-Knowledge proofs of knowledge using the following two
general transformations~\cite{efficient2pc}. We stress that we have
evaluated and benchmarked the full malicious verifier ZK proofs of
knowledge in Section~\ref{sec:implementation}, i.e., including the two
transformations.

\subsection{Zero Knowledge (instead of HVZK)}
\label{sec:extraction}
\ignore{We cannot use Fiat-Shamir transform and replace $e$, as we use
  Pedersen commitments for witness extraction.}
All our efficient ZK proofs below are three-move (``Sigma'') ZK
proofs. Recall that a three-move ZK proof comprises messages
$(t,e,s)$, where first message $t$ is a commitment from $P$ sent to
$V$, $e$ is $V$'s challenge sent to $P$, and $s$ is the final message
sent from $P$ to $V$.


To make these proofs zero-knowledge instead of only HVZK, we send an
additional message before first message $t$ of the regular three-move
proof.  In this new first message, $V$ sends a Pedersen commitment
$\com{}(e)=g_1^r\cdot{}g_2^e$ to their random challenge $e$ to
$V$. The proof  continues with $V$ sending their regular
commitment $t$ of the regular three-move proof and $V$ opening
$\com{}(e)$ by sending $(e,r)$. If $\com{}(e)$ matches
$(e,r)$, $P$ finally sends last message $s$ of the regular
proof. Verifier $V$ accepts, if $t$ and $s$ of the regular proof match
$e$.

This technique allows a simulator $\myS$ simulating $P$ to cheat in
the ZK proof. More specifically, after receiving $\com{}(e)$, $\myS$
internally computes a valid ZK proof $(t',e',s')$, assuming a random
challenge $e'$. $\myS$ sends $t'$ to $V$ and receives $(e,r)$. If
$(e,r)$ matches $\com{}(e)$, $\myS$ rewinds $V$ to the point after $V$
has sent $\com{}(e)$. Knowing $e$, $\myS$ computes a $t$ and $s$, such
that $(t,e,s)$ will be accepted by $V$. How exactly $t$ and $s$ are
chosen depends on the statement we want to prove, but are typically
straightforward for the Schnorr-style proofs we use below. We show an
example in \S\ref{pobit}.


\subsection{Witness Extraction for Pedersen Commitments}
To transform our ZK proofs to ZK proofs of knowledge, we rely on the
extractability of commitments.  Pedersen commitments are trapdoor
commitments which means that a party knowing a trapdoor $\rho$ can
open a commitment $\com(\cdot)$ to any plaintext they want
(equivocable).  We use this property for witness extraction in
three-move ZK proofs as follows.

Before starting the actual ZK proof by the first message $t$ from the
prover to the verifier, we send the following two messages.
\begin{enumerate}[leftmargin=*]
\item Prover $P$ sends to verifier $V$: $\hat{g}=g_1^\rho$ for random
  $\rho\getr\Z_p$. 
  \item Verifier $V$ will use this $\hat{g}$ instead of $g_2$ for the computation of the
    commitment to challenge $e$.  That is, $V$ computes and sends back
    commitment $\com(e)=g_1^r\cdot{}\hat{g}^{\,e}$ for their random challenge
    $e\in\Z_p$ as in the previous section.
\end{enumerate}

The ZK proof then continues as usual with $P$ sending $t$ and $V$
opening $\com(e)$ by sending $(e,r)$. If $(e,r)$ match $\com(e)$,
$P$ sends final message $s$ and $\rho$ to $V$. Only if
both is correct, the last ZK proof message $s$ matches $P$'s
commitment $t$ and challenge $e$, and $\rho$ matches $\hat{g}=g_1^\rho$, $V$
accepts.

This setup enables a simulator $\myS$ simulating $V$ to extract the
witness from $P$. After receiving trapdoor $\rho$ from $P$, $\myS$
rewinds $P$ until after the point were $P$ sends $t$ to $V$. Knowing
trapdoor $\rho$, $\myS$ can open $\com(e)$ to any $e'\neq{}e$ they
want by solving $r+\rho\cdot{}e=r'+\rho\cdot{}e'$ for $r'$, i.e., they
compute $r'=r+\rho\cdot{}(e-e')$. Running two executions of the ZK
proof with the same input and messages from $P$ but different
challenges extracts the witness of the ZK proof. Details on which $e$
to send in each execution again depend on the exact three-move ZK
proof, but are typically obvious. We refer to \citet{efficient2pc} for
more details.

In conclusion, these two transformation will render our three-move ZK
proofs below into (fully-maliciously secure) ZK proofs of knowledge. We
name each proof below with a hybrid which we will use in the main
proof later. So, for example, the hybrid for the proof of encryption
is called $\fzk{enc}$.


\subsection{ZK Building Blocks}
Before presenting our main proof of Construction~\ref{const:ioprf}, we
introduce the following ZK proofs that we use as building blocks.

\subsubsection{$\fzk{enc}$: Proof of Encryption/Commitment to $m$}
\label{poe}
To prove that an encryption
$c=(c[0],c[1])=(g_1^r,pk^r)\leftarrow\enc_{pk}(0)$ is an encryption of
$m=0$, $P$ proves that $(g_1,c[0],pk,c[1])$ is a DDH tuple.  You can
prove that tuple
$(u_1=g_1,u_2=g_1^r,u_3=g_1^{sk},u_4=g_1^{sk\cdot{}r})$ is a DDH tuple
using the \citet{cp92} protocol as follows.

\begin{enumerate}[leftmargin=*]
\item $P$ sends $(t_1=u_1^{\rho},t_2=u_3^{\rho})$ for $\rho\getr\Z_p$ to $V$.
  \item $V$ sends $e\getr\Z_p$ to $P$.
  \item $P$ sends $s=\rho+e\cdot{}r$ to $V$.
    \item $V$ accepts if $u_1^s=u_2^e\cdot{}t_1$ and $u_3^s=u_4^e\cdot{}t_2$.
\end{enumerate}

This proof has an important property.\ignore{ Besides showing that a
  tuple is a DDH tuple, it also shows DLOG equivalence, i.e.,
  $\log_{u_1}{u_2}=\log_{u_3}{u_4}$.} Instead of showing that some
ciphertext encrypts $m=0$, we can easily generalize it to show
encryption of arbitrary $m$. Specifically, we set
$c'[1]=\frac{c[1]}{g_2^m}$ and run the proof with $m = 0$ for new Elgamal
ciphertext $(c[0],c'[1])$.

%Finally, observe that Pedersen commitments are essentially just the
%right-hand side $c[1]$ of an Elgamal ciphertext. 
Finally, observe that Pedersen commitments are similarly structured as the
right-hand side $c[1]$ of an Elgamal ciphertext, just without the secret key.
Thus, to prove a
Pedersen commitment $\com(m)$ to $m$, parties divide $\com(m)$ by
$g_2^m$ and run a {\bf Schnorr proof} for $r$ used in the commitment ($P$
sends $t=g_1^\rho$, $V$ sends $e$, $P$ sends $s=\rho+e\cdot{}r$, and $V$
accepts if $g_1^s\sr\frac{\com(m)^e}{g_2^m}\cdot{}t$.)

\subsubsection{$\fzk{pop}$: Proof for Knowledge of Plaintext }
\label{pokop}
For $\com(m)=g_1^r\cdot{}g_2^m$,  prover $P$ can prove that they know
$m$.

\begin{enumerate}[leftmargin=*]
\item $P$ sends $t=g_1^{\rho_1}\cdot{}g_2^{\rho_2}$ for
  $\rho_1,\rho_2\getr\Z_p$ to $V$.
\item $V$ sends $e\getr\Z_p$ to $P$.
  \item $P$ sends $s_1=\rho_1+e\cdot{}r$ and $s_2=\rho_2+e\cdot{}m$ to
    $V$.
    \item $V$ checks whether $g_1^{s_1}\cdot{}g_2^{s_2}\sr\com(m)^e\cdot{}t$.
\end{enumerate}

\subsubsection{$\fzk{bit}$: Proof of Plaintext Bit }
\label{pobit}
For a commitment $\com(x_i)$, prover $P$ can prove that $x_i$ is a
bit, i.e., $x_i\in\{0,1\}$. This is an application of the
\emph{one-out-of-two} (OR) technique~\cite{ooot}. Essentially, $P$
proves that either $x_i=1$ which implies proving that ${\com(x_i)}$
equals ${g_1^{r_1}\cdot{}g_2}$ for some $r_1$, or $x_i=0$ which implies
proving that $\com(x_i)$ equals $g_1^{r_2}$ for some $r_2$. Proving
that ${\com(x_i)}$ equals ${g_1^{r_1}\cdot{}g_2}$ is equivalent to
proving that $\frac{{\com(x_i)}}{g_2}$ equals ${g_1^{r_1}}$.

$P$ will prove that they know (I) an $r$ such that
$g_1^{r}=\frac{{\com(x_i)}}{g_2}$ or (II) an $r$ such that
$g_1^{r}=\com(x_i)$. These are essentially two standard Schnorr
proofs.  The trick is that $P$ chooses $e_1$ and $e_2$ such that, for
the verifier's challenge $e$, we have $e=e_1+e_2$. Prover $P$ proves
knowledge of $r_1$ for (I) using challenge $e_1$ and knowledge of
$r_2$ for (II) using challenge $e_2$. Thus, $P$ can choose either
$e_1$ or $e_2$ before sending their first message of the ZK proof and
cheat in one proof. Without loss of generality, let $x_i=1$, so $P$
will cheat in proof (II). This works as follows.

\begin{enumerate}[leftmargin=*]
\item $P$ sends $t_1=g_1^{\rho_1}$ and
  $t_2=\com(x_i)^{-e_2}\cdot{}g_1^{s_2}$, where $\rho,s_2\getr\Z_p$, to
  $V$.
\item $V$ sends $e\getr\Z_p$ to $P$.
  \item $P$ calculates $e_1 = e - e_2$, sends $e_1,e_2,s_1=\rho_1+e_1\cdot{}r$, and $s_2$ to $V$.

\item $V$ checks $e\sr{}e_1+e_2$, $g_1^{s_1}\sr{}\left(\frac{\com(x_i)}{g_2}\right)^{e_1}\cdot{}t_1$ and $g_1^{s_2}\sr{}\com(x_i)^{e_2}\cdot{}t_2$.
\end{enumerate}

\ignore{If $x_i=0$ then $P$ will modify steps 1 and 3 so that they ``cheat''
on the other side of the proof:

\begin{enumerate}[leftmargin=*]
\item [(1)] $P$ sends $t_1=c_1^{-e_1}\cdot{}g_1^{\rho}\cdot{}g$ and $t_2=g_1^{\rho'}$.
\item [(3)] $P$ calculates $e_2 = e - e_1$, sends $e_1, e_2, s_1=\rho, s_2=\rho'+e_2\cdot{}r$.
\end{enumerate}
}

\subsubsection{$\fzk{sum}$: Proof of Sum of Plaintexts equals $1$ }
\label{pkseo}
For commitments $\com(x)=g_1^r\cdot{}g_2^x$ and
$\com(1-x)=g_1^{r'}\cdot{}g_2^{1-x}$, $P$ shows that the sum of
plaintexts equals $1$.

  \begin{enumerate}[leftmargin=*]
  \item $P$ and $V$ compute
    $\com(1)=\com(x)\cdot{}\com(1-x)=g_1^{r+r'}\cdot{}g_2$.
\item $P$ proves that $\com(1)$ is a commitment to $1$ (see
  \S\ref{poe}).
  \end{enumerate}


\ignore{
\subsubsection{$\fzk{mul}$: Proof of Scalar Multiplication with Group Elements }
\label{pomult}
Let a party commit to $x$ with commitment
$\com(x) =g_1^r\cdot{}g_2^x$.  Given two elements $(A,B)$ of DDH group
$\myG$, such as an Elgamal ciphertext tuple, this party can then prove
in ZK that $(C=A^x,D=B^x)$ are the result of exponentiation with $x$,
i.e., scalar multiplication of $x$ with underlying plaintexts.


\begin{enumerate}[leftmargin=*]
      \item $P$ sends $t_1=A^{\rho_1},t_2=B^{\rho_1},
        t_3=g_1^{\rho_2}\cdot{}g_2^{\rho_1}$, for randomly chosen
        $\rho_i\getr\Z_p$, to $V$.

      \item $V$ sends challenge $e\getr\Z_p$.

      \item $P$ sends $s_1=\rho_1+e\cdot{}x,s_2=\rho_2+e\cdot{}r$.
        \item $V$ checks $A^{s_1}\sr{}C^e\cdot{}t_1$,
          $B^{s_1}\sr{}D^e\cdot{}t_2$, and
          $g_1^{s_2}\cdot{}g_2^{s_1}\sr{}\com(x)^e\cdot{}t_3$.
          
      \end{enumerate}   
}%ignore
\subsubsection{$\fzk{ExR}$: Proof of Exponentiation and Re-Encryption }
\label{pexr}
  One can
      efficiently prove correctness of combinations of linear operations  in one step. 
      We present the  
      example for the correctness of exponentiation of
      two elements $(A,B)$ from group $\myG$ with a committed value $x$
      and then multiplying $A^x$ by $g_1^{r'}$ and $B^x$ by $pk^{r'}$ from our protocol. So, this
      can be used to prove correct exponentiation (homomorphic scalar multiplication) of an Elgamal ciphertext
      by a previously committed scalar  $x$ and subsequent re-randomization of
      the result (homomorphic addition of Elgamal encryption of $0$).

      Specifically, given two group elements $(A,B)$ and commitment
      $\com(x) =g_1^{r}\cdot{}g_2^x$, prove correctness that
      $(C=g_1^{r'}\cdot{}A^x,D=pk^{r'}\cdot{}B^x)$ are the result of
      exponentiation with $x$ and multiplying with $g_1^{r'}$ and
      $pk^{r'}$, $r'\getr\Z_p$, known to $P$.

\begin{enumerate}[leftmargin=*]
  \item $P$ sends $t_1=g_1^{\rho_1}\cdot{}A^{\rho_2},t_2=pk^{\rho_1}\cdot{}B^{\rho_2},t_3=g_1^{\rho_3}\cdot{}g_2^{\rho_2}$ to $V$.
  \item $V$ sends $e\getr\Z_p$ to $P$.
    \item $P$ sends $s_1=\rho_1+e\cdot{}r'$, $s_2=\rho_2+e\cdot{}x$,
      and $s_3=\rho_3+e\cdot{}r$ to $V$.
\item $V$ checks whether $g_1^{s_1}\cdot{}A^{s_2}\sr{}C^e\cdot{}t_1$,
  $pk^{s_1}\cdot{}B^{s_2}\sr{}D^e\cdot{}t_2$, and
  $g_1^{s_3}\cdot{}g_2^{s_2}\sr{}\com(x)^e\cdot{}t_3$.
\end{enumerate}

      
\ignore{
\section{Old ZK Tools}
\subsubsection{ZK Proofs for Exponents}
\subsubsection{Proof of Plaintext Equivalence}
Let $c_1=(c_1[0],c_1[1],)=(g_1^{r_1},pk^{r_1}\cdot{}g^m)$ and
$c_2=(c_2[0],c_2[1],)=(g_1^{r_2},pk^{r_2}\cdot{}g^m)$ be two
encryptions from $\enc_{pk}(m)$. To prove plaintext equivalence of
these two ciphertexts, the prover shows that
$(g_1,\frac{c_1[0]}{c_2[0]},pk,\frac{c_1[1]}{c_2[1]})$ is a DDH tuple.

To prove that some ciphertext $c_1$ encrypts a plaintext $m$ with
respect to base $g$, a simple trick for the prover is to compute
another encryption $c_2$ of $m$ with respect to base $g$, show
plaintext equivalence, and then open randomness of $c_2$.

\subsubsection{Proofs for Arithmetic with Pedersen Commitments}
We can do simple arithmetic on Pedersen Commitments.
\begin{itemize}
\item Addition: given $\com_g(a)$ and $\com_g(b)$, everybody can
  compute and thus verify commitment
  $\com_g(c)=\com_g(a)\cdot{}\com_g(b)$ which commits to
  $c=a+b$. Obviously, no other party than the one originally computing
  $\com_g(a)$ and $\com_g(b)$ can open $\com_g(c)$, but all parties
  know that $\com_g(c)$ is a commitment to $c=a+b$
  
\item Multiplication: a party committing
  \begin{align*}
  \com_g(a)=g_1^{r_a}\cdot{}g^a, \com_g(b)=g_1^{r_b}\cdot{}g^b,
  \com_g(c)=g_1^{r_c}\cdot{}g^{a\cdot{}b}
  \end{align*}
  can prove in ZK that
  $\com_g(c)$ commits to the product of the messages committed in
  $\com_g(a)$ and $\com_g(b)$.

  The trick is to rewrite
  $\com_g(c)=g_1^{r_c-a\cdot{}r_b}\cdot\com_g(b)^{a}$ and then prove
  that all commitments are well formed, and $\com_g(c)$ uses the same
  exponent $a$ as $\com_g(a)$, but with basis $\com_g(b)$ instead of
  $g$. Specifically,
  \begin{enumerate}
  \item $P$ computes and sends
    \begin{align*}
      t_1=g_1^{\rho_1}\cdot{}g^{\rho_2},
      t_2=g_1^{\rho_3}\cdot{}g^{\rho_4},
      t_3=g_1^{\rho_5}\cdot{}\com_g(b)^{\rho_2}
      \end{align*}
      for $\rho_i\getr\Z_p$. Observe that the same randomness $\rho_2$
      is used for the same witness $a$.
    \item $V$ replies by sending challenge $e\getr\Z_p$.
    \item $P$ sends
      \begin{align*}
        s_1&=\rho_1+e\cdot{}r_a,s_2=\rho_2+e\cdot{}a,s_3=\rho_3+e\cdot{}r_b,s_4=\rho_4+e\cdot{}b,\\s_5&=\rho_5+e\cdot{}(r_c-a\cdot{}r_b).
        \end{align*}
    \item $V$ checks
      \begin{align*}
        g_1^{s_1}\cdot{}g^{s_2}\sr{}\com_g(a)^e, g_1^{s_3}\cdot{}g^{s_4}\sr{}\com_g(b)^e, g_1^{s_5}\cdot{}\com_g(b)^{s_2}\sr{}\com_g(c)^e\cdot{}t_3.
        \end{align*}
\end{enumerate}

\end{itemize}
}%ignore


\subsubsection{Proof of Construction~\ref{const:ioprf}}
\label{mainproof}
We now turn to our main proof, showing that
Construction~\ref{const:ioprf} is a secure $\ioprf$. We prove in the
hybrid model, using ZK hybrids with their abbrevations as introduced
in the previous section.  Recall that, in the hybrid model, ZK hybrids
are run by separate trusted third parties. Yet, during simulation, it
is the simulator who takes the role of the TTP and thus automatically
gets the adversary's inputs and can also cheat, see \citet{howto} for
details.

\begin{theorem}
  Assume that Construction~\ref{const:newprf} is an iterative
  pseudo-random function family $\iprf_K(\cdot)$.  Then,
  Construction~\ref{const:ioprf} is an $\ioprf$, realizing
  functionality $\fioprf$ in the
  $\left(\fzk{enc},\fzk{pop},\fzk{bit},\fzk{sum},\fzk{ExR}\right)$
  hybrid-model.
\end{theorem}

\begin{proof}
  First, observe that Construction~\ref{const:ioprf} is correct. Let
  $x$ be the receiver's input, and $K$ the key chosen by the
  sender. If both sender and receiver are honest, then the sender
  outputs nothing, and the receiver outputs
  $(v_1,\ldots,v_\ell)=\iprf_K(x)$. Thus, we focus on proving security
  and build simulators for two cases: one where $S$ is compromised,
  and one where $R$ is compromised.

  We will show that a simulator $\myS$ can be constructed from both
  the perspective of $S$ and $R$ such that the adversary $\A$'s view
  is indistinguishable from real executions of the protocol.  Thus we
  show that neither a compromised $S$ nor a compromised $R$ learn
  anything from the real execution of Construction~\ref{const:ioprf}
  beyond what is specified by the ideal functionality in
  Figure~\ref{idealioprf}.

  In our presentation below, we will use the term
  ``$\myS$ \emph{aborts''} as a shorthand for $\myS$ sending \abort to
  the TTP, simulating its party aborting to $\A$, and then outputting
  whatever $\A$ outputs.

In both cases below, the simulator will faithfully act as a verifier
  for ZKPs when interacting with $\A$ as necessary, aborting if the
  proof does not verify correctly. We omit these messages for
  readability since they require no special knowledge or behavior from
  the simulator. Our strategy will broadly be to:

\begin{itemize}[leftmargin=*]
  \item Replace Elgamal ciphertexts sent by $R$ with encryptions of
  zero (arbitrarily chosen).  Due to Elgamal's IND-CPA property, these
  ciphertexts will be indistinguishable from the real protocol for
  $\A$.  Since $S$ reveives no output from the real execution of the
  protocol, ciphertexts do not have to conform to any
  expectations.

\item Replace computation of $X_i$ and $Y_i$ by $S$
  in the real protocol with an encryption of the output of the
  $\ioprf$ received from the TTP.  $\myS$ does not know
  $K_i=(\alpha_i,\beta_i)$ and so cannot faithfully compute $X_i$ or
  $Y_i$, but it knows from the TTP what output $v_i$
  should. Consequently, $\myS$ crafts these values accordingly to
  simulate the real protocol and ``cheat'' in ZKPs where $\myS$ acts
  as the prover (see, e.g., \S~\ref{sec:extraction}).
\end{itemize}

Together, this will allow the simulator to generate a view which is
indistinguishable from a real execution, \ignore{\fixme{we said that
in the first paragraphs}while having now knowledge beyond that given
by the TTP,} thus proving that our construction is secure according to
Definition~\ref{def:ioprf}.

Note that also for all ZKPs with $\myS$ as a prover, $\myS$ acts as
the TTP and ``cheats'' to convince $\A$.  In many instances, $\myS$
could honestly prove to $\A$, so ``cheating'' is not really. Yet, for
ease of exposition, we assume that all proofs are simulated this way.
  
\vskip 1eX\noindent{\bf Case 1:} We assume that $\A$ has compromised
$S$ and build simulator $\myS$ taking the role of $S$ in the ideal
world, internally simulating a receiver to $\A$ which it only has
black box access to.

$\myS$ starts $\A$ and receives $2\ell$ commitments
$(\com(\alpha_i),\com{}(\beta_i))$ from $\A$. $\myS$ also receives
corresponding $(\alpha_i,\beta_i)$ together with random coins from
$\fzk{pop}$ sent from $\A$ to $\fzk{pop}$. If these do not match the
commitments, $\myS$ \emph{aborts}.
 

$\myS$ also
generates an Elgamal key pair $(sk, pk)$, sends
$pk$ to $\A$, and simulates $\fzk{enc}$.  Also, $\myS$ generates
$V_0=\enc_{pk}(0)$ and $D_0=\enc_{pk}(0)$, sends them to $\A$, and
simulates $\fzk{enc}$.
    
\noindent{}During the $i^\text{th}$ round,
  \begin{enumerate}[leftmargin=*]
  \item $\myS$ sends two independent commitments of zero and simulates
  $\fzk{bit}$ and $\fzk{sum}$.

  \item $\myS$ also computes and sends $(c_i,c'_i,d_i,d'_i)$, all
    encryptions of zero, to $\A$ and simulates
    $\fzk{ExR}$.

  \item $\myS$ receives $(X_i, Y_i)$ from $\A$ as well as
    $(\alpha'_i,\beta'_i)$ and random coins from $\fzk{ExR}$. If
    $\alpha_i\neq\alpha'_i$ or $\beta_i\neq\beta'_i$ or if random
    coins do not match computations specified in
    Construction~\ref{const:ioprf}, then $\myS$ \emph{aborts}. If they
    match, $\myS$ forwards $K_i=(\alpha_i,\beta_i)$ to the TTP.
   
  \item $\myS$ sends $P_i,P'_i,Q_i,Q'_i$, encryptions of zero, to $\A$
    and simulates $\fzk{ExR}$.
    
  \end{enumerate}
  $\myS$ outputs what $\A$ outputs.
During simulation, whenever $\A$ aborts, $\myS$ also \emph{aborts}.

\paragraph{Indistinguishable views} In the protocol, there are three types
of messages that $\myS$ sends to $\A$: Pedersen commitments, Elgamal
ciphertexts, and ZKP messages.  All of the Elgamal ciphertexts are
freshly encrypted (or re-encrypted) using fresh randomness.  They are
thus indistinguishable from any other Elgamal encryption, regardless
of any a priori knowledge that $\A$ might have.  As stated above, the
ZKPs are simulated and are thus also indistinguishable from a real
execution.  Finally, the commitments are perfectly hiding and are
never revealed during the protocol, so they are also indistinguishable
from the commitments of a real execution.

\vskip
1eX\noindent{\bf Case 2:} We assume that $\A$ has compromised $R$ and
build simulator $\myS$ as follows.

$\myS$ starts $\A$.
 $\myS$ randomly selects $\ell$ pairs $(\alpha'_i,\beta'_i)\getr(\Z_p)^2$,
  commits to them, sends commitments to $\A$, and proves knowledge of
  $(\alpha'_i,\beta'_i)$ using $\fzk{pop}$.

$\myS$ receives $pk$ from $\A$ and $(sk',pk')$ from
$\fzk{enc}$ which $\A$ has sent. If $pk\neq{}pk'$ or
$g_1^{sk'}\neq{}pk$, $\myS$ \emph{aborts}.  Also, $\myS$ receives
$(V_0,D_0)$ from $\A$ and $\A$'s random coins from $\fzk{enc}$. If
random coins do not match encryptions of $1$ ($V_0$) or $0$ ($D_0$),
$\myS$ \emph{aborts}.

\noindent{}During the $i^\text{th}$ round, 
\begin{enumerate}[leftmargin=*]
\item $\myS$ receives $(\com(x_i)$, $\com(1-x_i))$ from $\A$ and
  $(x'_i,1-y'_i)$ with the commitments' random coins from
  $\fzk{bit}$. If $x'_i$ or $1-y'_i$ and random coins do not match
  commitments, $\myS$ \emph{aborts}. In the same way, $\myS$ receives
  $z$ and a random coin for the commitment from sum hybrid
  $\fzk{sum}$. If $z\neq{}1$ or $z\neq{}x'_i+1-y'_i$ or the random
  coin does not match the commitment, $\myS$ \emph{aborts}. If
  everything matches, $\myS$ knows $\A$'s input $(x_i,1-x_i)$.

  $\myS$ receives $(c_i,c'_i,d_i,d'_i)$ from $\A$ and random coins and
  $(x'_i,1-y'_i)$ from $\fzk{ExR}$. If $(x'_i,1-y'_i)$ do not match
  the ones from the previous step or if any of the computations do not
  match $(c_i,c'_i,d_i,d'_i)$, $\myS$ \emph{aborts}.

  $\myS$ computes $(T_i,U_i)$ as in Construction~\ref{const:ioprf}.
  
\ignore{   Crucially, as $\myS$ knows $x_i$,
  they also know which of $T_i$ or $U_i$ contains the encryption of
  previous $\iprf$ output $v_{i-1}$ and which contains an encryption
  of $0$. If $x_i=1$, then $T_i$ contains the encryption of $v_{i-1}$
  (encryption of $1$ for $v_0$), and $U_i$ contains an encryption of
  $0$. If $x=0$, it is the other way around.}

\item $\myS$ queries the TTP for $x'_i$ and gets back $v_i$. If
  $x_i=1$, $\myS$ sets $X_i\leftarrow\enc_{pk}(v_i)$ and
  $Y_i\leftarrow\enc_{pk}(0)$.  If $x_i=0$, $\myS$ sets
  $X_i\leftarrow\enc_{pk}(0)$ and $Y_i\leftarrow\enc_{pk}(v_i)$.
  $\myS$ sends $(X_i,Y_i)$ to $\A$ and \emph{cheats} in $\fzk{ExR}$,
  convincing $\A$ that $(X_i,Y_i)$ are the result of raising $T_i$ and
  $U_i$ to $\alpha'_i$ and $\beta'_i$ and then re-encrypting.
  
\ignore{
  If $x_i = 1$,
meaning that $T_i$ contains the input from $\A$ that should be
included in the PRF, then $\myS$ computes $X_i = \enc_pk(y)$ and $Y_i = \enc_pk(0)$. If $x_i = 0$ it computes $Y_i
= \enc_pk(y)$ and $X_i = \enc_pk(0)$.  In either case, it also
``cheats'' the proofs $\fzk{pop}$ and $\fzk{ExR}$ by rewinding after
receiving the challenge and producing a correct commitment to match
the challenge.  This is necessary because $\myS$ does not know
$\alpha$ or $\beta$, only the final output of the $\ioprf$.
}

\item Finally, $\myS$ receives $(P_i,P'_i,Q_i,Q'_i)$ from $\A$ and
  random coins and $(x'_i,1-y'_i)$ from $\fzk{ExR}$. Again, $\myS$
  verifies correct computation of $(P_i,P'_i,Q_i,Q'_i)$ and whether
  $(x'_i,1-y'_i)$ match previously received values. If anything does
  not match, $\myS$ \emph{aborts}.

  $\myS$ computes $(V_i,D_i)$ as in Construction~\ref{const:ioprf}.
  
\end{enumerate}
  $\myS$ outputs what $\A$ outputs.
During simulation, whenever $\A$ aborts, also $\myS$ \emph{aborts}.

\paragraph{Indistinguishable views} As before, the commitments are
perfectly hiding and are not revealed and so are indistinguishable
from commitments of a real protocol execution.  ZKPs are also
simulated as before and are indistinguishable for the same reason.

The only part that is different in this case is the returned values of
$X_i$ and $Y_i$, which have to decrypt to the correct output of the
$\ioprf$ in order to match the real protocol.  Fortunately, $\myS$ can
query the TTP for the correct output and generate encryptions that
match that output.  In the real protocol, $S$ reencrypts $X_i$ and
$Y_i$ before returning them to $R$, and so they are indistinguishable
from the fresh encryptions generated by $\myS$.
\end{proof}
As $R$ verifies whether $S$ sends the same commitments to
$(\alpha_i,\beta_i)$ during multiple executions of
Construction~\ref{const:ioprf}, we trivially achieve verifiability.



\subsection{OT-based $\proto$ Construction}

The $\iprf$ above can be computed as an oblivious PRF between a sender $S$ and a receiver $R$.  In this protocol,
the receiver $R$ has an array of bits $(x_1, \ldots, x_\ell)$ and wishes to compute $\iprf(x_1, \ldots, x_\ell)$ using
the key $K$ possessed by $S$.

Let $\ot(b, y_0, y_1)$ denote a secure 1-out-of-2 oblivious transfer protocol between $R$ and $S$ where
$S$ holds $y_0$ and $y_1$, $R$ holds $b\in\{0,1\}$, and $R$ obliviously retrieves $y_b$ from $S$.  The OT-based version of $\proto$ works as follows.

\begin{itemize}
\item $S$ generates $\ell$ random scalars $r_i\getr\Z_p$
\item For each $1 \leq i \leq \ell$, $R$ and $S$ execute $\ot(x_i, r_i\alpha_i, r_i\beta_i)$ and stores the result as $z_i$
\item $S$ sends to $R$ the vector $\vec{C}$ where $\forall 1 \leq i \leq \ell$, $C_i =  G \cdot \frac{1}{\prod_{j=1}^{i} r_j}$
\item $R$ recovers $\iprf$ output vector $v$ by calculating $v_i = C_i \cdot \prod_{j=1}^{i} z_j$
\end{itemize}


{\bf Correctness:} \todo{update notation} For all $1 \leq i \leq \ell$ we have
\begin{equation}
\begin{aligned}
v_i &= C_i \cdot \prod_{j=1}^{i} z_i \\
&= G \cdot \frac{1}{\prod_{j=1}^{i} a_j} \cdot \prod_{j=1}^{i} z_j \\
&= G \cdot \frac{1}{\prod_{j=1}^{i} a_j} \cdot \prod_{j=1}^{i} (a_jr_j)^{b_j}(a_js_j)^{1-b_j} a_j \\
&= G \cdot \prod_{j=1}^{i} r_j^{b_j}s_j^{1-b_j}
\end{aligned}
\end{equation}

\todo{Show that this is an OPRF. And show that the iterative evaluation is still secure.}


