\section{$\ioprf$ Background}
NOTE: for one-sided security, we use the OT-based solution.

\todo{Add description that PRF outputs must be hashed to become real
  output of the PRF. This is necessary for the leftover hash lemma.}


\fixme{This is still the old ``broken'' definition. Needs to be updated. What's important: an HMAC is also an iPRF, but our goal is that we can essentially query PRF while making sure that the queries are prefixes of each other. We also have delegation. Start with not having one function, but $\ell$ functions?}
\begin{definition}[$\iprf$]
  Consider function family
  $f_K(x_1,\ldots,x_\ell):\{0,1\}^{\lambda\cdot\ell}\times\{0,1\}^{\ell}\rightarrow{}(v_1,\ldots,v_\ell)\in\{0,1\}^{\lambda\cdot\ell}$. For
  a randomly chosen key $K$, range $V_\lambda=(v_1,\ldots,v_\ell)$ is
  a family of random variables (a probability ensemble) of bit strings
  of length $\ell\cdot\lambda$. We call function $f_K$ an iterated
  pseudo-random function family $(\iprf_K)$ \emph{iff} for all
  adversaries $\A$ and for all $(x_1,\ldots,x_\ell)\in\{0,1\}^\ell$
  there exists a negligible function $\epsilon$ such that for
  sufficiently large $\lambda$
\begin{align*}\forall{}i\in\{1,\ldots,\ell\}:|&Pr[(v_1,\ldots,v_\ell)\leftarrow{}V_\lambda:\A(v_1,\ldots,v_i)=1]-\\&Pr[(v_1,\ldots,v_\ell)\leftarrow{}V_\lambda,u\leftarrow{}U_\lambda:\A(v_1,\ldots,v_{i-1},u)=1]|\\&=\epsilon(\lambda),
  \end{align*}
  where $U_\lambda$ is the random variable describing uniformly random
  bit strings of length $\lambda$. The probabilities are taken over
  the random coins of $\A$ and $K$.

\end{definition}

\begin{definition}[$\proto$]
  Let $\iprf_K$ be an iterated pseudo-random function family.  An
  iterated \emph{oblivious} pseudo-random function is a probabilistic
  protocol $\proto$ between a sender $S$ with input key
  $K\in\{0,1\}^{\lambda\cdot\ell}$ and receiver $R$ with input bit string
  $(x_1,\ldots,x_\ell)\in\{0,1\}^{\ell}$ with the following
  properties.

  \begin{itemize}
   
\item Protocol $\proto$ realizes the ideal functionality $\iprf$: on
  input $K$ from $S$ and $(x_1,\ldots,x_\ell)$ from $R$, it outputs
  $(v_1,\ldots,v_\ell)=\iprf_K(x_1,\ldots,x_\ell),|v_i|=\lambda$, to
  $R$ and nothing to $S$.
  
\item For all adversaries $\A$ in the real world, there exists a
  simulator $\myS_R$ in the ideal world such that $R$'s view
  $\mathsf{REAL}_{\proto,\A,R}(x_1,\ldots,x_\ell,K)$ in the real world is
  computationally indistinguishable from $R$'s view
  $\mathsf{IDEAL}_{\iprf,\myS_R(x_1,\ldots,x_\ell)}(x_1,\ldots,x_\ell,K)$ in
  the ideal world.

\item
   For all adversaries $\A$ in the real world, there exists a
  simulator $\myS_S$ in the ideal world such that $S$'s view
  $\mathsf{REAL}_{\proto,\A,S}(b_1,\ldots,b_\ell,K)$ in the real world is
  computationally indistinguishable from $S$'s view
  $\mathsf{IDEAL}_{\iprf,\myS_S}(b_1,\ldots,b_\ell,K)$ in
  the ideal world.  
\end{itemize}
\end{definition}

\todo{Introduce delegatable PRFs}
