\section{$\ioprf$ Background}
NOTE: for one-sided security, we use the OT-based solution.

\todo{Add description that PRF outputs must be hashed to become real
  output of the PRF. This is necessary for the leftover hash lemma.}



\begin{definition}[$\iprf$]\label{defiprf}
  For keys $K=(K_1,\ldots,K_\ell)\in\{0,1\}^{\ell\cdot\lambda}$ and
  inputs $x=(x_1\ldots{}x_\ell)\in\{0,1\}^\ell$, an iterated
  pseudo-random function family $\iprf_{K}(x)$ is a sequence of
  function families
  $(f^1_{K_1}(x_{1}),\ldots,f^\ell_{K_1,\ldots,K_\ell}(x_{1}\ldots{}x_{\ell}))$,
  where each
  $f^i_{K_1,\ldots,K_i}(x_{1}\ldots{}x_{i}):\{0,1\}^{i\cdot\lambda}\times\{0,1\}^{i}\rightarrow{}v_i\in\{0,1\}^\lambda$
  is a pseudo-random function family with keys $(K_1,\ldots,K_i)$ from $K$ and
  inputs $(x_1\ldots{}x_i)$ from $x$.
\end{definition}

\paragraph{Remarks}
An implication of Definition~\ref{defiprf} is that random variable
$V_\lambda=(v_1,\ldots,v_\ell)=\iprf_K(x)\in\{0,1\}^{\ell\cdot\lambda}$
is computationally indistinguishable from random the variable
$U_\lambda$ of uniform bit strings of length $\lambda$.

A simple construction for an $\iprf$ is based on
variable length PRFs such as HMAC an a collision resistant hash
function $H$. For example, consider
$\iprf_K(x)=(\hmac_{H(K_1)}(x_1),\ldots,\hmac_{H(K_1||\ldots||K_\ell)}(x_1\ldots{}x_\ell))$.


\begin{figure}[tb]
\RestyleAlgo{boxed}
\LinesNumbered
\begin{functionality}[H]
$S\rightarrow{}\fioprf$: $K$\;
\For{$i=1$ {\bf to} {$\ell$} }{
  $R\rightarrow{}\fioprf:x_i$\;
  $\fioprf\rightarrow{}R: v_i$ such that $(v_1,\ldots,v_\ell)=\iprf_K(x_1\ldots{}x_\ell)$\;
}
\end{functionality}
\caption{Functionality $\fioprf$\label{idealioprf}}
\end{figure}

\begin{definition}[$\proto$]
  An iterated \emph{oblivious} pseudo-random function is an
  $\ell$-round probabilistic protocol $\proto$ between a sender $S$
  with input key $K\in\{0,1\}^{\lambda\cdot\ell}$ and receiver $R$
  with input bit string $x=(x_1\ldots{}x_\ell)\in\{0,1\}^{\ell}$ with
  the following properties.

  \begin{itemize}
   
  \item Protocol $\proto$ realizes the ideal functionality $\fioprf$
    shown in Figure~\ref{idealioprf}. After $\ell$ rounds, $R$ has
    received $(v_1,\ldots,v_\ell)=\iprf_K(x),|v_i|=\lambda$ from
    $\fioprf$, where $\iprf_K$ is an iterated pseudo-random function
    family. Sender $S$ receives nothing from $\fioprf$.
  
  \item For all adversaries $\A$ in the real world, there exists a
    simulator $\myS_R$ in the ideal world such that $R$'s view
    $\mathsf{REAL}_{\proto,\A,R}(x,K)$ in the real world is
    computationally indistinguishable from $R$'s view
    $\mathsf{IDEAL}_{\fioprf,\myS_R(x)}(x,K)$ in the ideal world.

  \item For all adversaries $\A$ in the real world, there exists a
    simulator $\myS_S$ in the ideal world such that $S$'s view
    $\mathsf{REAL}_{\proto,\A,S}(K)$ in the real world is
    computationally indistinguishable from $S$'s view
    $\mathsf{IDEAL}_{\fioprf,\myS_S}(K)$ in the ideal world.
\end{itemize}
\end{definition}

The crucial property of $\fioprf$ in contrast to standard $\oprf$s is
that $R$ can oblivious evaluate a PRF a total of $\ell$ times, but not
on arbitrary inputs. In each round $i$, $R$ submits input $x_i$, but
the PRF will be evaluated using $R$'s previous input
$x_1\ldots{}x_{i-1}$ as a prefix.

\todo{Introduce delegatable PRFs}
