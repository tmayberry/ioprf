\documentclass{article}

\usepackage{amsmath,amsthm,amssymb,parskip,enumitem,url}
\newtheorem{definition}{Definition}

\newcommand{\oprf}[0]{\mathsf{OPRF}}
\newcommand{\getr}[0]{\stackrel{\$}{\leftarrow}}
\newcommand{\enc}[0]{{\mathsf{Enc}}}
\newcommand{\dec}[0]{{\mathsf{Dec}}}
\newcommand{\fixme}[1]{{\bf FIXME:} {\emph{#1}}}
\newcommand{\dash}[0]{{\text -}}
\newcommand{\A}[0]{{\mathcal{A}}}

\newcommand{\ioprf}[0]{\mathsf{i}\mathsf{OPRF}}
\newcommand{\iprf}[0]{\mathsf{i}\mathsf{PRF}}
\newcommand{\proto}[0]{{\pi_{\ioprf}}}
\newcommand{\myS}[0]{{\mathcal{S}}}
\newcommand{\Z}[0]{\mathbb{Z}}


\newcommand{\N}[0]{{\mathbb{N}}}

\begin{document}

\begin{definition}[$\iprf$]
  Consider function family
  $f_K(b_1,\ldots,b_\ell):\{0,1\}^{\lambda\cdot\ell}\times\{0,1\}^{\ell}\rightarrow{}(v_1,\ldots,v_\ell)\in\{0,1\}^{\lambda\cdot\ell}$. For a
  randomly chosen key $K$, observe that range
  $V_\lambda=(v_1,\ldots,v_\ell)$ is a family of random variables (a
  probability ensemble) of length $\ell\cdot\lambda$ bit strings. We
  call function $f_K$ an iterated pseudo-random function family
  $(\iprf_K)$ \emph{iff} for all adversaries $\A$ and for all
  $(b_1,\ldots,b_\ell)\in\{0,1\}^\ell$ there exists a negligible
  function $\epsilon$ such that for sufficiently large $\lambda$
\begin{align*}\forall{}i\in\{1,\ldots,\ell\}:|&Pr[(v_1,\ldots,v_\ell)\leftarrow{}V_\lambda:\A(v_1,\ldots,v_i)=1]-\\&Pr[(v_1,\ldots,v_\ell)\leftarrow{}V_\lambda,u\leftarrow{}U_\lambda:\A(v_1,\ldots,v_{i-1},u)=1]|\\&=\epsilon(\lambda),
  \end{align*}
  where $U_\lambda$ is the random variable describing uniformly random
  bit strings of length $\lambda$. The probabilities are taken over
  the random coins of $\A$ and $K$.

\end{definition}

\begin{definition}[$\proto$]
  Let $\iprf_K$ be an iterated pseudo-random function family.  An
  iterated \emph{oblivious} pseudo-random function is a probabilistic
  protocol $\proto$ between a sender $S$ with input key
  $K\in\{0,1\}^{\lambda\cdot\ell}$ and receiver $R$ with input bits
  $(b_1,\ldots,b_\ell)\in\{0,1\}^{\ell}$ with the following
  properties.

  \begin{itemize}
   
\item Protocol $\proto$ realizes the ideal functionality $\iprf$. It
  outputs
  $(v_1,\ldots,v_\ell)=\iprf_K(b_1,\ldots,b_\ell),|v_i|=\lambda$, to
  $R$ and nothing to $S$.
  
\item For all adversaries $\A$ in the real world, there exists a
  simulator $\myS_R$ in the ideal world such $R$'s view
  $\mathsf{REAL}_{\proto,\A,R}(b_1,\ldots,b_\ell,K)$ in the real world is
  computationally indistinguishable from $R$'s view
  $\mathsf{IDEAL}_{\iprf,\myS_R(b_1,\ldots,b_\ell)}(b_1,\ldots,b_\ell,K)$ in
  the ideal world.

\item
   For all adversaries $\A$ in the real world, there exists a
  simulator $\myS_S$ in the ideal world such $S$'s view
  $\mathsf{REAL}_{\proto,\A,S}(b_1,\ldots,b_\ell,K)$ in the real world is
  computationally indistinguishable from $S$'s view
  $\mathsf{IDEAL}_{\iprf,\myS_S}(b_1,\ldots,b_\ell,K)$ in
  the ideal world.  
\end{itemize}
\end{definition}


\section{Our function $\iprf$}
Let $q$ be the order of the group in which we operate (e.g., points of
an elliptic curve with point addition), and $G$ is a generator.

Initialization: choose $\ell$ random scalars $r_i\getr\Z^*_q$

$\iprf(b_1,\ldots,b_\ell)$: output 
$({r_1^{b_1}}\cdot{}G,\ldots,{r_1^{b_1}\cdots{}r_\ell^{b_\ell}}\cdot{}G)$

\section{Our $\proto$}



\newpage

\section{Old}
\paragraph{Preliminaries} There are two parties $P_1$ and $P_2$ with
input sets $S_1=\{e_{1,1},\ldots,e_{1,n}\}$ and
$S_2=\{e_{2,1},\ldots,e_{2,n}\}$ of elements
$e_{i,j}\in\{0,1\}^\ell$. Both parties have agreed on an Elgamal key
pair $(pk,sk)$ where $sk$ is shared between the two of them.

For security parameter $\lambda$, there exists an oblivious
pseudo-random function
$\oprf:\{0,1\}^\lambda\times\{0,1\}^\ell\rightarrow{}\{0,1\}^\ell$.
In our context, $\oprf$ will be evaluated with $P_1$ being the sender
and $P_2$ being the receiver.

\paragraph{Protocol Overview}
\begin{enumerate}[label={\bf Step {\arabic*}:},leftmargin=*]
\item Party $P_1$ prepares a binary prefix tree of its input set as follows.

  First, $P_1$ generates a random key $k\getr\{0,1\}^\lambda$ for $\oprf$.

  For each $e_{1,i}\in{}S_1$, $P_1$ computes $V_i=\oprf_k(e_{1,i})$
  and builds a prefix tree $T$ with the $V_i$ as keys. Observe that
  each node $N_i$ in $T$ contains tuple
  $(\mathsf{prefix}_i,\mathsf{L}_i,\mathsf{R}_i)$, where
  $\mathsf{prefix}_i$ is node $N_i$'s bit string prefix, and
  $\mathsf{L}_i$ and $\mathsf{R}_i$ are pointers to the $N_i$'s left
  and right children and can therefore be $\bot$.

  $P_1$ stores $T$ in an array $A$, so pointers $\mathsf{L}_i$ and
  $\mathsf{R}_i$ are indices of $A$'s elements. Let the number
  of nodes in $T$ and therewith the number of elements in $A$ be
  $n'$.

  \fixme{Why would $P_1$ have to shuffle $T$?}
  
  Finally, $P_1$ Elgamal encrypts each element $N_i$ of array $A$ to
  $c_i=(\enc_{pk}(\mathsf{prefix}_i),\enc_{pk}(\mathsf{L}_i),\enc_{pk}(\mathsf{R}_i))$
  and sends the $c_i$ to $P_2$.

\item $P_2$ re-encrypts array $A$, i.e., all $c_i$ to $c'_i$, chooses
  a random permutation $\pi:\{1,\ldots,n\}\rightarrow\{1,\ldots,n\}$
  and randomly shuffles the $c'_i$ with $\pi$. Party $P_2$ sends
  resulting array $A'$, the sequence of $c'_{\pi(i)}$, back to $P_1$.

\item For each $e_{2,i}\in{}S_2$, $P_1$ and $P_2$ jointly evaluate
  $\oprf$ such that $P_2$ learns $v'_i=\oprf_k(e_{2,i})$, and $P_1$
  learns nothing.

\item Let $v'_i=b_{i,1}\ldots{}b_{i,\ell}$ be the bit representation
  of $v'_i$. Party $P_2$ fetches data from $P_1$ as follows.

  Party $P_2$ asks $P_1$ to partially decrypt element $\pi(0)$, the
  root, from $A'$. Upon receipt, $P_2$ finalizes decryption of
  $\pi(0)$ to $(\mathsf{prefix},\mathsf{L},\mathsf{R})$.

  They then use bit $b_{i,1}$ to either set
  ${\mathsf{next}}=\mathsf{L}$ or ${\mathsf{next}}=\mathsf{R}$, fetch
  the partial decryption of $\pi({\mathsf{next}})$ from $A'$ and so on.

  Note that $P_2$ never fetches the same element from $A'$ twice. 
  
\end{enumerate}

\fixme{Optimization: do not send back the shuffled array...}

\section{Related Work}
\begin{itemize}
\item Katzenbeisser: \url{https://dl.acm.org/doi/pdf/10.1145/1315245.1315309}:
  privacy-preserving evaluation of a FSM, semi-honest, number of
  states (and therewith this scheme's communication complexity) of the
  FSM is exponential in the edit distance:
  \url{https://store.fmi.uni-sofia.bg/fmi/logic/theses/mitankin-en.pdf}
\item Kerschbaum: \url{http://citeseerx.ist.psu.edu/viewdoc/download?doi=10.1.1.584.3879&rep=rep1&type=pdf}, semi-honest, $\ell^2$ per item
\end{itemize}
  \end{document}