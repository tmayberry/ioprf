\section{Introduction}
Structured encryption allows a data owner to encrypt data arranged in
a data structure and store it on an untrusted
server~\cite{chase2010structured}.  A crucial property of structured
encryption is that the data owner can later compute a special
decryption key for the server which permits the server to decrypt and
parse a well defined component of the data structure, saving the owner
from having to individually download and traverse each element
themselves.  A typical example for structured encryption is data
arranged in a graph encrypted and outsourced to a server, and the
owner computing keys for decryption of sub-graphs~\cite{sherman1}.
Computation of decryption keys is possible despite the owner retaining
only a constant-sized master key. Keyword-searchable encryption can be
implemented as a special-case of structured encryption where the graph
is composed of many linked lists, one for each keyword, containing all
the documents that match that keyword.

\subsection{New Applications} In this paper, we introduce a twist to
the standard application scenario of structured encryption.  A third
party, separate from the data owner and server, which we call the
client, can ask the data owner for permission to retrieve a specific
component of the owner's data structure.  The owner is said to
\emph{delegate} access to this portion of their data to the client.
However, the data owner and client do not trust each other, and the
client does not want to reveal to the owner which part of the data
structure they are interested in.  At the same time, the owner wants
to restrict the client's access to a specific component of their data
structure and might even put constraints on that component, e.g.,
where it begins, how many elements it contains, etc. This new setting
of mutually untrusted data owner, server, and client has several
interesting applications. One can imagine a data owner outsourcing a
medical database with patient records to an untrusted cloud. A
researcher (client) conducting a study gets access to a specific set
of patient records (the component) without leaking which set of
patients is affected to the owner or the database. At the same time,
the researcher is confined within one component of data structure and
cannot arbitrarily browse patient records.

In this work, we focus on tree data structures, but in return offer
more powerful confinement control for the data owner than standard
structured encryption.  In addition to decryption keys enabling
decryption of a sub-tree for the client, the data owner can also
compute keys which enable the client to access only one path, from the
root of the tree to a leaf.  Moreover, the client can ask to decrypt a
path in an iterative, adaptive fashion instead of querying the owner
for the whole path at once. Adaptive queries are necessary to support
iterative scenarios where the client will parse the tree node by node,
obliviously asking the owner to decrypt a single child node in the
tree only after fetching and decrypting the parent node.  For example,
after decrypting one node of a binary tree, the client can
\emph{obliviously} query the owner for the decryption key of
\emph{either} the left \emph{or} right child, depending on the current
node's data content.  At the same time, the data owner wants to ensure
that the client can only ask to decrypt one single node which is a
child of the current node, so that the client is confined to
decrypting exactly one path and cannot arbitrarily ``jump around'' in
the data structure.

 While we later present details on two specific applications for tree data structures, one for
RFID authentication and one for privacy-preserving decision tree
evaluation, we stress that techniques in this paper are general and
useful in other scenarios, too. As soon as data is tree-structured
(XML data, databases using B+ trees or hash maps, hash trees, search
trees, heaps, \ldots) and should be adaptively parsed, our techniques
will be required. We also note that our techniques can be used to
realize maliciously secure, adaptive $k$-out-of-$n$ oblivious
transfer~\cite{adaptiveot} and oblivious keyword search~\cite{oks}. We
discuss applications in more detail in Section~\ref{sec:applications}.

\subsection{Technical Challenges} A straightforward approach to
providing adaptive queries might be for the data owner to apply an
Oblivious Pseudo-Random Function (OPRF) as the PRF to encrypt nodes.
For a tree of height $\ell$, owner and client run $\ell$ instances of
the OPRF such that the client always learns the key for the next node
on the path they are interested in, and the owner learns nothing. To
actually fetch a node from the server in an oblivious fashion, the
client could employ standard PIR or OT protocols. However, this
approach is only secure against semi-honest clients that stick to the
rule of asking for the decryption key of one child node of the current
node. The difficulty lies in making parsing the tree structure secure
against a fully-malicious client without reverting to general, yet
expensive techniques like maliciously secure two-party computation and
general Zero-Knowledge (ZK) proofs.

\subsection{Our contributions} Consequently, we introduce the notion
of iterative Oblivious Pseudo-Random Functions ($\ioprf$s) and the
first candidate constructions. An $\ioprf$ is an $\ell$ round two
party protocol between a sender and a receiver. The intuition behind
$\ioprf$s is that the receiver adaptively queries $\ell$ input bits
$x_i$ in $\ell$ rounds such that in the end they learn outputs
$\text{PRF}_K(x_1),\ldots,\text{PRF}_K(x_1\ldots{}x_\ell)$ for key $K$
chosen by the sender, and the sender learns nothing. If such an
$\ioprf$ is used to encrypt the nodes, then fetching a wrong node is
useless for the client, as they cannot decrypt it anyways.

Our new candidate $\ioprf$ construction is based on a careful
adaptation of the PRF by \citet{prf}. We first augment the
\citeauthor{prf} PRF to become an iterative Pseudo-Random Function
($\iprf$) which has the property that, for input strings with the same
prefix, its generated output also shares the same prefix.
\ignore{
  As a
warm-up, we then use a similar trick as \citet{oprf} to convert the
$\iprf$ to an $\ioprf$. This first $\ioprf$ is OT-based and elegant,
yet it only offers one-sided security~\cite{efficient2pc} against a
malicious receiver and semi-honest sender.
}%ignore
We then present our main
construction, an $\ioprf$ which is secure against malicious sender and
malicious receiver. We achieve malicious security by using efficient
ZK proofs for DH-based statements over elliptic curves and
avoid costly maliciously secure oblivious transfer (OT).  We implement
and benchmark our new $\ioprf$ construction to show its practicality
and efficiency.


\ignore{We then show how to integrate an $\ioprf$ into further important
applications, such as RFID authentication and privacy-preserving
decision tree evaluation. We believe our construction is an important
step towards protocols that provide security to both parties.}

\vskip0.5eX\noindent{}In summary, the {technical highlights} of this
paper are:
\begin{itemize}

\item The definition of the new cryptographic primitives of $\iprf$
and $\ioprf$ which extends repeated $\oprf$ constructions with
security constraints on the client's input.

\item A candidate construction which is efficient and provably secure
  under the Decisional Diffie-Hellman assumption in the standard
  model.

  \item To show its practicality, we implement our construction and
  evaluate its performance. The implementation is available for
  download~\cite{srcode}.

\item The integration of our primitive into several example
applications, such as RFID authentication and privacy-preserving
decision tree evaluation.

\end{itemize}

