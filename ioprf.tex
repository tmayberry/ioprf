\subsection{DH-based $\proto$ Construction}
NOTE: for one-sided security, we use the OT-based solution.
\fixme{Switch to additive notation}


\subsubsection{Preliminaries}
Let there be three generators $g_1,g_2,g_3$ of prime order $p$ DDH
group $\G$. Nobody knows the discrete log of one generator
$g_i\in\{g_1,g_2,g_3\}$ to the basis of another generator
$g_{j\neq{}i}$. \fixme{They can come from the CRS.}

\paragraph{Elgamal Encryption}
We will use additive Elgamal encryption with private keys $sk\in\Z_p$
and public keys $pk=g_1^{sk}$. For either $g=g_2$ or $g=g_3$, the
ciphertext $c$ to encrypt $m\in\Z_p$ is
$c=(c[0],c[1])=(g_1^r,pk^r\cdot{}g^m)\leftarrow\enc_{pk,g}(m)$, where
$r\getr\Z_p$.

\paragraph{Pedersen Commitments}
A Pedersen commitment $\com_{g}(m)\in\G$ to message $m\in\Z_p$ is defined as
$\com_{g}(m)=g_1^r\cdot{}g^m$, where $r\getr\Z_p$, and $g\neq{}g_1$ is
a generator of $\G$.  To open $\com_{g}(m)$, $(m,r)$ is revealed.


\subsubsection{$\proto$ Initialization}
The sender in $\proto$ knows the $\iprf$'s secret key
$K=(\alpha_1,\ldots,\alpha_\ell,\beta_1,\ldots,\beta_\ell)$ as before. For some input
string $x=(x_1,\ldots{},x_\ell)$, we define the output of $\proto$ for
the receiver as
$\iprf_K(x)=g_2^{\prod_{i=1}^{\ell}(\alpha_ix_i+\beta_i(1-x_i))}$.


The receiver computes $V_0 \leftarrow\enc_{pk,g_2}(1)$ and
$D_0\leftarrow\enc_{pk,g_3}(1)$, sends them to the sender and proves
that these are encryptions of $1$ with respect to bases $g_2$ and
$g_3$ (see below for details). The sender computes Pedersen
commitments
$(\com_{g_2}(\alpha_1),\ldots,\com_{g_2}(\alpha_\ell),\com_{g_2}(\beta_1),\ldots,\com_{g_2}(\beta_\ell))$,
sends them to the receiver, and proves knowledge of plaintexts in ZK
(see below).

\subsubsection{$\proto$ Iterative Processing in $\ell$ Rounds}
In round $i\in\{1,\ldots,\ell\}$, for sender's input bit $x_i$:
\begin{enumerate}
\item {\bf Shuffling:}
\begin{enumerate}
\item For input bit $x_i$, the receiver computes Pedersen commitment
  $\com_{g_2}(x_i)$ and proves that $x_i\in\{0,1\}$ (see
  below). Similarly, the receiver computes $\com_{g_2}(1-x_i)$ and
  proves knowledge of $(1-x_i)$ (see below). Finally, the receiver
  proves that the sum of plaintexts behind $\com_{g_2}(x_i)$ and
  $\com_{g_2}(1-x_i)$ equals $1$ (see below).


\item  The receiver chooses $r,r',r'',r'''\getr\Z_p$ and computes Elgamal ciphertexts
  \begin{align*}
    &c_i=(g_1^r\cdot{}V_{i-1}[0]^{x_i},pk^{r}\cdot{}V_{i-1}[1]^{x_i})
    \\&c'_i=(g_1^{r'}\cdot{}V_{i-1}[0]^{1-x_i},pk^{r'}\cdot{}V_{i-1}[1]^{1-x_i})
    \\&d_i=(g_1^{r''}\cdot{}D_{i-1}[0]^{x_i},pk^{r''}\cdot{}D_{i-1}[1]^{x_i})
    \\&d'_i=(g_1^{r'''}\cdot{}D_{i-1}[0]^{1-x_i},pk^{r'''}\cdot{}D_{i-1}[1]^{1-x_i})%\text{ and}
  \end{align*}
  and sends $(c_i,c'_i,d_i,d'_i)$ to the sender.

\item The receiver proves correctness of the above computations in
  ZK. Specifically, $(c_i,c'_i,d_i,d'_i)$ result from correct
  exponentiation with $x_i$ (or $1-x_i$) from $\com_{g_2}(x_i)$ (or
  $\com_{g_2}(1-x_i)$), and multiplication with a random power of
  $g_1$ and $pk$, i.e., re-randomization (homomorphic addition of
  encryption of $0$).  See below for technical details.

   Both parties can now compute
    \begin{align*}&T_i=(c_i[0]\cdot{}d'_i[0],c_i[1]\cdot{}d'_i[1])
    \\&U_i=(c'_i[0]\cdot{}d_i[0],c'_i[1]\cdot{}d_i[1]).
    \end{align*}
   
  \end{enumerate}

\item {\bf PRF:} The sender computes Elgamal ciphertexts
  \begin{align*}
&X_i=(T_i[0]^{\alpha_i},T_i[1]^{\alpha_i})
\\&Y_i=(U_i[0]^{\beta_i},U_i[1]^{\beta_i}),
  \end{align*}
sends $(X_i,Y_i)$ to the receiver and proves correct exponentiation
(multiplication of plaintexts) in ZK (see below).

\item {\bf Shuffling back:}
  For $r,r',r'',r'''\getr\Z_p$, the receiver computes
  \begin{align*}
    P_i&=(g_1^r\cdot{}X_i[0]^{x_i},pk^r\cdot{}X_i[1]^{x_i})
    \\P'_i&=(g_1^{r'}\cdot{}X_i[0]^{1-x_i},pk^{r'}\cdot{}X_i[1]^{1-x_i})
   \\Q_i&=(g_1^{r''}\cdot{}Y_i[0]^{x_i},pk^{r''}\cdot{}Y_i[1]^{x_i})
   \\Q'_i&=(g_1^{r'''}\cdot{}Y_i[0]^{1-x_i},pk^{r'''}\cdot{}Y_i[1]^{1-x_i})
  \end{align*} 
  and sends $(P_i,P'_i,Q_i,Q'_i)$ together with ZK proofs of correct
  computation (see below) to the sender.

  Both sender and receiver compute
  $V_i=(P_i[0]\cdot{}Q'_i[0],P_i[1]\cdot{}Q'_i[1])$ and
  $D_i=(P'_i[0]\cdot{}Q_i[0],P'_i[1]\cdot{}Q_i[1])$.
  
\end{enumerate}

After $\ell$ rounds, the receiver computes output
$\iprf_K(x)=\frac{V_\ell[1]}{V_{\ell}[0]^{sk}}$.


\subsection{Security Analysis}
\subsubsection{Dual-iPRF is still a PRF}

\begin{construction} 
Define the function ensemble $F = \{F_n\}_{n\in N}$.  For every $n$, a key of a function $F_n$ is a tuple, $\langle P,Q,g,\vec{a}\rangle$, 
where $P$ is an $n$-bit prime, $Q$ a prime divisor of $P-1$, $g$ an element of order $Q$ in $\mathbb{Z}_{p}^*$ and $\vec{a}=\langle 
a_0,a_1, \ldots , a_n \rangle$ a uniformly random sequence of $n+1$ elements of $\mathbb{Z}_Q$.  For any $n$-bit input, $x=x_1 x_2 \ldots x_n$, the 
function $f_{P,Q,g,\vec{a}}$ is defined by:
 $$f_{P,Q,g,\vec{a}}(x) = (g^{a_0})^{\prod_{x_i=1}a_i}$$
\end{construction}

\begin{theorem}
\label{theorem:naor}
If the DDH-Assumption holds, then for every probabilistic polynomial-time oracle machine $\mathcal{M}$, every constant $\alpha > 0$, and for all byt a finite number of $n$'s
$$| Pr[\mathcal{M}^{f_{P,Q,g,\vec{a}}}(P,Q,g)=1] - Pr[\mathcal{M}^{R_{P,Q,g}}(P,Q,g) = 1]| < \frac{1}{n^\alpha} $$
where in the first probability, $f_{P,Q,g,\vec{a}}$ is distributed according to $F_n$ and in the second probability the distribution of $R_{P,Q,g}$ is uniformly chosen in the set of functions with the domain $\{0,1\}^n$ and range $\langle g\rangle$.
\end{theorem}

\begin{construction}
\label{const:newprf}
Define the function ensemble $F' = \{F'_n\}_{n\in N}$.  For every $n$, a key of a function $F'_n$ is a tuple, $\langle P,Q,g,\vec{a},\vec{b}\rangle$, 
where $P$ is an $n$-bit prime, $Q$ a prime divisor of $P-1$, $g$ an element of order $Q$ in $\mathbb{Z}_{p}^*$ and $\vec{a}=\langle 
a_0,a_1, \ldots , a_n \rangle$ and $\vec{b}=\langle 
b_0,b_1, \ldots ,b_n \rangle$, uniformly random sequences of $n+1$ elements of $\mathbb{Z}_Q$.  For any $n$-bit input, $x=x_1 x_2 \ldots x_n$, the 
function $f'_{P,Q,g,\vec{a},\vec{b}}$ is defined by:
 $$f'_{P,Q,g,\vec{a},\vec{b}}(x) = (g^{a_0})^{\prod_{x_i=1}a_i \prod_{x_i=0}b_i}$$
 \end{construction}
 
\begin{theorem}
\label{theorem:newprf}

If the DDH-Assumption holds, then for every probabilistic polynomial-time oracle machine $\mathcal{M}$, every constant $\alpha > 0$, and for all byt a finite number of $n$'s

$$| Pr[\mathcal{M}^{f'_{P,Q,g,\vec{a}}}(P,Q,g)=1] - Pr[\mathcal{M}^{R_{P,Q,g}}(P,Q,g) = 1]| < \frac{1}{n^\alpha} $$

where in the first probability, $f'_{P,Q,g,\vec{a}}$ is distributed according to $F'_n$ and in the second probability the distribution of $R_{P,Q,g}$ is uniformly chosen in the set of functions with the domain $\{0,1\}^n$ and range $\langle g\rangle$.
\end{theorem}

\begin{proof}
We prove this by reduction, showing that if $\mathcal{M}$ exists that can distinguish between $f'$ and a random function, then we can build $\mathcal{M}'$ that can distinguish between $f$ and a random function.

First, assume that $\mathcal{M}$ exists that can violate the inequality from Theorem \ref{theorem:newprf}.  We create $\mathcal{M'}$ as follows.  First, $\mathcal{M}'$ creates and stores a uniformly random vector $\vec{b}$ as in the Construction \ref{const:newprf}.  $\mathcal{M}'$ then runs $\mathcal{M}$ as a subroutine.  Each time $\mathcal{M}$ queries the oracle for an evaluation on input $y$, $\mathcal{M}'$ does the following:

\begin{enumerate}
\item Query his own oracle for $z = f(y)$
\item Calculate $z' = z^{\prod_{y_i = 0} b_i}$
\item Returns $z'$ to $\mathcal{M}$
\end{enumerate}

$\mathcal{M}'$ outputs the same as $\mathcal{M}$.  If  $\mathcal{M}'$ is interacting with the function $f$, then the responses he gives to $\mathcal{M}$ will be identical to the real function $f'$, due to him being able to add in the extra $b$ components to the result.  If $\mathcal{M}'$ is interacting with a random function, then the responses he gives will be distributed identically to a random function.  Therefore, if $\mathcal{M}$ has a distinguishing advantage, so will $\mathcal{M}'$.
\end{proof}
 
\subsubsection{Proof}
\todo{We prove in the hybrid model and make use of several ZK hybrids
  introduced below. Description is HVZK for ease of reading, but we
  convert to malicious verifier using CRS model introduced below,
  too. For ease of reading, we present ZK proofs, but convert to ZKPoK
  using a generic transformation introduced below, too.}

\paragraph{Witness Extraction for Pedersen Commitments}
Pedersen commitments are trapdoor commitments which means that a party
knowing a trapdoor $t$ can open a commitment $\com_g(\cdot)$ to any
plaintext they want (equivocable).  We use this property for witness
extraction in three-move ZK proofs as follows.

Before starting the actual ZK proof by the first message from the
prover to the verifier, we send the following two messages:
\begin{enumerate}
\item Prover $P$ sends to verifier $V$: $g=g_1^\rho$ for random
  $\rho\getr\Z_p$. Verifier $V$ will use this $g$ for the computation
  of their commitment.
  \item $V$ computes and sends back commitment $\com_g(e)$ for their
    challenge $e\in\Z_p$.
\end{enumerate}

The ZK proof then continues as usual with the difference that $V$
opens $e$ as their challenge together with randomness $r$ used to
compute the commitment, and $P$ verifies whether $(e,r)$ match
$\com_g(e)$.  After $P$ has sent their final message, $P$ also reveals
$\rho$ to $V$. Only if both is correct, the last ZK proof message
matches  challenge $e$, and $\rho$ matches $g=g_1^\rho$, $V$ accepts.

This setup enables a simulator $\myS$ to extract the witness from
$P$. After receiving trapdoor $\rho$ from $V$, $\myS$ rewinds $P$ until
after the point were $V$ sends $\com_g(e)$ to $P$. Knowing trapdoor
$\rho$, $\myS$ can open $\com_g(e)$ to any $e'\neq{}e$ they want by
solving $r+\rho\cdot{}e=r'+\rho\cdot{}e'$ for $r'$, i.e., they compute
$r'=r+\rho\cdot{}(e-e')$. Running two executions of the ZK proof with the
same input and messages from $P$, but different challenges extracts
the witness of the ZK proof. Details on which $e$ to send in each
execution depend on the exact three-move ZK proof, but are typically
obvious. We will show an example later.

\paragraph{Zero Knowledge (instead of Honest-Verifier ZK)}
To ease readability, all  ZK proofs here are described in their
honest-verifier ZK version. To make these proofs zero-knowledge
instead, we replace the first message of the regular three-move protocol by a
Pedersen commitment from prover $P$ to the first message. The protocol
then continues with $V$ sending their challenge $e$, and finally $P$
sending the last message of the regular protocol together with opening
the commitment. Verifier $V$ accepts, if the commitment matches and
the last message of the regular protocol.

In the CRS model, this technique allows a simulator $\myS$ to learn
$e$ before sending the first message of the three-move
protocol. Before interacting with the verifier, $\myS$ chooses
$\rho\getr\Z_p$ and places $g'=g_1^\rho$ on the CRS. Parties agree
that $P$'s commitment has to be using a Pedersen $\com_g(\cdot)$
commitment, where $g$ is from the CRS. The commitment of $\myS$ at the
beginning is then just a commitment to some arbitrary $r$. As soon as
$\myS$ has received $e$, they can fake the last message of the
three-move protocol and open the initial commitment accordingly.  See
\citet{crs} for details.

Observe that witness extraction and zero knowledge can be combined in
the natural way: first, $P$ sends $g=g_1^{\rho_1}$ to $V$ who replies
with $\com_g(e)$, then $P$ sends commitment $\com_{g'}(r)$, where $g'$
is from the CRS ($g'=g_1^{\rho_2}$, $\rho_2$ known by $\myS$ in the
proof), then $V$ opens $\com_g(e)$, and finally $P$ opens
$\com_{g'}(r)$ and sends the last message of the three-move protocol.

\subsubsection{ZK Proofs for Exponents}
\paragraph{Proof for Knowledge of Plaintext}
For $\com_{g}(m)=g_1^r\cdot{}g^m$,  prover $P$ can prove that they know
$m$.

\begin{enumerate}
\item $P$ sends $t=g_1^{\rho_1}\cdot{}g^{\rho_2}$ for
  $\rho_1,\rho_2\getr\Z_p$ to $V$.
\item $V$ sends $e\getr\Z_p$ to $P$.
  \item $P$ sends $s_1=\rho_1+e\cdot{}r$ and $s_2=\rho_2+e\cdot{}m$ to
    $V$.
    \item $V$ checks whether $g_1^{s_1}\cdot{}g^{s_2}\sr\com_{g}(m)^e\cdot{}t$.
\end{enumerate}


\paragraph{Proof of DLOG Equivalence/DDH tuple/Proof of Encryption of $0$}
You can prove that tuple $(a=g_1,b=g_1^r,c=g_1^x,d=g_1^{xr})$ is a DDH
tuple, i.e., you show that $\log_{a}{b}=\log_c{d}$.

\begin{enumerate}
\item $P$ sends $(t_1=a^{\rho},t_2=c^{\rho})$ for $\rho\getr\Z_p$ to $V$.
  \item $V$ sends $e\getr\Z_p$ to $P$.
  \item $P$ sends $s=\rho+e\cdot{}r$ to $V$.
    \item $V$ accepts if $a^s=b^e\cdot{}t_1$ and $c^s=d^e\cdot{}t_2$.
\end{enumerate}

Observe that this technique also proves that an encryption
$c=(c[0],c[1])=(g_1^r,pk^r\cdot{}g)\leftarrow\enc_{pk,g}(0)$ is an
encryption of $0$ (with respect to base $g$), as $(g_1,c[0],pk,c[1])$
is a DDH tuple.

\paragraph{Proof of Plaintext Equivalence}
Let $c_1=(c_1[0],c_1[1],)=(g_1^{r_1},pk^{r_1}\cdot{}g^m)$ and
$c_2=(c_2[0],c_2[1],)=(g_1^{r_2},pk^{r_2}\cdot{}g^m)$ be two
encryptions from $\enc_{pk,g}(m)$. To prove plaintext equivalence of
these two ciphertexts, the prover shows that
$(g_1,\frac{c_1[0]}{c_2[0]},pk,\frac{c_1[1]}{c_2[1]})$ is a DDH tuple.

To prove that some ciphertext $c_1$ encrypts a plaintext $m$ with
respect to base $g$, a simple trick for the prover is to compute
another encryption $c_2$ of $m$ with respect to base $g$, show
plaintext equivalence, and then open randomness of $c_2$.


\paragraph{Proof of Plaintext Bit}
For a commitment $\com_g(x_i)$, prover $P$ can prove that
$x_i\in\{0,1\}$, i.e., a bit. This is an application of the
\emph{one-out-of-two} (OR) technique~\cite{ooot}. Essentially, $P$
proves that $\com_g(x_i)=g_1^{r}\cdot{}g=c_1$ or
$\com_g(x_i)=g_1^{r'}=c_2$ by proving that they know an $r$ or $r'$
matching $c_1$ or $c_2$. The trick is that $P$ chooses $e_1$ and $e_2$
such that, for the verifier's challenge $e$, we have
$e=e_1+e_2$. Prover $P$ proves knowledge of $r$ for $c_1$ using
challenge $e_1$ and knowledge of $r'$ for $c_2$ using challenge
$e_2$. Thus, $P$ can choose either $e_1$ or $e_2$ before sending their
first message of the ZK proof and cheat in one proof. Without loss of
generality, let $x_i=1$, so $P$ will cheat for the proof of
$c_2$. This works as follows.

\begin{enumerate}
\item $P$ sends $t_1=g_1^\rho\cdot{}g$ and
  $t_2=c_2^{-e_2}\cdot{}g_1^{\rho'}$, for $\rho,\rho'\getr\Z_p$, to
  $V$.
\item $V$ sends $e\getr\Z_p$ to $P$.
  \item $P$ sends $e_1,e_2,s_1=\rho+e_1\cdot{}r$, and $s_2=\rho'$ to $V$.

\item $V$ checks $e\sr{}e_1+e_2$, $g_1^{s_1}\cdot{}g^{e+1}\sr{}c_1^e\cdot{}t_1$ and $g_1^{s_2}\sr{}c_2^e\cdot{}t_2$.
\end{enumerate}

\subsubsection{Proofs for Arithmetic with Pedersen Commitments}
We can do simple arithmetic on Pedersen Commitments.
\begin{itemize}
\item Addition: given $\com_g(a)$ and $\com_g(b)$, everybody can
  compute and thus verify commitment
  $\com_g(c)=\com_g(a)\cdot{}\com_g(b)$ which commits to
  $c=a+b$. Obviously, no other party than the one originally computing
  $\com_g(a)$ and $\com_g(b)$ can open $\com_g(c)$, but all parties
  know that $\com_g(c)$ is a commitment to $c=a+b$
  
\item Multiplication: a party committing
  \begin{align*}
  \com_g(a)=g_1^{r_a}\cdot{}g^a, \com_g(b)=g_1^{r_b}\cdot{}g^b,
  \com_g(c)=g_1^{r_c}\cdot{}g^{a\cdot{}b}
  \end{align*}
  can prove in ZK that
  $\com_g(c)$ commits to the product of the messages committed in
  $\com_g(a)$ and $\com_g(b)$.

  The trick is to rewrite
  $\com_g(c)=g_1^{r_c-a\cdot{}r_b}\cdot\com_g(b)^{a}$ and then prove
  that all commitments are well formed, and $\com_g(c)$ uses the same
  exponent $a$ as $\com_g(a)$, but with basis $\com_g(b)$ instead of
  $g$. Specifically,
  \begin{enumerate}
  \item $P$ computes and sends
    \begin{align*}
      t_1=g_1^{\rho_1}\cdot{}g^{\rho_2},
      t_2=g_1^{\rho_3}\cdot{}g^{\rho_4},
      t_3=g_1^{\rho_5}\cdot{}\com_g(b)^{\rho_2}
      \end{align*}
      for $\rho_i\getr\Z_p$. Observe that the same randomness $\rho_2$
      is used for the same witness $a$.
    \item $V$ replies by sending challenge $e\getr\Z_p$.
    \item $P$ sends
      \begin{align*}
        s_1&=\rho_1+e\cdot{}r_a,s_2=\rho_2+e\cdot{}a,s_3=\rho_3+e\cdot{}r_b,s_4=\rho_4+e\cdot{}b,\\s_5&=\rho_5+e\cdot{}(r_c-a\cdot{}r_b).
        \end{align*}
    \item $V$ checks
      \begin{align*}
        g_1^{s_1}\cdot{}g^{s_2}\sr{}\com_g(a)^e, g_1^{s_3}\cdot{}g^{s_4}\sr{}\com_g(b)^e, g_1^{s_5}\cdot{}\com_g(b)^{s_2}\sr{}\com_g(c)^e\cdot{}t_3.
        \end{align*}
\end{enumerate}

  \item Multiplication with group elements: given two elements $(A,B)$
    of our DDH group, such as an Elgamal ciphertext tuple, a party can
    commit to a value $x$ with $\com_g(x) =g_1^r\cdot{}g^x$ and prove
    that $(C=A^x,D=B^x)$ are the result of exponentiation with $x$.
    \begin{enumerate}
      \item $P$ sends $t_1=A^{\rho_1},t_2=B^{\rho_1},
        t_3=g_1^{\rho_2}\cdot{}g^{\rho_1}$, for randomly chosen
        $\rho_i\getr\Z_p$, to $V$.

      \item $V$ sends challenge $e\getr\Z_p$.

      \item $P$ sends $s_1=\rho_1+e\cdot{}x,s_2=\rho_2+e\cdot{}r$.
        \item $V$ checks $A^{s_1}\sr{}C^e\cdot{}t_1$,
          $B^{s_1}\sr{}D^e\cdot{}t_2$, and
          $g_1^{s_2}\cdot{}g^{s_1}\sr{}\com_g(x)^e\cdot{}t_3$.
          
      \end{enumerate}   

    \item Simple combinations: you can efficiently combine arithmetic
      operations. For example, the following proves correctness of
      exponentiation of two group elements with a committed value $x$
      and multiplying one by $g_1^{r'}$ and $pk^{r'}$. So, you can
      prove correct scalar multiplication of an Elgamal ciphertext by
      a previously committed $x$ and subsequent re-randomization of
      the result (addition of Elgamal encryption of $0$).

      Given two group elements $(A,B)$ and commitment
      $\com_g(x) =g_1^{r}\cdot{}g^x$, prove correctness that
      $(C=g_1^{r'}\cdot{}A^x,D=pk^{r'}\cdot{}B^x)$ are the result of
      exponentiation with $x$ and multiplying with $g^{r'}$ and
      $pk^{r'}$, $r'\getr\Z_p$, known to the prover.

\begin{enumerate}
  \item $P$ sends $t_1=g_1^{\rho_1}\cdot{}A^{\rho_2},t_2=pk^{\rho_1}\cdot{}B^{\rho_2},t_3=g_1^{\rho_3}\cdot{}g^{\rho_2}$ to $V$.
  \item $V$ sends $e\getr\Z_p$ to $P$.
    \item $P$ sends $s_1=\rho_1+e\cdot{}r'$, $s_2=\rho_2+e\cdot{}x$,
      and $s_3=\rho_3+e\cdot{}r$ to $V$.
\item $V$ check whether $g_1^{s_1}\cdot{}A^{s_2}\sr{}C^e\cdot{}t_1$,
  $pk^{s_1}\cdot{}B^{s_2}\sr{}D^e\cdot{}t_2$, and
  $g_1^{s_3}\cdot{}g^{s_2}\sr{}\com_g(x)^e\cdot{}t_3$.
\end{enumerate}

\item Sum of plaintexts equals $1$: for commitments
  $\com_{g}(x)=g_1^r\cdot{}g^x$ and
  $\com_{g}(1-x)=g_1^{r'}\cdot{}g^{1-x}$, the prover can show that the sum of
  plaintexts equals $1$.

  \begin{enumerate}
  \item $P$ and $V$ compute
    $\com_{g}(1)=\com_{g}(x)\cdot{}\com_{g}(1-x)=g_1^{r+r'}\cdot{}g$.
\item $P$ proves that $\com_{g}(1)$ is a commitment to $1$ (see
  above).
  \end{enumerate}
\end{itemize}




\ignore{
\newpage
\section{Old}
Let there be 2 generators $G_1$ and $G_2$ of some DDH group. The
sender has secret key $K=(a_1,\ldots,a_\ell,b_1,\ldots,b_\ell)$ as
before. For some input $x=x_1\ldots{}x_\ell$, we define
$\ioprf_K(x)=\prod_{i=1}^{\ell}(a_ix_i+b_i(1-x_i))\cdot{}G_1$.

\paragraph{Init}
The receiver sets $V_0 = G_1$ and $D_0 = G_2$, commits to $V_0$ and
$D_0$, but also sends randomness used for commitments to the
sender. Therewith, the sender knows that the commitments are really
containing $G_1$ and $G_2$. The sender commits to $a_i,b_i$.

\paragraph{Iterative Processing $\ell$ rounds}
In round $i\in\{1,\ldots,\ell\}$, for sender's input bit $x_i$:

\begin{enumerate}
\item {\bf Blinding:} The receiver computes $V'_i = t_i\cdot{}V_{i-1}$ and
  $D'_i=t_i \cdot D_{i-1}$ for a randomly chosen $t_i$. The receiver
  commits to $t_i$, $V'_i$, and $D'_i$ and proves the following two
  Groth-Sahai (GS) equations to the sender
\begin{align}
  \myO &= t_i \cdot V_{i-1} - 1 \cdot V'_i \\
  \myO &= t_i \cdot D_{i-1} - 1\cdot{}D'_i
\end{align}

      (Constants are $0$ and $-1$, and variables are $t_i, V_i, D_i,
      V'_i, D'_i$.)

      These two equations prove correctness of commitments $V'_i$ and
      $D'_i$. Observe that the receiver does not send $V'_i,D'_i$ to
      the sender, but only their commitments.

    \item {\bf Shuffling:} For input bit $x_i$, the receiver computes
      \begin{align}
        R_i = x_i \cdot V'_i + (1-x_i) \cdot D'_i
        \\S_i = (1-x_i) \cdot V'_i + x_i \cdot D'_i.
      \end{align}

      The receiver
      commits to $x_i$ (and has to prove that $x_i$ is either a 0 or
      1, see below). The receiver sends both $R_i$ and $S_i$ proves them as GS equations.

      (Constants are $R_i,S_i$, and
      the -1 (there are some tricks), and variables are $x_i, V'_i$, and
      $D'_i$.)

      Therewith, the receiver has given a random shuffle (depending on
      $x_i$) of $V_i$ and $D_i$ to the sender. The sender does not
      know which of $R_i$ and $S_i$ is $V_i$ or $D_i$.
      
    \item The sender computes \begin{align}
                                X_i &= a_i \cdot R_i\\
                                Y_i &= b_i \cdot S_i
                                \end{align}

                                and sends $X_i,Y_i$ back and proves
                                them with two GS equations.

                              \item The receiver commits to $t_i^{-1}$
                                and proves $t_i^{-1}\cdot{}t_i = 1$. This is a quadratic equation.


                              \item The receiver computes
\begin{align}
  V_{i} = t_i^{-1} \cdot X_i\\
  D_{i} = t_i^{-1} \cdot Y_i,
\end{align}
and proves them.


\end{enumerate}

Proving that $x_i$ is a bit is a quadratic equation:
$x_i \cdot (1-x_i) = 0$.


It is important that $G_1$ and $G_2$ are random. Specifically, the
receiver does not know the elliptic curve DLOG of $G_2$ to basis
$G_1$, i.e., $\log_{G_1}{G_2}$. Both $G_1$ and $G_2$ could be part of
the CRS.
}%ignore
