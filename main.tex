\documentclass[sigconf]{acmart}
\usepackage{xfrac,amsmath,amsthm,enumitem,url}
%\usepackage{amssymb}
%\usepackage[numbers,sort&compress]{natbib}
\usepackage[linesnumbered]{algorithm2e}


%\newtheorem{definition}{Definition}
%\newtheorem{theorem}{Theorem}

\newcounter{construction}
\newenvironment{construction}[1][]{\refstepcounter{construction}\par\medskip
   \noindent \textbf{Construction~\theconstruction} (#1)\textbf{.} \rmfamily}{\medskip}

\newcommand{\oprf}[0]{\mathsf{OPRF}}
\newcommand{\getr}[0]{\stackrel{\$}{\leftarrow}}
\newcommand{\enc}[0]{{\mathsf{Enc}}}
\newcommand{\dec}[0]{{\mathsf{Dec}}}
\newcommand{\fixme}[1]{{\bf{}\ [FIXME:} {\emph{#1}} {\bf{}]}}
\newcommand{\todo}[1]{{\bf ToDo:} {{\bf #1}}}
\newcommand{\dash}[0]{{\text -}}
\newcommand{\A}[0]{{\mathcal{A}}}
\newcommand{\D}[0]{{\mathcal{D}}}
\newcommand{\myO}[0]{\mathcal{O}}
\newcommand{\ioprf}[0]{\mathsf{i}\mathsf{OPRF}}
\newcommand{\hmac}[0]{\mathsf{HMAC}}
\newcommand{\iprf}[0]{\mathsf{i}\mathsf{PRF}}
\newcommand{\fioprf}[0]{\mathcal{F}_{\mathsf{i}\mathsf{OPRF}}}
\newcommand{\ot}[0]{\mathsf{OT}}
\newcommand{\proto}[0]{{\pi_{\ioprf}}}
\newcommand{\myS}[0]{{\mathsf{Sim}}}
\newcommand{\myF}[0]{{\mathcal{F}}}
\newcommand{\Z}[0]{\mathbb{Z}}
\newcommand{\myG}[0]{\mathbb{G}}
\newcommand{\Hide}[1]{}

\newcommand{\prg}[0]{\mathsf{PRG}}
\newcommand{\seed}[0]{\mathsf{seed}}
\newcommand{\myroot}[0]{\mathsf{ROOT}}

\newcommand{\trace}[0]{\mathsf{Trace}}
\newcommand{\crs}[0]{\mathsf{CRS}}
\newcommand{\com}[0]{\mathsf{com}}
\newcommand{\sr}[0]{{\stackrel{?}{=}}}

\let\ignore\Hide

\newcommand{\N}[0]{{\mathbb{N}}}

\makeatletter
\def\old@comma{,}
\catcode`\,=13
\def,{%
  \ifmmode%
    \old@comma\discretionary{}{}{}%
  \else%
    \old@comma%
  \fi%
}
\makeatother


\newenvironment{functionality}[1][htb]
  {\renewcommand{\algorithmcfname}{Functionality}% Update algorithm name
   \begin{algorithm}[#1]%
  }{\end{algorithm}}

\makeatletter
\newcommand{\removelatexerror}{\let\@latex@error\@gobble}
\makeatother

\AtBeginDocument{%
  \providecommand\BibTeX{{%
    \normalfont B\kern-0.5em{\scshape i\kern-0.25em b}\kern-0.8em\TeX}}}


\copyrightyear{2022}
\acmYear{2022}
\setcopyright{licensedusgovmixed}\acmConference[ASIA CCS '22]{Proceedings of the 2022 ACM Asia Conference on Computer and Communications Security}{May 30-June 3, 2022}{Nagasaki, Japan}
\acmBooktitle{Proceedings of the 2022 ACM Asia Conference on Computer and Communications Security (ASIA CCS '22), May 30-June 3, 2022, Nagasaki, Japan}
\acmPrice{15.00}
\acmDOI{10.1145/3488932.3517403}
\acmISBN{978-1-4503-9140-5/22/05}

\settopmatter{printacmref=true}


\begin{CCSXML}
<ccs2012>
<concept>
<concept_id>10002978.10002979</concept_id>
<concept_desc>Security and privacy~Cryptography</concept_desc>
<concept_significance>500</concept_significance>
</concept>
<concept>
<concept_id>10002978.10002991.10002995</concept_id>
<concept_desc>Security and privacy~Privacy-preserving protocols</concept_desc>
<concept_significance>500</concept_significance>
</concept>
<concept>
<concept_id>10003752.10003777.10003788</concept_id>
<concept_desc>Theory of computation~Cryptographic primitives</concept_desc>
<concept_significance>500</concept_significance>
</concept>
<concept>
<concept_id>10003752.10003777.10003789</concept_id>
<concept_desc>Theory of computation~Cryptographic protocols</concept_desc>
<concept_significance>500</concept_significance>
</concept>
</ccs2012>
\end{CCSXML}

\keywords{OPRF, 2PC, malicious security, decision trees}


\ccsdesc[500]{Security and privacy~Cryptography}
\ccsdesc[500]{Security and privacy~Privacy-preserving protocols}
\ccsdesc[500]{Theory of computation~Cryptographic primitives}
\ccsdesc[500]{Theory of computation~Cryptographic protocols}


\ignore{
<ccs2012>
<concept>
<concept_id>10002978.10002979</concept_id>
<concept_desc>Security and privacy~Cryptography</concept_desc>
<concept_significance>500</concept_significance>
</concept>
<concept>
<concept_id>10002978.10002991.10002995</concept_id>
<concept_desc>Security and privacy~Privacy-preserving protocols</concept_desc>
<concept_significance>500</concept_significance>
</concept>
<concept>
<concept_id>10003752.10003777.10003788</concept_id>
<concept_desc>Theory of computation~Cryptographic primitives</concept_desc>
<concept_significance>500</concept_significance>
</concept>
<concept>
<concept_id>10003752.10003777.10003789</concept_id>
<concept_desc>Theory of computation~Cryptographic protocols</concept_desc>
<concept_significance>500</concept_significance>
</concept>
</ccs2012>

  }%ignore

\begin{document}
\fancyhead{}


\title[Iterative Oblivious Pseudo-Random Functions and Applications]{Iterative Oblivious Pseudo-Random Functions \\and Applications}
\author{Erik-Oliver Blass}
\email{erik-oliver.blass@airbus.com}
\affiliation{\institution{Airbus}\city{Munich}  \country{Germany}
}

\author{Florian Kerschbaum}
\email{florian.kerschbaum@uwaterloo.ca}
\affiliation{\institution{University of Waterloo}\city{Waterloo} \country{Canada}
}

\author{Travis Mayberry}
\email{mayberry@usna.edu}
\affiliation{\institution{US Naval Academy}\city{Annapolis, MD}\country{USA}}

\begin{abstract}
x
\end{abstract}
\maketitle


\section{Introduction}
Structured encryption allows a data owner to encrypt structured data,
e.g., an XML-document or a tree, and outsource it to an untrusted
server.  They crucial property of structured encryption is that the
data owner can later compute a token which permits the server to
decrypt and parse some well defined part of the data structure, e.g.,
certain elements of the XML-document or a path of the
tree. Computation of tokens is possible despite the owner retaining
only a constant-sized key.

In this paper, we introduce a twist to the standard setting of
structured encryption. Another party, the client, can ask the data
owner to prepare a token to decrypt a specific part of the owner's
data structure. Yet, data owner and client are mutually
untrusted. That is, the client does not want to reveal which part of
the data structure the client is interested in, e.g., which path in
the tree, and the owner wants to prepare a decryption token which
confines the client to accessing only one part of the data
structure. Using the token, the client then fetches the part of the
data structure they are interested in from the server and decrypt
it. In case the client does not trust the server, the client reverts
to standard techniques such as PIR to fetch data from the server. We
will see later that the new setting of structured encryption with an
untrusted client has several interesting real-world applications.

Yet, enabling this added functionality turns out to be technically
challenging. The client parses the owner's data structure in an
iterative fashion, where they can query to decrypt the next item from
the owner's data structure after receiving the previous item.  For
example, after decrypting one node of a binary tree, the client should
be able to query for decryption of either the left or right child
depending on the current node's data content.  At the same time, the
data owner wants to ensure that the client's next node is linked to
the current node, and the client cannot arbitrarily ``jump around'' in
the data structure.

Straightforward, intuitive approaches, where data
is encrypted using, for example, an Oblivious Pseudo-Random Function
(OPRF), fail as...

\section{Background and Related Work}
Before introducing $\iprf$s, $\ioprf$s, and their constructions, we
briefly revisit seminal PRF and OPRF schemes and some useful
security definitions.  They will be helpful in understanding the
intuition behind $\iprf$s and $\ioprf$s.


%\paragraph{PRFs and OPRFs}
While there exist many different
PRFs~\cite{chaum,prf,dodis,ggm,lewko,bonehprf} and
OPRFs~\cite{oprf,stan,chase,koles,boneh,kia}, we present the DH-based
techniques by \citet{prf} and \citet{oprf}, as our constructions are
build on their main idea.

Let $\myG$ be a group of prime order $p$ where the DDH assumption
holds, and $g$ is a random generator of $\myG$. For a security
parameter $\lambda$, we set $|p|=\mathsf{poly}(\lambda)$.

\begin{construction}[\citeauthor{prf} Function] \label{nrprf}
For any $\ell\in\N$, consider function family (ensemble)
$F_K(x):(\Z_p)^{\ell+1}\times\{0,1\}^\ell\rightarrow\myG$,
where key $K$ is defined as sequence $K=(\alpha_0,\ldots,\alpha_\ell)$
of $\ell+1$ random elements $\alpha_i$ from $\Z_p$.  For any $\ell$ bit input
$x=x_1 \ldots x_\ell$, function $F_K$ is defined by
$$F_K(x) = (g^{\alpha_0})^{\prod_{x_i=1}\alpha_i}.$$
  \end{construction}

Function $F_k$ holds the following important randomness property. We
will come back to it later in the proof of our own construction.

\begin{definition}[\citeauthor{prf} Pseudo-Randomness]\label{def:pr}
For any $\ell\in\N$, function family $F_K(x):(\Z_p)^{\ell+1}\times\{0,1\}^{\ell}\rightarrow\myG$ has \emph{pseudo-random output}, \emph{iff}
  for every PPT distinguisher
$\mathcal{D}$, there exists a negligible function $\epsilon$ such that
for sufficiently large $\lambda$
$$| Pr[\mathcal{D}^{F_K(\cdot)}(1^\lambda)=1] - Pr[\mathcal{D}^{R(\cdot)}(1^\lambda) = 1]|
    =\epsilon(\lambda), $$ where $K\getr(\Z_p)^{\ell+1}$, and $R$ is a randomly chosen function
    from the set of functions with domain $\{0,1\}^{\ell}$ and image
    $\myG$.
\end{definition}


\begin{theorem}[Theorem~4.1 of~\cite{prf}]
\label{theorem:naor} 
If the DDH-Assumption holds, then $F_K$ from Construction~\ref{nrprf}
has pseudo-random output.
\end{theorem}
Observe that $F_K$ from Construction~\ref{nrprf} is \emph{not} a
pseudo-random function. The standard PRF textbook definition (which we
omit here) requires indistinguishability of PRF output from output of
a random function which $F_K$ does not provide.  However, $F_K$ can
trivially be converted into a PRF. If $H_\lambda$ is a family of
pairwise independent hash functions, and $h\getr{}H_\lambda$, then
$\hat{F}_K(\cdot)=h(F_K(\cdot))$ is a PRF by a standard argument of
the leftover hash lemma~\cite{leftover}. We will use the same argument
later for our techniques and thus concentrate only on the
pseudo-randomness property of Definition~\ref{def:pr}.

\begin{definition}[$\oprf$]
  Let $F_K$ be a pseudo-random function family. An $\oprf$ is a
  2-party protocol between a sender and a receiver realizing the
  following ideal functionality.  A trusted third party receives a key
  $K\in\{0,1\}^\lambda$ from the sender and input $x\in\{0,1\}^\ell$
  from the receiver and sends $F_K(x)$ to the receiver.
\end{definition}

\begin{construction}[OPRF$_K(x)$ from~\cite{oprf}]
\label{ot-oprf}
  During initialization, sender $S$ chooses key
  $K=(\alpha_0,\ldots,\alpha_\ell)$ by randomly sampling $\ell+1$
  scalars $\alpha_i\getr\Z_p$.
  To evaluate receiver $R$'s input $x=(x_1\ldots{}x_\ell)$, parties perform the following steps.
  \begin{enumerate}
  \item $S$ randomly selects $(r_1,\ldots,r_\ell),r_i\getr\Z_p$.
  \item $S$ and $R$ engage in $\ell$ rounds of $\binom{2}{1}$-OT. In round
    $i$, the server's input to OT is $(r_i,r_i\cdot\alpha_i)$, and the
    receiver's input is $x_i$. So, depending on $x_i$, the receiver gets either $z_i=r_i$ or $z_i=r_i\cdot{}\alpha_i$.
  \item $S$ sends $\hat{g}=g^{\frac{1}{\prod_{i=1}^{\ell}r_i}}$ to $R$, and
    $R$ outputs $\text{OPRF}_K(x)=\hat{g}^{\prod^{\ell}_{i=1}z_i}$.
    
    \end{enumerate}
\end{construction}

\citet{oprf} present a proof sketch for
Construction~\ref{ot-oprf}. Effectively, this OPRF assembles the
\citeauthor{prf} function $F_K$ on input $x$ in $\ell$ rounds.  If
the DDH assumptions holds, and the underlying OT is secure and does
not simultaneously leak $r_i$ and $r_i\cdot\alpha_i$, 
Construction~\ref{ot-oprf} is an OPRF (semi-honest
model).

\section{$\iprf$ and $\ioprf$ Definition}
In this paper we introduce the notion of iterative pseudo-random
functions ($\iprf$) and iterated oblivious pseudo-random functions
($\ioprf$).

Informally, an $\iprf$ is a keyed function with bit strings
$x=(x_1\ldots{}x_\ell)$ of length $\ell$ as input. It outputs $\ell$
bit strings $v_i$, each of length $\lambda$. Besides that each $v_i$
is indistinguishable from a randomly chosen bit string, the crucial
property which we target is that, for two bit strings $x$ and $x'$
sharing the same length $k$ bit prefix, the first $k$ outputs
$(v_1,\ldots,v_k)$ of $\iprf$ will be the same.

Similar to OPRFs, an $\ioprf$ is a two party protocol, where a
receiver gets $\iprf_K(x)$ for their input $x$, and the sender with
input key $K$ does not learn $x$. However, unlike standard OPRFs,
$\ioprf$s run in $\ell$ rounds as required by the application
scenarios we consider. In round $i$, the receiver adaptively inputs
$x_i$ such that eventually they receive all $\ell$ outputs from
$\iprf_K(x)$, where $x=(x_1\ldots{}x_\ell)$ is as specified during the
$\ell$ rounds.

\subsection{$\iprf$}
\begin{definition}[$\iprf$]\label{defiprf}
  For inputs $x=(x_1\ldots{}x_\ell)\in\{0,1\}^\ell$ and randomly
  chosen keys $K=(K_1,\ldots,K_\ell)\in\{0,1\}^{\ell\cdot\lambda}$, an
  \emph{iterative pseudo-random function} family $\iprf_K(x)$ is a sequence
  of mutually independent function families
  $$\iprf_K(x)=(f^1_{K_1}(x_{1}),\ldots,f^\ell_{K_1,\ldots,K_\ell}(x_{1}\ldots{}x_{\ell})),$$
  where each
  $f^i_{K_1,\ldots,K_i}(x_{1}\ldots{}x_{i}):\{0,1\}^{i\cdot\lambda}\times\{0,1\}^{i}\rightarrow{}\{0,1\}^\lambda$
  is a pseudo-random function family with key $(K_1,\ldots,K_i)$ from
  $K$ and input $(x_1\ldots{}x_i)$ from $x$.
 Concatenated output
  $V_\lambda=v_1||\ldots||v_\ell,v_i=f^i_{K_1,\ldots,K_i}(x_1\ldots{}x_i)$
  is a family of mutually independent random variables (a probability ensemble) of bit
  strings of length $\ell\cdot\lambda$.
\end{definition}

Definition~\ref{defiprf} implies that each probability ensemble
$v_i=\{(v_i)_\lambda\}_{\lambda\in\N}$ of length $\lambda$ bit strings
is computationally indistinguishable from an ensemble $u_i$ describing
uniformly random bit strings of length $\lambda$. However, probability
ensemble $V_\lambda=v_1||\ldots||v_\ell$ is \emph{not}
indistinguishable from an ensemble of uniformly random bit strings of
length $\lambda\cdot\ell$. Instead, if any two inputs $x$ and $x'$
share the same prefix of length $i$, then the first $i$ outputs
$(v_1,\ldots,v_i)$ of $\iprf_K(x)$ will equal those of $\iprf(x')$. Mutual independence means that $v_j$ does not depend on (combinations of) other $v_{i\neq{}j}$.


Besides being PRFs, we do not require anything else from underlying
functions $f^i$. Note that, in general, PRFs do not need to be
length-preserving~\cite{nonlengthpres}.

\mypara{Simple Constructions}
Observe that the hashed \citeauthor{prf} PRF $\hat{F}_K$ from
Construction~\ref{nrprf} is not an $\iprf$ and cannot easily be
converted into an $\iprf$. First, to support $\lambda\cdot\ell$
outputs, $\lambda$ for each input bit $x_i$, one might try and create
an $\iprf$ out of
$(\hat{F}_{K_1}(x_1),\ldots,\hat{F}_{K_1,\ldots,K_\ell}(x_1\ldots{}x_\ell))$,
where $K_1=\alpha_1,\ldots,K_\ell=\alpha_\ell$.  However, this is in
fact not an $\iprf$, as exemplified by inputs like
$x=(10\ldots{}0)$. There, we have
$\hat{F}_{K_1}(1)=\hat{F}_{K_1,K_2}(10)=\ldots=\hat{F}_{K_1,\ldots,K_\ell}(10\ldots{}0)$,
so the output repeats starting from the $2^\text{nd}$ invocation of
$\hat{F}_K$. In general, for any input $x=\mathsf{PREFIX}||0\ldots{}0$ ending
with a sequence of zeros, $\hat{F}_K(x)$ will be equal to
$\hat{F}_K(\mathsf{PREFIX})$ violating mutual independence of the $v_i$ in Definition~\ref{defiprf}.

Many simple construction from symmetric key PRFs for an $\iprf$
could be based on variable input length PRFs such as HMAC and a
collision resistant hash function $H$. For example, consider
$\iprf_K(x)=(\hmac_{H(K_1)}(x_1),\ldots,\hmac_{H(K_1||\ldots||K_\ell)}(x_1\ldots{}x_\ell))$.
While this and other variations and adoptions of standard symmetric key PRF-based setups (also PRG-based PRFs~\cite{ggm}) might result in valid $\iprf$s,
we dismiss them in favor of our new 
Construction~\ref{const:newprf} (Section $\S$\ref{sec:newprf}), as it offers several advantages. First, it
builds on the \citeauthor{prf} pseudo-randomness, so we can prove malicious security by an elegant, formal reduction from DDH to the $\iprf$ property.
More importantly, its key advantage
is that you can use it as a building block to construct an efficient $\ioprf$
which also supports delegation and verifiability. As we will see, the $\ioprf$ offers malicious security with highly efficient, practical ZK proofs, i.e., without reverting to reductions of expensive general ZK proofs. 

\subsection{$\ioprf$}
\begin{figure}[tb]
\RestyleAlgo{boxed}
\LinesNumbered
\begingroup
\removelatexerror% Nullify \@latex@error
\begin{spacing}{0.7}
\begin{functionality}[H]\small
  \tcp{Let $\iprf$ be an iterative pseudo-random function family}
  \For{$i=1$ {\bf to} {$\ell$} }{
    $R\rightarrow{}\ttpioprf:x_i$\;
    $S\rightarrow{}\ttpioprf$: $K_i$\tcp*{$K=(K_1,\ldots,K_\ell)$}
    $\ttpioprf\rightarrow{}R: v_i$ such
    that $(v_1,\ldots,v_\ell)=\iprf_K(x_1\ldots{}x_\ell)$\;
  }
\end{functionality}
\end{spacing}
\endgroup
\caption{Ideal Functionality $\fioprf$\label{idealioprf}}
\end{figure}

\begin{definition}[$\proto$]
  \label{def:ioprf}
  Let $\iprf_K$ be an iterative pseudo-random function family.  An
  \emph{iterative {oblivious} pseudo-random function} is an $\ell$-round
  probabilistic protocol $\proto$ between a sender $S$ with input key
  $K\in\{0,1\}^{\lambda\cdot\ell}$ and receiver $R$ with input bit
  string $x=(x_1\ldots{}x_\ell)\in\{0,1\}^{\ell}$ with the following
  properties.

  
\begin{enumerate}[leftmargin=*]
  \item Protocol $\proto$ realizes the ideal functionality $\fioprf$
    shown in Figure~\ref{idealioprf}. This is a reactive
    functionality allowing queries from $R$ in a total of $\ell$
    rounds.  After $\ell$ rounds, $R$ has received
    $(v_1,\ldots,v_\ell)=\iprf_K(x),|v_i|=\lambda$, from a trusted
    third party $\ttpioprf$.  Sender $S$ sends $K_i$ in round $i$, but
    receives nothing from $\fioprf$. We denote receiver $R$'s output $(v_1,\ldots,v_\ell)$ by $\ioprf_K(x)$.
  
  \item For all adversaries $\A$ in the real world, there exists a
    simulator $\myS_R$ in the ideal world such that $R$'s view
    $\mathsf{REAL}_{\proto,\A,R}(x,K)$ in the real world is
    computationally indistinguishable from $R$'s view
    $\mathsf{IDEAL}_{\fioprf,\myS_R(x)}(x,K)$ in the ideal world.

  \item For all adversaries $\A$ in the real world, there exists a
    simulator $\myS_S$ in the ideal world such that $S$'s view
    $\mathsf{REAL}_{\proto,\A,S}(K)$ in the real world is
    computationally indistinguishable from $S$'s view
    $\mathsf{IDEAL}_{\fioprf,\myS_S}(K)$ in the ideal world.
\end{enumerate}
\end{definition}

The crucial difference of $\ioprf$s in contrast to regular
$\oprf$s~\cite{oprf,stan,chase,koles,boneh,kia} is that at the end of
the protocol execution, $R$ has received not one but $\ell$ PRF values
$v_i$ with $(v_1,\ldots,v_\ell)=\iprf_K(x)$. For two inputs $x$ and $x'$
with the same length $i$ bit prefix, values $v_1,\ldots,v_i$ will be
the same. Note that receiver $R$ can specify their input adaptively
during $\ell$ rounds. Before sending $x_i$, $R$ has learned $v_{i-1}$
from $\fioprf$. Still, $R$ receives output strings matching an
$\iprf$, so they cannot combine outputs from different $\ioprf$
executions with different input. For example, knowledge of
$\ioprf_K(10\ldots)$ and $\ioprf_K(01\ldots)$ should not allow $R$ to
learn anything about $\ioprf_K(11\ldots)$.  Against a fully-malicious
$R$, this cannot be accomplished easily with regular OPRFs.  One
might try and run $\ell$ instances of the OPRF, but the challenge is
that one would have to force $R$ to link their input during the
$i^\text{th}$ instance of the OPRF to the $(i-1)^\text{th}$ instance.
Our $\ioprf$ in Section~\S\ref{our-ioprf} offers a solution to this
challenge.

\mypara{Verifiability}
An important aspect of OPRFs which we also require for $\ioprf$s is
that of {verifiablity}, see \citet{kia} for technical
details. Essentially, verifiablity implies that $S$ proves to $R$ that
$R$'s output $(v_1,\ldots,v_\ell)$ has been computed correctly. Towards
providing malicious security, verifiability is especially important
when the $\ioprf$ is run multiple times, as $S$ could cheat by using
different keys for different protocol runs.
We refer to \cite{kia} for a treatment with more formal definitions in
the context of OPRFs which also hold for $\ioprf$s.  For our
constructions, we will prove that $R$'s output has been
correctly computed by using a key which $S$ has been committed to
before.

Observe that the original \citet{oprf} OPRF
(Construction~\ref{ot-oprf}) is not maliciously secure and thus does
not offer verifiablity. Even if OT as a building block would be secure
against a malicious adversary, it is unclear how to verify that the
sender has used the same key $K$ for different OPRF protocol runs.

\mypara{Efficiency}
The last crucial property we require is that $\ioprf$s are efficient
with respect to their communication and computational complexity.
Efficiency is important in practice, as a client can perform
$q\geq{}1$ queries to decrypt $q$ paths in the owner's data structure.
For each query, after all $\ell$ rounds, an $\ioprf$ has output $\ell$
bit strings of length security parameter, so the data exchanged
between $S$ and $R$ and the number of computations involved to realize
the $\ioprf$ should be linear in $\ell$.
Communication and computational complexities of an $\ioprf$ are
asymptotically \emph{optimal} if, after any $q$ queries, they are both
in $O(q\cdot\ell)$.  Our main contribution
(Construction~\ref{const:ioprf}, \S\ref{our-ioprf}) has optimal
communication and computational complexities.

\subsection{Delegation for $\iprf$s and $\ioprf$s}
Informally, a PRF $F$ with domain $D$ is delegatable, if for some
subset $D'\subset{}D$ you can (efficiently) compute a sub-key $K'$
from key $K$ and another PRF $F'$ from $F$, such that $F'_{K'}$ equals
$F_K$ on all $x\in{}D'$, but is random everywhere else. There exists a
rich theory on delegatable PRFs, see \citet{delegate} for details.

In the context of $\iprf$s, we are particularly interested in
delegating iterative PRF computation for strings
$x=(x_1\ldots{}x_\ell)$ sharing the same fixed prefix. That is, a
party $P_1$ knowing key $K$ specifies a prefix
$x^*=(x^*_1\ldots{}x^*_i)$, computes $K'$ and $\iprf'$, and gives
$(\iprf',K')$ to party $P_2$. Party $P_2$ is then capable of computing
$\iprf_K(x)$ for all bit strings $x$ having the same prefix $x^*$. At
the same time, for all bit strings $x$ with a different prefix than
$x^*$, $K'$ does not help $P_2$ in distinguishing the first $i$
outputs of $\iprf_{{K}}(x)$ from the output of random bit strings.  We
formalize this intuition in Definition~\ref{def:del}.

\begin{definition} \label{def:del}
  Let $\iprf$ be an iterative pseudo-random function on length $\ell$
  bit input strings with random key $K$.  We call an $\iprf$
  \emph{delegatable}, \emph{iff}
  \begin{enumerate}[leftmargin=*]
  \item There exists a PPT transformation algorithm $T$, which on
    input $(\iprf,K,x^*_1\ldots{}x^*_i)$ outputs $(\iprf',K')$, where
    $\iprf':\{0,1\}^{\lambda\cdot(\ell-i)}\times\{0,1\}^{\ell-i}\rightarrow{}\{0,1\}^{\lambda\cdot(\ell-i)}$
    and
    $\forall{}x'=(x'_1\ldots{}x'_{\ell-i}):\iprf'_{K'}(x')=\mathsf{SUFFIX}_{\ell-i}(\iprf_{K}(x^*_1\ldots{}x^*_ix'_1\ldots{}x'_{\ell-i}))$.

    Here,
    $\mathsf{SUFFIX}_{\ell-i}(\cdots)$ denotes the last $\ell-i$ PRF outputs,
    each of length $\lambda$ bit, of $\iprf_K(\cdots)$.

\item For all PPT distinguishers $\D$ and randomly chosen $K$,
    there exists a negligible function $\epsilon$ such that for
    sufficiently large $\lambda$ we have
    \begin{align*}
&\forall{}x^*=(x^*_1\ldots{}x^*_i),\forall{}x=(x_1\ldots{}x_\ell),x_1\ldots{}x_i\neq{}x^*_1\ldots{}x^*_i:
\\      &|Pr[(v_1,\ldots,v_\ell)=\iprf_{K}(x):\D(1^\lambda,\iprf',K',x,{v}_1,\ldots,{v}_i)=1]-\\&Pr[(r_1,\ldots,r_i)\getr{}U_{\lambda}:\D(1^\lambda,\iprf',K',x,r_1,\ldots,r_i)=1]|=\epsilon(\lambda),
\ignore{      
           &\forall{}x=(x_1\ldots{}x_\ell),x_1\ldots{}x_i\neq{}x^*_1\ldots{}x^*_i:\\&|Pr[\hat{K}=(\hat{K}_1,\ldots,\hat{K}_\ell)\getr\{0,1\}^{\lambda\cdot\ell},(v_1,\ldots,v_\ell)=\iprf_K(x),\\&(\hat{v}_1,\ldots,\hat{v}_\ell)=\iprf_{\hat{K}}(x):\D(1^\lambda,\iprf',K',(\hat{v}_1,\ldots,\hat{v}_i,v_{i+1},\ldots,v_\ell))=1]\\-&Pr[\D(1^\lambda,\iprf',K',v_1,\ldots,v_\ell)=1]|=\epsilon(\lambda).
}%ignore
    \end{align*}
    where $U_{\lambda}$ is the probability ensemble of random bit
    strings of length $\lambda$, $K$ is a randomly chosen key for
    $\iprf$, and $(\iprf',K')$ are output by $T(\iprf, K,
    x^*_1\ldots{}x^*_i).$
\end{enumerate}

\end{definition}
%Along the same lines,
A \emph{delegatable} $\ioprf$ is an $\ioprf$
where the underlying $\iprf$ supports delegation. 


\mypara{Discussion}
Note that knowledge of $K'$ and the first $i$ values of the output
$(v_1,\ldots,v_i)$ of $\iprf_K(x)$ does permit $P_2$ to enumerate all
suffixes of strings $x$ which share the same length $i$ prefix as
$x$. At first, this property might look like a severe restriction to
the value of this type of delegation, but we will show in
Section~\ref{sec:applications} that it has very interesting
real-world applications.

We implicitly require delegation non-triviality (bandwidth
efficiency~\cite{delegate}). For example, $P_1$ could
delegate the capability to evaluate strings with prefix $x^*$ by
computing $\iprf_K(x)$ for all strings $x$ with prefix $x^*$
and sending the output to $P_2$. Tuple $(\iprf',K')$ should be
smaller in size than the concatenation of all strings with prefix
$x^*$.

Finally, we point out that delegation can be extended from $\iprf$s to
$\ioprf$s in the natural way. If $P_1$ gives $(\iprf',K')$ to $P_2$,
then $P_2$ is also able to run a 2-party protocol with another party
$P_3$, where $P_3$ correctly receives
$\ioprf'_{K'}(x')=\iprf'_{K'}(x')$ for input $x'$ with prefix $x^*$
while $P_2$ learns nothing about $x'$.



\section{New Constructions}
We present our new constructions for both $\iprf$ and $\ioprf$.  To
ease readability, we omit an important technicality in the description
and proofs: our $\iprf$ and $\ioprf$ constructions do not output
pseudo-random bit strings of length $\lambda$, but pseudo-random
elements of DDH group $\myG$. Yet, converting elements to bit strings
follows from a standard application of the leftover hash
lemma~\cite{leftover}. As $|p|\geq\lambda$, we have
$|\myG|\ge{}2^\lambda$, and we silently assume in the following that
each party implicitly hashes the output of $\iprf$ and $\ioprf$ using
any pairwise independent family of hash functions.


\subsection{$\iprf$ Construction}
\label{sec:newprf}
\begin{construction}[Our iPRF]
  \label{const:newprf}
For any $\ell\in\N$, choose a key $K=(\vec{{\alpha}},\vec{\beta})$ by
sampling $\ell$ pairs of random scalars
$K_i=({\alpha_{i}},{\beta_{i}})\getr{}(\Z_p)^2.$ For any $\ell$ bit
input $x=x_1\ldots{}x_\ell$, we define function family
$\iprf_K(x_1,\ldots,x_\ell)=(f^1_{(\alpha_1,\beta_1)}(x_1),\ldots,f^\ell_{(\alpha_1,\beta_1),\ldots,(\alpha_\ell,\beta_\ell)}(x_1,\ldots,x_\ell))$,
where
$$  f^i_{(\alpha_1,\beta_1),\ldots,(\alpha_i,\beta_i)} \stackrel{\text{def}}{=}
  g^{\prod_{x_i=1}\alpha_i \prod_{x_i=0}\beta_i}
=g^{\prod_{j=1}^{i}{\alpha_i^{x_i}\cdot{}\beta_i^{1-x_i}}}.$$


Observe that you can also rewrite expression
$g^{\prod_{j=1}^{i}{\alpha_i^{x_i}\cdot{}\beta_i^{1-x_i}}}$ as
$g^{\prod_{j=1}^{i}(\alpha_j{}x_j+\beta_j(1-x_j))}$. This
representation of $f^i$ will be very useful for during the
presentation of $\proto$ later.

\ignore{Define the function ensemble $F' = \{F'_n\}_{n\in N}$.  For every $n$, a key of a function $F'_n$ is a tuple, $\langle P,Q,g,\vec{\alpha},\vec{\beta}\rangle$, 
where $P$ is an $n$-bit prime, $Q$ a prime divisor of $P-1$, $g$ an element of order $Q$ in $\mathbb{Z}_{p}^*$ and $\vec{\alpha}=\langle 
\alpha_0,\alpha_1, \ldots , \alpha_n \rangle$ and $\vec{\beta}=\langle 
\beta_0,\beta_1, \ldots ,\beta_n \rangle$, uniformly random sequences of $n+1$ elements of $\mathbb{Z}_Q$.  For any $n$-bit input, $x=x_1 x_2 \ldots x_n$, the 
function $f'_{P,Q,g,\vec{a},\vec{b}}$ is defined by:
$$f'_{P,Q,g,\vec{\alpha},\vec{\beta}}(x) =(g^{\alpha_0})^{\prod_{x_i=1}\alpha_i \prod_{x_i=0}\beta_i}$$
}%ignore
 \end{construction}

\todo{Show that this is a delegatable PRF.}

To show that our new $\iprf$ is actually an $\iprf$ according to
Definition~\ref{defiprf}, it is sufficient to show that each $f^i$ is
still a pseudo-random function.

\begin{theorem}
\label{theorem:newprf}
If the DDH-Assumption holds, then for every $i\leq\ell$ and for every
PPT distinguisher $\mathcal{D}$, there exists a negligible function
$\epsilon$ such that for sufficiently large $\lambda$
$$|
Pr[\mathcal{D}^{f^i_{(\alpha_1,\beta_1),\ldots,(\alpha_i,\beta_i)}}=1]
- Pr[\mathcal{D}^{R^i} = 1]| =\epsilon(\lambda), $$ where the
$(\alpha_1,\ldots_1),\ldots,(\alpha_i,\beta_i)$ are chosen randomly as
in Construction~\ref{const:newprf}, and $R^i$ is a randomly chosen
function from the set of functions with domain $\{0,1\}^i$ and image
$\myG$.
\ignore{
where in the first probability, $f'_{P,Q,g,\vec{a},\vec{\beta}}$ is distributed according to $F'_n$, and in the second probability the distribution of $R_{P,Q,g}$ is uniformly chosen in the set of functions with the domain $\{0,1\}^n$ and range $\langle g\rangle$.
}%ignore

\ignore{$$| Pr[\mathcal{M}^{f'_{P,Q,g,\vec{\alpha},\vec{\beta}}}(P,Q,g)=1] - Pr[\mathcal{M}^{R_{P,Q,g}}(P,Q,g) = 1]| < \frac{1}{n^\gamma} $$

where in the first probability, $f'_{P,Q,g,\vec{a},\vec{\beta}}$ is distributed according to $F'_n$, and in the second probability the distribution of $R_{P,Q,g}$ is uniformly chosen in the set of functions with the domain $\{0,1\}^n$ and range $\langle g\rangle$.
}%ignore
\end{theorem}

\begin{proof}
Fix any $i\leq\ell$ and consider $f^i$.  We prove the claim by
reduction, showing that if $\mathcal{D}$ exists that can distinguish
between $f^i$ and a random function $R^i$, then we can build
$\mathcal{D}'$ that can distinguish between $F_K$ from
Construction~\ref{nrprf} (on $i$ bit inputs and $i$ element keys) and a
random function $R$ (on $i$ bit inputs).

First, assume that $\mathcal{D}$ exists that can violate the
inequality from Theorem~\ref{theorem:newprf}.  We create
$\mathcal{D'}$ as follows.  First, $\mathcal{D}'$ creates and stores a
uniformly random vector $\vec{\beta}$ as in
Construction~\ref{const:newprf}.  $\mathcal{D}'$ then runs
$\mathcal{D}$ as a subroutine.  Each time $\mathcal{D}$ queries the
oracle for an evaluation on input $y\in\{0,1\}^i$, $\mathcal{D}'$ does the
following:

\begin{enumerate}
\item Query their own oracle for $y$ and receive back $z$.
\item Calculate $z' = z^{\prod_{y_i = 0} \beta_i}$.
\item Return $z'$ to $\mathcal{D}$.
\end{enumerate}

Eventually, $\mathcal{D}'$ outputs the same as $\mathcal{D}$.  If
$\mathcal{D}'$ is interacting with PRF $F_K$, then the $z'$ values
$\mathcal{D}'$ gives to $\mathcal{D}$ will be identical to function
$f^i$, due to $\mathcal{D}'$ being able to multiply in the extra
$\beta$ components.  If $\mathcal{D}'$ is interacting with a real
random function, then the responses they give will be distributed
identically to a random function. \fixme{Yes, but why? There is one
  sentence of explanation missing.}  Therefore, if $\mathcal{D}$ has a
distinguishing advantage, so will $\mathcal{D}'$.
\end{proof}



\refstepcounter{construction}\label{const:ioprf}
\section{Construction~\ref{const:ioprf}: DH-based $\ioprf$}
\label{our-ioprf}
We now present a new $\proto$ protocol which realizes the ideal $\ioprf$
functionality $\fioprf$ from Figure~\ref{idealioprf}.

\subsection{Preliminaries}
%\subsubsection{Preliminaries}
Let there be two generators
$g_1,g_2$ of prime order $p$ group $\myG$ where the DDH assumption
holds. Neither party should know the discrete log of one generator
$g_i$ to the basis of the other generator $g_{j\neq{}i}$, which is
true with high probability if they are chosen at random.
%\fk{I just want to note that in a prime-order group, if the discrete logarithm $\log_{g_1}(g_2)$ is known, then so is $\log_{g_2}(g_1)$.}

\paragraph{Elgamal Encryption}
We will use additive Elgamal encryption with private keys $sk\in\Z_p$
and public keys $pk=g_1^{sk}$. Ciphertext $c$ to encrypt $m\in\Z_p$ is
$c=(c[0],c[1])=(g_1^r,pk^r\cdot{}g_2^m)\leftarrow\enc_{pk}(m)$, where
$r\getr\Z_p$.

\paragraph{Pedersen Commitments}
A Pedersen commitment $\com(m)\in\myG$ to message $m\in\Z_p$ is
defined as $\com(m)=g_1^r\cdot{}g_2^m$, where $r\getr\Z_p$.  To open
$\com(m)$, reveal tuple $(m,r)$. Pedersen commitments are perfectly
hiding and computationally binding.

\subsection{High-Level Intuition}
In round $i$ of $\ell$ rounds, sender $S$ will receive
two ciphertexts $V_i$ and $D_i$ from receiver $R$.  During the course of the
protocol, one of these ciphertexts will contain the $\ioprf$ output and one acts
as a ``dummy'', to keep $S$ from learning input bits $x_i$ of $R$.  They are
interchanged between rounds depending on the input bits.

For each round, using the $i^{\text{th}}$ round's keys $(\alpha_i,\beta_i)$, $S$
will then ``apply'' $\alpha_i$ to $V_i$ and $\beta_i$ to $D_i$, and send the
results back to $R$. In preparation for the next round $(i+1)$, if $x_{i+1}\neq
x_{i}$, $R$ will swap $V_i$ and $D_i$ for the next round.  After $\ell$ rounds,
$V_\ell$ will have the keys applied which correspond to the input bits of $R$, and
$D_\ell$ will have the complementary combination of keys applied.  $V_0$ is initialized
as an encryption of 1, so $V_\ell$ will contain the correct $\ioprf$ output, whereas
$D_0$ is initialized as an encryption of 0 so it will not contain any information.

\subsection{Technical Details}
For some input string $x=(x_1\ldots{}x_\ell)$, we define the output of
$\proto$ for the receiver as $(v_1,\ldots,v_\ell)=\ioprf_{K}(x)$ with
$v_i=g_2^{\prod_{j=1}^{i}(\alpha_j{}x_j+\beta_j(1-x_j))}$ and $K=\{(\alpha_i,\beta_i)\}^\ell_{i=1}$. 
We now describe details of Construction~\ref{const:ioprf} by its formal
$\proto$ interface (Definition~\ref{def:ioprf}), i.e., first its
initialization and then its iterative processing.

\subsubsection{$\proto$ Initialization}
Sender $S$ randomly chooses secret key
$K=((\alpha_1,\beta_1),\ldots,(\alpha_\ell,\beta_\ell)),
(\alpha_i,\beta_i)\getr(\Z_p)^2$.

$S$ also commits to $K$ by computing $2\ell$ Pedersen commitments
  $(\com(\alpha_i),\com(\beta_i))$. $S$ sends them to $R$ and
  proves knowledge of plaintexts in ZK (see \S\ref{pokop}).

Receiver $R$ computes a random
Elgamal private key $sk\getr\Z_p$ and public key $pk=g_1^{sk}$, and
sends $pk$ to $S$. Receiver $R$ proves knowledge of $sk$ using
a standard Schnorr ZK proof of knowledge (see \S\ref{poe}).


Receiver $R$ computes $V_0 \leftarrow\enc_{pk}(1)$ and
$D_0\leftarrow\enc_{pk}(0)$, sends them to $S$ and proves that
these are encryptions of $1$ and $0$ (see \S\ref{poe} below). 

\subsubsection{$\proto$ Iterative Processing in $\ell$ Rounds}
In round $i\in\{1,\ldots,\ell\}$, for $S$' input bit $x_i$:
\begin{enumerate}
  
\item {\bf Receiver shuffles:}
\begin{enumerate}%[leftmargin=0.3cm]
\item For input bit $x_i$, $R$ computes Pedersen commitment
  $\com{}(x_i)$ and proves that $x_i\in\{0,1\}$ (see
  \S\ref{pobit}). Similarly, $R$ computes $\com{}(1-x_i)$
  and proves that $(1-x_i)\in\{0,1\}$ (see \S\ref{pobit}). Finally,
  $R$ proves that the sum of plaintexts behind
  $\com{}(x_i)$ and $\com{}(1-x_i)$ equals $1$ (see
  \S\ref{pkseo}).


\item  Receiver $R$ chooses $r,r',r'',r'''\getr\Z_p$ and computes Elgamal ciphertexts
  \begin{align*}
    c_i&=(g_1^r\cdot{}V_{i-1}[0]^{x_i},pk^{r}\cdot{}V_{i-1}[1]^{x_i})
    \\c'_i&=(g_1^{r'}\cdot{}V_{i-1}[0]^{1-x_i},pk^{r'}\cdot{}V_{i-1}[1]^{1-x_i})
    \\d_i&=(g_1^{r''}\cdot{}D_{i-1}[0]^{x_i},pk^{r''}\cdot{}D_{i-1}[1]^{x_i})
    \\d'_i&=(g_1^{r'''}\cdot{}D_{i-1}[0]^{1-x_i},pk^{r'''}\cdot{}D_{i-1}[1]^{1-x_i})
 \end{align*} 
  \ignore{
    \vskip 1eX
\NoIndent{\begin{tabular}{@{}l@{\hskip 0.3cm}l}
    $c_i=(g_1^r\cdot{}V_{i-1}[0]^{x_i},pk^{r}\cdot{}V_{i-1}[1]^{x_i})$
    &$c'_i=(g_1^{r'}\cdot{}V_{i-1}[0]^{1-x_i},pk^{r'}\cdot{}V_{i-1}[1]^{1-x_i})$
    \\$d_i=(g_1^{r''}\cdot{}D_{i-1}[0]^{x_i},pk^{r''}\cdot{}D_{i-1}[1]^{x_i})$
    &$d'_i=(g_1^{r'''}\cdot{}D_{i-1}[0]^{1-x_i},pk^{r'''}\cdot{}D_{i-1}[1]^{1-x_i})$%\text{ and}
          \end{tabular}}
        }%ignore
  and sends $(c_i,c'_i,d_i,d'_i)$ to $S$.
\item Receiver $R$ proves correctness of the above computations in
  ZK. Specifically, $(c_i,c'_i,d_i,d'_i)$ result from correct
  exponentiation with $x_i$ (or $1-x_i$) from $\com{}(x_i)$ (or
  $\com{}(1-x_i)$), and multiplication with a random power of
  $g_1$ and $pk$, i.e., re-randomization (homomorphic addition of
  encryption of $0$).  See \S\ref{pexr} below for details.
   Both parties compute
\begin{align*}
   T_i&=(c_i[0]\cdot{}d'_i[0],c_i[1]\cdot{}d'_i[1])
    \\U_i&=(c'_i[0]\cdot{}d_i[0],c'_i[1]\cdot{}d_i[1]).
\end{align*}   
  \end{enumerate}
In the first round, after this step, $T_1$ is an encryption of $1$ and $U_1$ is an encryption of $0$ if $x_1 = 1$.
If $x_1 = 0$, then $T_1$ is an encryption of $0$ and $U_1$ is an encryption of $1$.
However,  sender $S$ does not know which of the two is the case.

\item {\bf Sender computes PRF:} For $r,r'\getr\Z_p$, $S$ computes the two Elgamal ciphertexts
  \begin{align*}
    X_i&=(g^r_1\cdot{}T_i[0]^{\alpha_i},pk^r\cdot{}T_i[1]^{\alpha_i})
    \\Y_i&=(g^{r'}_1\cdot{}U_i[0]^{\beta_i},pk^{r'}\cdot{}U_i[1]^{\beta_i}),
    \end{align*}
  sends $(X_i,Y_i)$ to $R$, and proves correct exponentiation
  (scalar multiplication of plaintexts) with $\alpha_i$ and $\beta_i$
  coming from previous commitments $\com{}(\alpha_i),\com{}(\beta_i)$
  \emph{and} re-randomization of ciphertexts (see \S\ref{pexr}).

\item {\bf Receiver shuffles back:}
  For $r,r',r'',r'''\getr\Z_p$, $R$ computes%\vskip 2eX
  \begin{align*}
    P_i&=(g_1^r\cdot{}X_i[0]^{x_i},pk^r\cdot{}X_i[1]^{x_i})
    \\P'_i&=(g_1^{r'}\cdot{}X_i[0]^{1-x_i},pk^{r'}\cdot{}X_i[1]^{1-x_i})
   \\Q_i&=(g_1^{r''}\cdot{}Y_i[0]^{x_i},pk^{r''}\cdot{}Y_i[1]^{x_i})
   \\Q'_i&=(g_1^{r'''}\cdot{}Y_i[0]^{1-x_i},pk^{r'''}\cdot{}Y_i[1]^{1-x_i})
\end{align*}
   \ignore{
     \begin{centering}
    \begin{tabular}{l@{\hskip 0.5cm}l}
    $P_i=(g_1^r\cdot{}X_i[0]^{x_i},pk^r\cdot{}X_i[1]^{x_i})$
    &$P'_i=(g_1^{r'}\cdot{}X_i[0]^{1-x_i},pk^{r'}\cdot{}X_i[1]^{1-x_i})$
   \\$Q_i=(g_1^{r''}\cdot{}Y_i[0]^{x_i},pk^{r''}\cdot{}Y_i[1]^{x_i})$
   &$Q'_i=(g_1^{r'''}\cdot{}Y_i[0]^{1-x_i},pk^{r'''}\cdot{}Y_i[1]^{1-x_i})$
  \end{tabular}
  \end{centering}
  \vskip 2eX
  }%ignore
  and sends $(P_i,P'_i,Q_i,Q'_i)$ together with ZK proofs of correct
  computation (see \S\ref{pexr}) to $S$.

  Both $S$ and $R$ compute
  $V_i=(P_i[0]\cdot{}Q'_i[0],P_i[1]\cdot{}Q'_i[1])$ and
  $D_i=(P'_i[0]\cdot{}Q_i[0],P'_i[1]\cdot{}Q_i[1])$.
  
In round $i$, after this step, $V_i$ is an encryption of $\iprf_{K}(x_1,\ldots,x_i)$, and $U_i$ is an encryption of $0$.
When computing $T_{i+1}$ and $U_{i+1}$, these values will be used instead of the encryptions of $0$ and $1$ and the iterative computation of the PRF continues.
Since both parties compute $V_i$ and $U_i$, $R$ cannot cheat and substitute for a value of his choice.

\item Receiver $R$ computes and outputs one $\iprf$ value
  $v_i=\frac{V_i[1]}{V_{i}[0]^{sk}}$.
\end{enumerate}

\paragraph{Discussion}
Observe that, in the last step, $R$ can never decrypt
additively homomorphic Elgamal ciphertext $(V_i[0],V_i[1])$ and thus
compute an $\alpha_i$ or $\beta_i$. As $\alpha_i$ or $\beta_i$ are in
the exponent and due to the hardness DLOG, $R$ can only
compute $v_i=g_2^{\ldots\alpha_i\ldots}$ or
$v_i=g_2^{\ldots\beta_i\ldots}$.
If $R$ wants to run several execution of
Construction~\ref{const:ioprf} and wants that $S$ uses the same key,
then $R$ will verify that commitments sent by $S$ during initialization do not change between executions. This
leads to {verifiability}.
Also note that communication complexity and computational complexity
are both in $O(\ell)$ per query, i.e.,  asymptotically
optimal.

%\vskip 1eX\noindent{\bf Security Analysis:} Due to space constraints, we defer %our full security analysis, including formal proofs, to Appendix~\ref{sec:sec-analysis}.
\section{Security Analysis}
%\subsection{Security Analysis}
\label{sec:sec-analysis}
We prove security of Construction~\ref{const:ioprf} using simulation
in the standard model. The simulation uses several efficient
Zero-Knowledge Proofs of Knowledge hybrids introduced first.  To ease
readability, we actually present Honest-Verifier Zero-Knowledge (HVZK)
versions of the proofs, but one can convert these to maliciously
verifier Zero-Knowledge proofs of knowledge using the following two
general transformations~\cite{efficient2pc}. We stress that we have
evaluated and benchmarked the full malicious verifier ZK proofs of
knowledge in Section~\ref{sec:implementation}, i.e., including the two
transformations.

\subsection{Zero Knowledge (instead of HVZK)}
\label{sec:extraction}
\ignore{We cannot use Fiat-Shamir transform and replace $e$, as we use
  Pedersen commitments for witness extraction.}
All our efficient ZK proofs below are three-move (``Sigma'') ZK
proofs. Recall that a three-move ZK proof comprises messages
$(t,e,s)$, where first message $t$ is a commitment from $P$ sent to
$V$, $e$ is $V$'s challenge sent to $P$, and $s$ is the final message
sent from $P$ to $V$.


To make these proofs zero-knowledge instead of only HVZK, we send an
additional message before first message $t$ of the regular three-move
proof.  In this new first message, $V$ sends a Pedersen commitment
$\com{}(e)=g_1^r\cdot{}g_2^e$ to their random challenge $e$ to
$V$. The proof  continues with $V$ sending their regular
commitment $t$ of the regular three-move proof and $V$ opening
$\com{}(e)$ by sending $(e,r)$. If $\com{}(e)$ matches
$(e,r)$, $P$ finally sends last message $s$ of the regular
proof. Verifier $V$ accepts, if $t$ and $s$ of the regular proof match
$e$.

This technique allows a simulator $\myS$ simulating $P$ to cheat in
the ZK proof. More specifically, after receiving $\com{}(e)$, $\myS$
internally computes a valid ZK proof $(t',e',s')$, assuming a random
challenge $e'$. $\myS$ sends $t'$ to $V$ and receives $(e,r)$. If
$(e,r)$ matches $\com{}(e)$, $\myS$ rewinds $V$ to the point after $V$
has sent $\com{}(e)$. Knowing $e$, $\myS$ computes a $t$ and $s$, such
that $(t,e,s)$ will be accepted by $V$. How exactly $t$ and $s$ are
chosen depends on the statement we want to prove, but are typically
straightforward for the Schnorr-style proofs we use below. We show an
example in \S\ref{pobit}.


\subsection{Witness Extraction for Pedersen Commitments}
To transform our ZK proofs to ZK proofs of knowledge, we rely on the
extractability of commitments.  Pedersen commitments are trapdoor
commitments which means that a party knowing a trapdoor $\rho$ can
open a commitment $\com(\cdot)$ to any plaintext they want
(equivocable).  We use this property for witness extraction in
three-move ZK proofs as follows.

Before starting the actual ZK proof by the first message $t$ from the
prover to the verifier, we send the following two messages.
\begin{enumerate}[leftmargin=*]
\item Prover $P$ sends to verifier $V$: $\hat{g}=g_1^\rho$ for random
  $\rho\getr\Z_p$. 
  \item Verifier $V$ will use this $\hat{g}$ instead of $g_2$ for the computation of the
    commitment to challenge $e$.  That is, $V$ computes and sends back
    commitment $\com(e)=g_1^r\cdot{}\hat{g}^{\,e}$ for their random challenge
    $e\in\Z_p$ as in the previous section.
\end{enumerate}

The ZK proof then continues as usual with $P$ sending $t$ and $V$
opening $\com(e)$ by sending $(e,r)$. If $(e,r)$ match $\com(e)$,
$P$ sends final message $s$ and $\rho$ to $V$. Only if
both is correct, the last ZK proof message $s$ matches $P$'s
commitment $t$ and challenge $e$, and $\rho$ matches $\hat{g}=g_1^\rho$, $V$
accepts.

This setup enables a simulator $\myS$ simulating $V$ to extract the
witness from $P$. After receiving trapdoor $\rho$ from $P$, $\myS$
rewinds $P$ until after the point were $P$ sends $t$ to $V$. Knowing
trapdoor $\rho$, $\myS$ can open $\com(e)$ to any $e'\neq{}e$ they
want by solving $r+\rho\cdot{}e=r'+\rho\cdot{}e'$ for $r'$, i.e., they
compute $r'=r+\rho\cdot{}(e-e')$. Running two executions of the ZK
proof with the same input and messages from $P$ but different
challenges extracts the witness of the ZK proof. Details on which $e$
to send in each execution again depend on the exact three-move ZK
proof, but are typically obvious. We refer to \citet{efficient2pc} for
more details.

In conclusion, these two transformation will render our three-move ZK
proofs below into (fully-maliciously secure) ZK proofs of knowledge. We
name each proof below with a hybrid which we will use in the main
proof later. So, for example, the hybrid for the proof of encryption
is called $\fzk{enc}$.


\subsection{ZK Building Blocks}
Before presenting our main proof of Construction~\ref{const:ioprf}, we
introduce the following ZK proofs that we use as building blocks.

\subsubsection{$\fzk{enc}$: Proof of Encryption/Commitment to $m$}
\label{poe}
To prove that an encryption
$c=(c[0],c[1])=(g_1^r,pk^r)\leftarrow\enc_{pk}(0)$ is an encryption of
$m=0$, $P$ proves that $(g_1,c[0],pk,c[1])$ is a DDH tuple.  You can
prove that tuple
$(u_1=g_1,u_2=g_1^r,u_3=g_1^{sk},u_4=g_1^{sk\cdot{}r})$ is a DDH tuple
using the \citet{cp92} protocol as follows.

\begin{enumerate}[leftmargin=*]
\item $P$ sends $(t_1=u_1^{\rho},t_2=u_3^{\rho})$ for $\rho\getr\Z_p$ to $V$.
  \item $V$ sends $e\getr\Z_p$ to $P$.
  \item $P$ sends $s=\rho+e\cdot{}r$ to $V$.
    \item $V$ accepts if $u_1^s=u_2^e\cdot{}t_1$ and $u_3^s=u_4^e\cdot{}t_2$.
\end{enumerate}

This proof has an important property.\ignore{ Besides showing that a
  tuple is a DDH tuple, it also shows DLOG equivalence, i.e.,
  $\log_{u_1}{u_2}=\log_{u_3}{u_4}$.} Instead of showing that some
ciphertext encrypts $m=0$, we can easily generalize it to show
encryption of arbitrary $m$. Specifically, we set
$c'[1]=\frac{c[1]}{g_2^m}$ and run the proof with $m = 0$ for new Elgamal
ciphertext $(c[0],c'[1])$.

%Finally, observe that Pedersen commitments are essentially just the
%right-hand side $c[1]$ of an Elgamal ciphertext. 
Finally, observe that Pedersen commitments are similarly structured as the
right-hand side $c[1]$ of an Elgamal ciphertext, just without the secret key.
Thus, to prove a
Pedersen commitment $\com(m)$ to $m$, parties divide $\com(m)$ by
$g_2^m$ and run a {\bf Schnorr proof} for $r$ used in the commitment ($P$
sends $t=g_1^\rho$, $V$ sends $e$, $P$ sends $s=\rho+e\cdot{}r$, and $V$
accepts if $g_1^s\sr\frac{\com(m)^e}{g_2^m}\cdot{}t$.)

\subsubsection{$\fzk{pop}$: Proof for Knowledge of Plaintext }
\label{pokop}
For $\com(m)=g_1^r\cdot{}g_2^m$,  prover $P$ can prove that they know
$m$.

\begin{enumerate}[leftmargin=*]
\item $P$ sends $t=g_1^{\rho_1}\cdot{}g_2^{\rho_2}$ for
  $\rho_1,\rho_2\getr\Z_p$ to $V$.
\item $V$ sends $e\getr\Z_p$ to $P$.
  \item $P$ sends $s_1=\rho_1+e\cdot{}r$ and $s_2=\rho_2+e\cdot{}m$ to
    $V$.
    \item $V$ checks whether $g_1^{s_1}\cdot{}g_2^{s_2}\sr\com(m)^e\cdot{}t$.
\end{enumerate}

\subsubsection{$\fzk{bit}$: Proof of Plaintext Bit }
\label{pobit}
For a commitment $\com(x_i)$, prover $P$ can prove that $x_i$ is a
bit, i.e., $x_i\in\{0,1\}$. This is an application of the
\emph{one-out-of-two} (OR) technique~\cite{ooot}. Essentially, $P$
proves that either $x_i=1$ which implies proving that ${\com(x_i)}$
equals ${g_1^{r_1}\cdot{}g_2}$ for some $r_1$, or $x_i=0$ which implies
proving that $\com(x_i)$ equals $g_1^{r_2}$ for some $r_2$. Proving
that ${\com(x_i)}$ equals ${g_1^{r_1}\cdot{}g_2}$ is equivalent to
proving that $\frac{{\com(x_i)}}{g_2}$ equals ${g_1^{r_1}}$.

$P$ will prove that they know (I) an $r$ such that
$g_1^{r}=\frac{{\com(x_i)}}{g_2}$ or (II) an $r$ such that
$g_1^{r}=\com(x_i)$. These are essentially two standard Schnorr
proofs.  The trick is that $P$ chooses $e_1$ and $e_2$ such that, for
the verifier's challenge $e$, we have $e=e_1+e_2$. Prover $P$ proves
knowledge of $r_1$ for (I) using challenge $e_1$ and knowledge of
$r_2$ for (II) using challenge $e_2$. Thus, $P$ can choose either
$e_1$ or $e_2$ before sending their first message of the ZK proof and
cheat in one proof. Without loss of generality, let $x_i=1$, so $P$
will cheat in proof (II). This works as follows.

\begin{enumerate}[leftmargin=*]
\item $P$ sends $t_1=g_1^{\rho_1}$ and
  $t_2=\com(x_i)^{-e_2}\cdot{}g_1^{s_2}$, where $\rho,s_2\getr\Z_p$, to
  $V$.
\item $V$ sends $e\getr\Z_p$ to $P$.
  \item $P$ calculates $e_1 = e - e_2$, sends $e_1,e_2,s_1=\rho_1+e_1\cdot{}r$, and $s_2$ to $V$.

\item $V$ checks $e\sr{}e_1+e_2$, $g_1^{s_1}\sr{}\left(\frac{\com(x_i)}{g_2}\right)^{e_1}\cdot{}t_1$ and $g_1^{s_2}\sr{}\com(x_i)^{e_2}\cdot{}t_2$.
\end{enumerate}

\ignore{If $x_i=0$ then $P$ will modify steps 1 and 3 so that they ``cheat''
on the other side of the proof:

\begin{enumerate}[leftmargin=*]
\item [(1)] $P$ sends $t_1=c_1^{-e_1}\cdot{}g_1^{\rho}\cdot{}g$ and $t_2=g_1^{\rho'}$.
\item [(3)] $P$ calculates $e_2 = e - e_1$, sends $e_1, e_2, s_1=\rho, s_2=\rho'+e_2\cdot{}r$.
\end{enumerate}
}

\subsubsection{$\fzk{sum}$: Proof of Sum of Plaintexts equals $1$ }
\label{pkseo}
For commitments $\com(x)=g_1^r\cdot{}g_2^x$ and
$\com(1-x)=g_1^{r'}\cdot{}g_2^{1-x}$, $P$ shows that the sum of
plaintexts equals $1$.

  \begin{enumerate}[leftmargin=*]
  \item $P$ and $V$ compute
    $\com(1)=\com(x)\cdot{}\com(1-x)=g_1^{r+r'}\cdot{}g_2$.
\item $P$ proves that $\com(1)$ is a commitment to $1$ (see
  \S\ref{poe}).
  \end{enumerate}


\ignore{
\subsubsection{$\fzk{mul}$: Proof of Scalar Multiplication with Group Elements }
\label{pomult}
Let a party commit to $x$ with commitment
$\com(x) =g_1^r\cdot{}g_2^x$.  Given two elements $(A,B)$ of DDH group
$\myG$, such as an Elgamal ciphertext tuple, this party can then prove
in ZK that $(C=A^x,D=B^x)$ are the result of exponentiation with $x$,
i.e., scalar multiplication of $x$ with underlying plaintexts.


\begin{enumerate}[leftmargin=*]
      \item $P$ sends $t_1=A^{\rho_1},t_2=B^{\rho_1},
        t_3=g_1^{\rho_2}\cdot{}g_2^{\rho_1}$, for randomly chosen
        $\rho_i\getr\Z_p$, to $V$.

      \item $V$ sends challenge $e\getr\Z_p$.

      \item $P$ sends $s_1=\rho_1+e\cdot{}x,s_2=\rho_2+e\cdot{}r$.
        \item $V$ checks $A^{s_1}\sr{}C^e\cdot{}t_1$,
          $B^{s_1}\sr{}D^e\cdot{}t_2$, and
          $g_1^{s_2}\cdot{}g_2^{s_1}\sr{}\com(x)^e\cdot{}t_3$.
          
      \end{enumerate}   
}%ignore
\subsubsection{$\fzk{ExR}$: Proof of Exponentiation and Re-Encryption }
\label{pexr}
  One can
      efficiently prove correctness of combinations of linear operations  in one step. 
      We present the  
      example for the correctness of exponentiation of
      two elements $(A,B)$ from group $\myG$ with a committed value $x$
      and then multiplying $A^x$ by $g_1^{r'}$ and $B^x$ by $pk^{r'}$ from our protocol. So, this
      can be used to prove correct exponentiation (homomorphic scalar multiplication) of an Elgamal ciphertext
      by a previously committed scalar  $x$ and subsequent re-randomization of
      the result (homomorphic addition of Elgamal encryption of $0$).

      Specifically, given two group elements $(A,B)$ and commitment
      $\com(x) =g_1^{r}\cdot{}g_2^x$, prove correctness that
      $(C=g_1^{r'}\cdot{}A^x,D=pk^{r'}\cdot{}B^x)$ are the result of
      exponentiation with $x$ and multiplying with $g_1^{r'}$ and
      $pk^{r'}$, $r'\getr\Z_p$, known to $P$.

\begin{enumerate}[leftmargin=*]
  \item $P$ sends $t_1=g_1^{\rho_1}\cdot{}A^{\rho_2},t_2=pk^{\rho_1}\cdot{}B^{\rho_2},t_3=g_1^{\rho_3}\cdot{}g_2^{\rho_2}$ to $V$.
  \item $V$ sends $e\getr\Z_p$ to $P$.
    \item $P$ sends $s_1=\rho_1+e\cdot{}r'$, $s_2=\rho_2+e\cdot{}x$,
      and $s_3=\rho_3+e\cdot{}r$ to $V$.
\item $V$ checks whether $g_1^{s_1}\cdot{}A^{s_2}\sr{}C^e\cdot{}t_1$,
  $pk^{s_1}\cdot{}B^{s_2}\sr{}D^e\cdot{}t_2$, and
  $g_1^{s_3}\cdot{}g_2^{s_2}\sr{}\com(x)^e\cdot{}t_3$.
\end{enumerate}

      
\ignore{
\section{Old ZK Tools}
\subsubsection{ZK Proofs for Exponents}
\subsubsection{Proof of Plaintext Equivalence}
Let $c_1=(c_1[0],c_1[1],)=(g_1^{r_1},pk^{r_1}\cdot{}g^m)$ and
$c_2=(c_2[0],c_2[1],)=(g_1^{r_2},pk^{r_2}\cdot{}g^m)$ be two
encryptions from $\enc_{pk}(m)$. To prove plaintext equivalence of
these two ciphertexts, the prover shows that
$(g_1,\frac{c_1[0]}{c_2[0]},pk,\frac{c_1[1]}{c_2[1]})$ is a DDH tuple.

To prove that some ciphertext $c_1$ encrypts a plaintext $m$ with
respect to base $g$, a simple trick for the prover is to compute
another encryption $c_2$ of $m$ with respect to base $g$, show
plaintext equivalence, and then open randomness of $c_2$.

\subsubsection{Proofs for Arithmetic with Pedersen Commitments}
We can do simple arithmetic on Pedersen Commitments.
\begin{itemize}
\item Addition: given $\com_g(a)$ and $\com_g(b)$, everybody can
  compute and thus verify commitment
  $\com_g(c)=\com_g(a)\cdot{}\com_g(b)$ which commits to
  $c=a+b$. Obviously, no other party than the one originally computing
  $\com_g(a)$ and $\com_g(b)$ can open $\com_g(c)$, but all parties
  know that $\com_g(c)$ is a commitment to $c=a+b$
  
\item Multiplication: a party committing
  \begin{align*}
  \com_g(a)=g_1^{r_a}\cdot{}g^a, \com_g(b)=g_1^{r_b}\cdot{}g^b,
  \com_g(c)=g_1^{r_c}\cdot{}g^{a\cdot{}b}
  \end{align*}
  can prove in ZK that
  $\com_g(c)$ commits to the product of the messages committed in
  $\com_g(a)$ and $\com_g(b)$.

  The trick is to rewrite
  $\com_g(c)=g_1^{r_c-a\cdot{}r_b}\cdot\com_g(b)^{a}$ and then prove
  that all commitments are well formed, and $\com_g(c)$ uses the same
  exponent $a$ as $\com_g(a)$, but with basis $\com_g(b)$ instead of
  $g$. Specifically,
  \begin{enumerate}
  \item $P$ computes and sends
    \begin{align*}
      t_1=g_1^{\rho_1}\cdot{}g^{\rho_2},
      t_2=g_1^{\rho_3}\cdot{}g^{\rho_4},
      t_3=g_1^{\rho_5}\cdot{}\com_g(b)^{\rho_2}
      \end{align*}
      for $\rho_i\getr\Z_p$. Observe that the same randomness $\rho_2$
      is used for the same witness $a$.
    \item $V$ replies by sending challenge $e\getr\Z_p$.
    \item $P$ sends
      \begin{align*}
        s_1&=\rho_1+e\cdot{}r_a,s_2=\rho_2+e\cdot{}a,s_3=\rho_3+e\cdot{}r_b,s_4=\rho_4+e\cdot{}b,\\s_5&=\rho_5+e\cdot{}(r_c-a\cdot{}r_b).
        \end{align*}
    \item $V$ checks
      \begin{align*}
        g_1^{s_1}\cdot{}g^{s_2}\sr{}\com_g(a)^e, g_1^{s_3}\cdot{}g^{s_4}\sr{}\com_g(b)^e, g_1^{s_5}\cdot{}\com_g(b)^{s_2}\sr{}\com_g(c)^e\cdot{}t_3.
        \end{align*}
\end{enumerate}

\end{itemize}
}%ignore


\subsubsection{Proof of Construction~\ref{const:ioprf}}
\label{mainproof}
We now turn to our main proof, showing that
Construction~\ref{const:ioprf} is a secure $\ioprf$. We prove in the
hybrid model, using ZK hybrids with their abbrevations as introduced
in the previous section.  Recall that, in the hybrid model, ZK hybrids
are run by separate trusted third parties. Yet, during simulation, it
is the simulator who takes the role of the TTP and thus automatically
gets the adversary's inputs and can also cheat, see \citet{howto} for
details.

\begin{theorem}
  Assume that Construction~\ref{const:newprf} is an iterative
  pseudo-random function family $\iprf_K(\cdot)$.  Then,
  Construction~\ref{const:ioprf} is an $\ioprf$, realizing
  functionality $\fioprf$ in the
  $\left(\fzk{enc},\fzk{pop},\fzk{bit},\fzk{sum},\fzk{ExR}\right)$
  hybrid-model.
\end{theorem}

\begin{proof}
  First, observe that Construction~\ref{const:ioprf} is correct. Let
  $x$ be the receiver's input, and $K$ the key chosen by the
  sender. If both sender and receiver are honest, then the sender
  outputs nothing, and the receiver outputs
  $(v_1,\ldots,v_\ell)=\iprf_K(x)$. Thus, we focus on proving security
  and build simulators for two cases: one where $S$ is compromised,
  and one where $R$ is compromised.

  We will show that a simulator $\myS$ can be constructed from both
  the perspective of $S$ and $R$ such that the adversary $\A$'s view
  is indistinguishable from real executions of the protocol.  Thus we
  show that neither a compromised $S$ nor a compromised $R$ learn
  anything from the real execution of Construction~\ref{const:ioprf}
  beyond what is specified by the ideal functionality in
  Figure~\ref{idealioprf}.

  In our presentation below, we will use the term
  ``$\myS$ \emph{aborts''} as a shorthand for $\myS$ sending \abort to
  the TTP, simulating its party aborting to $\A$, and then outputting
  whatever $\A$ outputs.

In both cases below, the simulator will faithfully act as a verifier
  for ZKPs when interacting with $\A$ as necessary, aborting if the
  proof does not verify correctly. We omit these messages for
  readability since they require no special knowledge or behavior from
  the simulator. Our strategy will broadly be to:

\begin{itemize}[leftmargin=*]
  \item Replace Elgamal ciphertexts sent by $R$ with encryptions of
  zero (arbitrarily chosen).  Due to Elgamal's IND-CPA property, these
  ciphertexts will be indistinguishable from the real protocol for
  $\A$.  Since $S$ reveives no output from the real execution of the
  protocol, ciphertexts do not have to conform to any
  expectations.

\item Replace computation of $X_i$ and $Y_i$ by $S$
  in the real protocol with an encryption of the output of the
  $\ioprf$ received from the TTP.  $\myS$ does not know
  $K_i=(\alpha_i,\beta_i)$ and so cannot faithfully compute $X_i$ or
  $Y_i$, but it knows from the TTP what output $v_i$
  should. Consequently, $\myS$ crafts these values accordingly to
  simulate the real protocol and ``cheat'' in ZKPs where $\myS$ acts
  as the prover (see, e.g., \S~\ref{sec:extraction}).
\end{itemize}

Together, this will allow the simulator to generate a view which is
indistinguishable from a real execution, \ignore{\fixme{we said that
in the first paragraphs}while having now knowledge beyond that given
by the TTP,} thus proving that our construction is secure according to
Definition~\ref{def:ioprf}.

Note that also for all ZKPs with $\myS$ as a prover, $\myS$ acts as
the TTP and ``cheats'' to convince $\A$.  In many instances, $\myS$
could honestly prove to $\A$, so ``cheating'' is not really. Yet, for
ease of exposition, we assume that all proofs are simulated this way.
  
\vskip 1eX\noindent{\bf Case 1:} We assume that $\A$ has compromised
$S$ and build simulator $\myS$ taking the role of $S$ in the ideal
world, internally simulating a receiver to $\A$ which it only has
black box access to.

$\myS$ starts $\A$ and receives $2\ell$ commitments
$(\com(\alpha_i),\com{}(\beta_i))$ from $\A$. $\myS$ also receives
corresponding $(\alpha_i,\beta_i)$ together with random coins from
$\fzk{pop}$ sent from $\A$ to $\fzk{pop}$. If these do not match the
commitments, $\myS$ \emph{aborts}.
 

$\myS$ also
generates an Elgamal key pair $(sk, pk)$, sends
$pk$ to $\A$, and simulates $\fzk{enc}$.  Also, $\myS$ generates
$V_0=\enc_{pk}(0)$ and $D_0=\enc_{pk}(0)$, sends them to $\A$, and
simulates $\fzk{enc}$.
    
\noindent{}During the $i^\text{th}$ round,
  \begin{enumerate}[leftmargin=*]
  \item $\myS$ sends two independent commitments of zero and simulates
  $\fzk{bit}$ and $\fzk{sum}$.

  \item $\myS$ also computes and sends $(c_i,c'_i,d_i,d'_i)$, all
    encryptions of zero, to $\A$ and simulates
    $\fzk{ExR}$.

  \item $\myS$ receives $(X_i, Y_i)$ from $\A$ as well as
    $(\alpha'_i,\beta'_i)$ and random coins from $\fzk{ExR}$. If
    $\alpha_i\neq\alpha'_i$ or $\beta_i\neq\beta'_i$ or if random
    coins do not match computations specified in
    Construction~\ref{const:ioprf}, then $\myS$ \emph{aborts}. If they
    match, $\myS$ forwards $K_i=(\alpha_i,\beta_i)$ to the TTP.
   
  \item $\myS$ sends $P_i,P'_i,Q_i,Q'_i$, encryptions of zero, to $\A$
    and simulates $\fzk{ExR}$.
    
  \end{enumerate}
  $\myS$ outputs what $\A$ outputs.
During simulation, whenever $\A$ aborts, $\myS$ also \emph{aborts}.

\paragraph{Indistinguishable views} In the protocol, there are three types
of messages that $\myS$ sends to $\A$: Pedersen commitments, Elgamal
ciphertexts, and ZKP messages.  All of the Elgamal ciphertexts are
freshly encrypted (or re-encrypted) using fresh randomness.  They are
thus indistinguishable from any other Elgamal encryption, regardless
of any a priori knowledge that $\A$ might have.  As stated above, the
ZKPs are simulated and are thus also indistinguishable from a real
execution.  Finally, the commitments are perfectly hiding and are
never revealed during the protocol, so they are also indistinguishable
from the commitments of a real execution.

\vskip
1eX\noindent{\bf Case 2:} We assume that $\A$ has compromised $R$ and
build simulator $\myS$ as follows.

$\myS$ starts $\A$.
 $\myS$ randomly selects $\ell$ pairs $(\alpha'_i,\beta'_i)\getr(\Z_p)^2$,
  commits to them, sends commitments to $\A$, and proves knowledge of
  $(\alpha'_i,\beta'_i)$ using $\fzk{pop}$.

$\myS$ receives $pk$ from $\A$ and $(sk',pk')$ from
$\fzk{enc}$ which $\A$ has sent. If $pk\neq{}pk'$ or
$g_1^{sk'}\neq{}pk$, $\myS$ \emph{aborts}.  Also, $\myS$ receives
$(V_0,D_0)$ from $\A$ and $\A$'s random coins from $\fzk{enc}$. If
random coins do not match encryptions of $1$ ($V_0$) or $0$ ($D_0$),
$\myS$ \emph{aborts}.

\noindent{}During the $i^\text{th}$ round, 
\begin{enumerate}[leftmargin=*]
\item $\myS$ receives $(\com(x_i)$, $\com(1-x_i))$ from $\A$ and
  $(x'_i,1-y'_i)$ with the commitments' random coins from
  $\fzk{bit}$. If $x'_i$ or $1-y'_i$ and random coins do not match
  commitments, $\myS$ \emph{aborts}. In the same way, $\myS$ receives
  $z$ and a random coin for the commitment from sum hybrid
  $\fzk{sum}$. If $z\neq{}1$ or $z\neq{}x'_i+1-y'_i$ or the random
  coin does not match the commitment, $\myS$ \emph{aborts}. If
  everything matches, $\myS$ knows $\A$'s input $(x_i,1-x_i)$.

  $\myS$ receives $(c_i,c'_i,d_i,d'_i)$ from $\A$ and random coins and
  $(x'_i,1-y'_i)$ from $\fzk{ExR}$. If $(x'_i,1-y'_i)$ do not match
  the ones from the previous step or if any of the computations do not
  match $(c_i,c'_i,d_i,d'_i)$, $\myS$ \emph{aborts}.

  $\myS$ computes $(T_i,U_i)$ as in Construction~\ref{const:ioprf}.
  
\ignore{   Crucially, as $\myS$ knows $x_i$,
  they also know which of $T_i$ or $U_i$ contains the encryption of
  previous $\iprf$ output $v_{i-1}$ and which contains an encryption
  of $0$. If $x_i=1$, then $T_i$ contains the encryption of $v_{i-1}$
  (encryption of $1$ for $v_0$), and $U_i$ contains an encryption of
  $0$. If $x=0$, it is the other way around.}

\item $\myS$ queries the TTP for $x'_i$ and gets back $v_i$. If
  $x_i=1$, $\myS$ sets $X_i\leftarrow\enc_{pk}(v_i)$ and
  $Y_i\leftarrow\enc_{pk}(0)$.  If $x_i=0$, $\myS$ sets
  $X_i\leftarrow\enc_{pk}(0)$ and $Y_i\leftarrow\enc_{pk}(v_i)$.
  $\myS$ sends $(X_i,Y_i)$ to $\A$ and \emph{cheats} in $\fzk{ExR}$,
  convincing $\A$ that $(X_i,Y_i)$ are the result of raising $T_i$ and
  $U_i$ to $\alpha'_i$ and $\beta'_i$ and then re-encrypting.
  
\ignore{
  If $x_i = 1$,
meaning that $T_i$ contains the input from $\A$ that should be
included in the PRF, then $\myS$ computes $X_i = \enc_pk(y)$ and $Y_i = \enc_pk(0)$. If $x_i = 0$ it computes $Y_i
= \enc_pk(y)$ and $X_i = \enc_pk(0)$.  In either case, it also
``cheats'' the proofs $\fzk{pop}$ and $\fzk{ExR}$ by rewinding after
receiving the challenge and producing a correct commitment to match
the challenge.  This is necessary because $\myS$ does not know
$\alpha$ or $\beta$, only the final output of the $\ioprf$.
}

\item Finally, $\myS$ receives $(P_i,P'_i,Q_i,Q'_i)$ from $\A$ and
  random coins and $(x'_i,1-y'_i)$ from $\fzk{ExR}$. Again, $\myS$
  verifies correct computation of $(P_i,P'_i,Q_i,Q'_i)$ and whether
  $(x'_i,1-y'_i)$ match previously received values. If anything does
  not match, $\myS$ \emph{aborts}.

  $\myS$ computes $(V_i,D_i)$ as in Construction~\ref{const:ioprf}.
  
\end{enumerate}
  $\myS$ outputs what $\A$ outputs.
During simulation, whenever $\A$ aborts, also $\myS$ \emph{aborts}.

\paragraph{Indistinguishable views} As before, the commitments are
perfectly hiding and are not revealed and so are indistinguishable
from commitments of a real protocol execution.  ZKPs are also
simulated as before and are indistinguishable for the same reason.

The only part that is different in this case is the returned values of
$X_i$ and $Y_i$, which have to decrypt to the correct output of the
$\ioprf$ in order to match the real protocol.  Fortunately, $\myS$ can
query the TTP for the correct output and generate encryptions that
match that output.  In the real protocol, $S$ reencrypts $X_i$ and
$Y_i$ before returning them to $R$, and so they are indistinguishable
from the fresh encryptions generated by $\myS$.
\end{proof}
As $R$ verifies whether $S$ sends the same commitments to
$(\alpha_i,\beta_i)$ during multiple executions of
Construction~\ref{const:ioprf}, we trivially achieve verifiability.



\subsection{OT-based $\proto$ Construction}
Our $\iprf$ in Construction~\ref{const:newprf} can also be computed as
an $\ioprf$ with one sided security as follows.  Let $\ot(b, y_0,
y_1)$ denote any maliciously secure $\binom{2}{1}$ oblivious transfer
protocol between receiver $R$ and sender $S$, where $S$ holds $y_0$
and $y_1$ from $\Z_p$, $R$ holds $b\in\{0,1\}$, and $R$ obliviously
retrieves $y_b$ from $S$.  The OT-based version of $\proto$ works as
follows.

\begin{itemize}
\item $S$ generates $\ell$ random scalars $r_i\getr\Z_p$.
\item For each $1 \leq i \leq \ell$, $R$ and $S$ execute $\ot(x_i, r_i\beta_i,r_i\alpha_i)$, and $R$ stores the result as $z_i$.
\item $S$ sends to $R$ the vector $\vec{C}$ where $\forall 1 \leq i \leq \ell$, $C_i =  g^\frac{1}{\prod_{j=1}^{i} r_j}.$
\item $R$ recovers $\iprf$ output vector $\vec{v}$ by calculating $v_i
  = C_i^{\prod_{j=1}^{i} z_j}.$
\end{itemize}


{\bf Correctness:} \todo{update notation} For all $1 \leq i \leq \ell$ we have
\begin{equation}
\begin{aligned}
v_i &= C_i^{\prod_{j=1}^{i} z_i} \\
&= g^{\frac{1}{\prod_{j=1}^{i} r_j} \cdot \prod_{j=1}^{i} z_j} \\
&= g^{\frac{1}{\prod_{j=1}^{i} r_j} \cdot \prod_{j=1}^{i} (\alpha_jr_j)^{x_j}(\beta_jr_j)^{1-x_j} } \\
&= g^{\prod_{j=1}^{i} \alpha_j^{x_i}\beta_j^{1-x_i}}
\end{aligned}
\end{equation}

\todo{Show that this is an OPRF. And show that the iterative evaluation is still secure.}



\refstepcounter{construction}\label{const:ioprf}
\section{Construction~\ref{const:ioprf}: DH-based $\ioprf$}
\label{our-ioprf}
We now present a new $\proto$ protocol which realizes the ideal $\ioprf$
functionality $\fioprf$ from Figure~\ref{idealioprf}.

\subsection{Preliminaries}
%\subsubsection{Preliminaries}
Let there be two generators
$g_1,g_2$ of prime order $p$ group $\myG$ where the DDH assumption
holds. Neither party should know the discrete log of one generator
$g_i$ to the basis of the other generator $g_{j\neq{}i}$, which is
true with high probability if they are chosen at random.
%\fk{I just want to note that in a prime-order group, if the discrete logarithm $\log_{g_1}(g_2)$ is known, then so is $\log_{g_2}(g_1)$.}

\paragraph{Elgamal Encryption}
We will use additive Elgamal encryption with private keys $sk\in\Z_p$
and public keys $pk=g_1^{sk}$. Ciphertext $c$ to encrypt $m\in\Z_p$ is
$c=(c[0],c[1])=(g_1^r,pk^r\cdot{}g_2^m)\leftarrow\enc_{pk}(m)$, where
$r\getr\Z_p$.

\paragraph{Pedersen Commitments}
A Pedersen commitment $\com(m)\in\myG$ to message $m\in\Z_p$ is
defined as $\com(m)=g_1^r\cdot{}g_2^m$, where $r\getr\Z_p$.  To open
$\com(m)$, reveal tuple $(m,r)$. Pedersen commitments are perfectly
hiding and computationally binding.

\subsection{High-Level Intuition}
In round $i$ of $\ell$ rounds, sender $S$ will receive
two ciphertexts $V_i$ and $D_i$ from receiver $R$.  During the course of the
protocol, one of these ciphertexts will contain the $\ioprf$ output and one acts
as a ``dummy'', to keep $S$ from learning input bits $x_i$ of $R$.  They are
interchanged between rounds depending on the input bits.

For each round, using the $i^{\text{th}}$ round's keys $(\alpha_i,\beta_i)$, $S$
will then ``apply'' $\alpha_i$ to $V_i$ and $\beta_i$ to $D_i$, and send the
results back to $R$. In preparation for the next round $(i+1)$, if $x_{i+1}\neq
x_{i}$, $R$ will swap $V_i$ and $D_i$ for the next round.  After $\ell$ rounds,
$V_\ell$ will have the keys applied which correspond to the input bits of $R$, and
$D_\ell$ will have the complementary combination of keys applied.  $V_0$ is initialized
as an encryption of 1, so $V_\ell$ will contain the correct $\ioprf$ output, whereas
$D_0$ is initialized as an encryption of 0 so it will not contain any information.

\subsection{Technical Details}
For some input string $x=(x_1\ldots{}x_\ell)$, we define the output of
$\proto$ for the receiver as $(v_1,\ldots,v_\ell)=\ioprf_{K}(x)$ with
$v_i=g_2^{\prod_{j=1}^{i}(\alpha_j{}x_j+\beta_j(1-x_j))}$ and $K=\{(\alpha_i,\beta_i)\}^\ell_{i=1}$. 
We now describe details of Construction~\ref{const:ioprf} by its formal
$\proto$ interface (Definition~\ref{def:ioprf}), i.e., first its
initialization and then its iterative processing.

\subsubsection{$\proto$ Initialization}
Sender $S$ randomly chooses secret key
$K=((\alpha_1,\beta_1),\ldots,(\alpha_\ell,\beta_\ell)),
(\alpha_i,\beta_i)\getr(\Z_p)^2$.

$S$ also commits to $K$ by computing $2\ell$ Pedersen commitments
  $(\com(\alpha_i),\com(\beta_i))$. $S$ sends them to $R$ and
  proves knowledge of plaintexts in ZK (see \S\ref{pokop}).

Receiver $R$ computes a random
Elgamal private key $sk\getr\Z_p$ and public key $pk=g_1^{sk}$, and
sends $pk$ to $S$. Receiver $R$ proves knowledge of $sk$ using
a standard Schnorr ZK proof of knowledge (see \S\ref{poe}).


Receiver $R$ computes $V_0 \leftarrow\enc_{pk}(1)$ and
$D_0\leftarrow\enc_{pk}(0)$, sends them to $S$ and proves that
these are encryptions of $1$ and $0$ (see \S\ref{poe} below). 

\subsubsection{$\proto$ Iterative Processing in $\ell$ Rounds}
In round $i\in\{1,\ldots,\ell\}$, for $S$' input bit $x_i$:
\begin{enumerate}
  
\item {\bf Receiver shuffles:}
\begin{enumerate}%[leftmargin=0.3cm]
\item For input bit $x_i$, $R$ computes Pedersen commitment
  $\com{}(x_i)$ and proves that $x_i\in\{0,1\}$ (see
  \S\ref{pobit}). Similarly, $R$ computes $\com{}(1-x_i)$
  and proves that $(1-x_i)\in\{0,1\}$ (see \S\ref{pobit}). Finally,
  $R$ proves that the sum of plaintexts behind
  $\com{}(x_i)$ and $\com{}(1-x_i)$ equals $1$ (see
  \S\ref{pkseo}).


\item  Receiver $R$ chooses $r,r',r'',r'''\getr\Z_p$ and computes Elgamal ciphertexts
  \begin{align*}
    c_i&=(g_1^r\cdot{}V_{i-1}[0]^{x_i},pk^{r}\cdot{}V_{i-1}[1]^{x_i})
    \\c'_i&=(g_1^{r'}\cdot{}V_{i-1}[0]^{1-x_i},pk^{r'}\cdot{}V_{i-1}[1]^{1-x_i})
    \\d_i&=(g_1^{r''}\cdot{}D_{i-1}[0]^{x_i},pk^{r''}\cdot{}D_{i-1}[1]^{x_i})
    \\d'_i&=(g_1^{r'''}\cdot{}D_{i-1}[0]^{1-x_i},pk^{r'''}\cdot{}D_{i-1}[1]^{1-x_i})
 \end{align*} 
  \ignore{
    \vskip 1eX
\NoIndent{\begin{tabular}{@{}l@{\hskip 0.3cm}l}
    $c_i=(g_1^r\cdot{}V_{i-1}[0]^{x_i},pk^{r}\cdot{}V_{i-1}[1]^{x_i})$
    &$c'_i=(g_1^{r'}\cdot{}V_{i-1}[0]^{1-x_i},pk^{r'}\cdot{}V_{i-1}[1]^{1-x_i})$
    \\$d_i=(g_1^{r''}\cdot{}D_{i-1}[0]^{x_i},pk^{r''}\cdot{}D_{i-1}[1]^{x_i})$
    &$d'_i=(g_1^{r'''}\cdot{}D_{i-1}[0]^{1-x_i},pk^{r'''}\cdot{}D_{i-1}[1]^{1-x_i})$%\text{ and}
          \end{tabular}}
        }%ignore
  and sends $(c_i,c'_i,d_i,d'_i)$ to $S$.
\item Receiver $R$ proves correctness of the above computations in
  ZK. Specifically, $(c_i,c'_i,d_i,d'_i)$ result from correct
  exponentiation with $x_i$ (or $1-x_i$) from $\com{}(x_i)$ (or
  $\com{}(1-x_i)$), and multiplication with a random power of
  $g_1$ and $pk$, i.e., re-randomization (homomorphic addition of
  encryption of $0$).  See \S\ref{pexr} below for details.
   Both parties compute
\begin{align*}
   T_i&=(c_i[0]\cdot{}d'_i[0],c_i[1]\cdot{}d'_i[1])
    \\U_i&=(c'_i[0]\cdot{}d_i[0],c'_i[1]\cdot{}d_i[1]).
\end{align*}   
  \end{enumerate}
In the first round, after this step, $T_1$ is an encryption of $1$ and $U_1$ is an encryption of $0$ if $x_1 = 1$.
If $x_1 = 0$, then $T_1$ is an encryption of $0$ and $U_1$ is an encryption of $1$.
However,  sender $S$ does not know which of the two is the case.

\item {\bf Sender computes PRF:} For $r,r'\getr\Z_p$, $S$ computes the two Elgamal ciphertexts
  \begin{align*}
    X_i&=(g^r_1\cdot{}T_i[0]^{\alpha_i},pk^r\cdot{}T_i[1]^{\alpha_i})
    \\Y_i&=(g^{r'}_1\cdot{}U_i[0]^{\beta_i},pk^{r'}\cdot{}U_i[1]^{\beta_i}),
    \end{align*}
  sends $(X_i,Y_i)$ to $R$, and proves correct exponentiation
  (scalar multiplication of plaintexts) with $\alpha_i$ and $\beta_i$
  coming from previous commitments $\com{}(\alpha_i),\com{}(\beta_i)$
  \emph{and} re-randomization of ciphertexts (see \S\ref{pexr}).

\item {\bf Receiver shuffles back:}
  For $r,r',r'',r'''\getr\Z_p$, $R$ computes%\vskip 2eX
  \begin{align*}
    P_i&=(g_1^r\cdot{}X_i[0]^{x_i},pk^r\cdot{}X_i[1]^{x_i})
    \\P'_i&=(g_1^{r'}\cdot{}X_i[0]^{1-x_i},pk^{r'}\cdot{}X_i[1]^{1-x_i})
   \\Q_i&=(g_1^{r''}\cdot{}Y_i[0]^{x_i},pk^{r''}\cdot{}Y_i[1]^{x_i})
   \\Q'_i&=(g_1^{r'''}\cdot{}Y_i[0]^{1-x_i},pk^{r'''}\cdot{}Y_i[1]^{1-x_i})
\end{align*}
   \ignore{
     \begin{centering}
    \begin{tabular}{l@{\hskip 0.5cm}l}
    $P_i=(g_1^r\cdot{}X_i[0]^{x_i},pk^r\cdot{}X_i[1]^{x_i})$
    &$P'_i=(g_1^{r'}\cdot{}X_i[0]^{1-x_i},pk^{r'}\cdot{}X_i[1]^{1-x_i})$
   \\$Q_i=(g_1^{r''}\cdot{}Y_i[0]^{x_i},pk^{r''}\cdot{}Y_i[1]^{x_i})$
   &$Q'_i=(g_1^{r'''}\cdot{}Y_i[0]^{1-x_i},pk^{r'''}\cdot{}Y_i[1]^{1-x_i})$
  \end{tabular}
  \end{centering}
  \vskip 2eX
  }%ignore
  and sends $(P_i,P'_i,Q_i,Q'_i)$ together with ZK proofs of correct
  computation (see \S\ref{pexr}) to $S$.

  Both $S$ and $R$ compute
  $V_i=(P_i[0]\cdot{}Q'_i[0],P_i[1]\cdot{}Q'_i[1])$ and
  $D_i=(P'_i[0]\cdot{}Q_i[0],P'_i[1]\cdot{}Q_i[1])$.
  
In round $i$, after this step, $V_i$ is an encryption of $\iprf_{K}(x_1,\ldots,x_i)$, and $U_i$ is an encryption of $0$.
When computing $T_{i+1}$ and $U_{i+1}$, these values will be used instead of the encryptions of $0$ and $1$ and the iterative computation of the PRF continues.
Since both parties compute $V_i$ and $U_i$, $R$ cannot cheat and substitute for a value of his choice.

\item Receiver $R$ computes and outputs one $\iprf$ value
  $v_i=\frac{V_i[1]}{V_{i}[0]^{sk}}$.
\end{enumerate}

\paragraph{Discussion}
Observe that, in the last step, $R$ can never decrypt
additively homomorphic Elgamal ciphertext $(V_i[0],V_i[1])$ and thus
compute an $\alpha_i$ or $\beta_i$. As $\alpha_i$ or $\beta_i$ are in
the exponent and due to the hardness DLOG, $R$ can only
compute $v_i=g_2^{\ldots\alpha_i\ldots}$ or
$v_i=g_2^{\ldots\beta_i\ldots}$.
If $R$ wants to run several execution of
Construction~\ref{const:ioprf} and wants that $S$ uses the same key,
then $R$ will verify that commitments sent by $S$ during initialization do not change between executions. This
leads to {verifiability}.
Also note that communication complexity and computational complexity
are both in $O(\ell)$ per query, i.e.,  asymptotically
optimal.

%\vskip 1eX\noindent{\bf Security Analysis:} Due to space constraints, we defer %our full security analysis, including formal proofs, to Appendix~\ref{sec:sec-analysis}.
\section{Security Analysis}
%\subsection{Security Analysis}
\label{sec:sec-analysis}
We prove security of Construction~\ref{const:ioprf} using simulation
in the standard model. The simulation uses several efficient
Zero-Knowledge Proofs of Knowledge hybrids introduced first.  To ease
readability, we actually present Honest-Verifier Zero-Knowledge (HVZK)
versions of the proofs, but one can convert these to maliciously
verifier Zero-Knowledge proofs of knowledge using the following two
general transformations~\cite{efficient2pc}. We stress that we have
evaluated and benchmarked the full malicious verifier ZK proofs of
knowledge in Section~\ref{sec:implementation}, i.e., including the two
transformations.

\subsection{Zero Knowledge (instead of HVZK)}
\label{sec:extraction}
\ignore{We cannot use Fiat-Shamir transform and replace $e$, as we use
  Pedersen commitments for witness extraction.}
All our efficient ZK proofs below are three-move (``Sigma'') ZK
proofs. Recall that a three-move ZK proof comprises messages
$(t,e,s)$, where first message $t$ is a commitment from $P$ sent to
$V$, $e$ is $V$'s challenge sent to $P$, and $s$ is the final message
sent from $P$ to $V$.


To make these proofs zero-knowledge instead of only HVZK, we send an
additional message before first message $t$ of the regular three-move
proof.  In this new first message, $V$ sends a Pedersen commitment
$\com{}(e)=g_1^r\cdot{}g_2^e$ to their random challenge $e$ to
$V$. The proof  continues with $V$ sending their regular
commitment $t$ of the regular three-move proof and $V$ opening
$\com{}(e)$ by sending $(e,r)$. If $\com{}(e)$ matches
$(e,r)$, $P$ finally sends last message $s$ of the regular
proof. Verifier $V$ accepts, if $t$ and $s$ of the regular proof match
$e$.

This technique allows a simulator $\myS$ simulating $P$ to cheat in
the ZK proof. More specifically, after receiving $\com{}(e)$, $\myS$
internally computes a valid ZK proof $(t',e',s')$, assuming a random
challenge $e'$. $\myS$ sends $t'$ to $V$ and receives $(e,r)$. If
$(e,r)$ matches $\com{}(e)$, $\myS$ rewinds $V$ to the point after $V$
has sent $\com{}(e)$. Knowing $e$, $\myS$ computes a $t$ and $s$, such
that $(t,e,s)$ will be accepted by $V$. How exactly $t$ and $s$ are
chosen depends on the statement we want to prove, but are typically
straightforward for the Schnorr-style proofs we use below. We show an
example in \S\ref{pobit}.


\subsection{Witness Extraction for Pedersen Commitments}
To transform our ZK proofs to ZK proofs of knowledge, we rely on the
extractability of commitments.  Pedersen commitments are trapdoor
commitments which means that a party knowing a trapdoor $\rho$ can
open a commitment $\com(\cdot)$ to any plaintext they want
(equivocable).  We use this property for witness extraction in
three-move ZK proofs as follows.

Before starting the actual ZK proof by the first message $t$ from the
prover to the verifier, we send the following two messages.
\begin{enumerate}[leftmargin=*]
\item Prover $P$ sends to verifier $V$: $\hat{g}=g_1^\rho$ for random
  $\rho\getr\Z_p$. 
  \item Verifier $V$ will use this $\hat{g}$ instead of $g_2$ for the computation of the
    commitment to challenge $e$.  That is, $V$ computes and sends back
    commitment $\com(e)=g_1^r\cdot{}\hat{g}^{\,e}$ for their random challenge
    $e\in\Z_p$ as in the previous section.
\end{enumerate}

The ZK proof then continues as usual with $P$ sending $t$ and $V$
opening $\com(e)$ by sending $(e,r)$. If $(e,r)$ match $\com(e)$,
$P$ sends final message $s$ and $\rho$ to $V$. Only if
both is correct, the last ZK proof message $s$ matches $P$'s
commitment $t$ and challenge $e$, and $\rho$ matches $\hat{g}=g_1^\rho$, $V$
accepts.

This setup enables a simulator $\myS$ simulating $V$ to extract the
witness from $P$. After receiving trapdoor $\rho$ from $P$, $\myS$
rewinds $P$ until after the point were $P$ sends $t$ to $V$. Knowing
trapdoor $\rho$, $\myS$ can open $\com(e)$ to any $e'\neq{}e$ they
want by solving $r+\rho\cdot{}e=r'+\rho\cdot{}e'$ for $r'$, i.e., they
compute $r'=r+\rho\cdot{}(e-e')$. Running two executions of the ZK
proof with the same input and messages from $P$ but different
challenges extracts the witness of the ZK proof. Details on which $e$
to send in each execution again depend on the exact three-move ZK
proof, but are typically obvious. We refer to \citet{efficient2pc} for
more details.

In conclusion, these two transformation will render our three-move ZK
proofs below into (fully-maliciously secure) ZK proofs of knowledge. We
name each proof below with a hybrid which we will use in the main
proof later. So, for example, the hybrid for the proof of encryption
is called $\fzk{enc}$.


\subsection{ZK Building Blocks}
Before presenting our main proof of Construction~\ref{const:ioprf}, we
introduce the following ZK proofs that we use as building blocks.

\subsubsection{$\fzk{enc}$: Proof of Encryption/Commitment to $m$}
\label{poe}
To prove that an encryption
$c=(c[0],c[1])=(g_1^r,pk^r)\leftarrow\enc_{pk}(0)$ is an encryption of
$m=0$, $P$ proves that $(g_1,c[0],pk,c[1])$ is a DDH tuple.  You can
prove that tuple
$(u_1=g_1,u_2=g_1^r,u_3=g_1^{sk},u_4=g_1^{sk\cdot{}r})$ is a DDH tuple
using the \citet{cp92} protocol as follows.

\begin{enumerate}[leftmargin=*]
\item $P$ sends $(t_1=u_1^{\rho},t_2=u_3^{\rho})$ for $\rho\getr\Z_p$ to $V$.
  \item $V$ sends $e\getr\Z_p$ to $P$.
  \item $P$ sends $s=\rho+e\cdot{}r$ to $V$.
    \item $V$ accepts if $u_1^s=u_2^e\cdot{}t_1$ and $u_3^s=u_4^e\cdot{}t_2$.
\end{enumerate}

This proof has an important property.\ignore{ Besides showing that a
  tuple is a DDH tuple, it also shows DLOG equivalence, i.e.,
  $\log_{u_1}{u_2}=\log_{u_3}{u_4}$.} Instead of showing that some
ciphertext encrypts $m=0$, we can easily generalize it to show
encryption of arbitrary $m$. Specifically, we set
$c'[1]=\frac{c[1]}{g_2^m}$ and run the proof with $m = 0$ for new Elgamal
ciphertext $(c[0],c'[1])$.

%Finally, observe that Pedersen commitments are essentially just the
%right-hand side $c[1]$ of an Elgamal ciphertext. 
Finally, observe that Pedersen commitments are similarly structured as the
right-hand side $c[1]$ of an Elgamal ciphertext, just without the secret key.
Thus, to prove a
Pedersen commitment $\com(m)$ to $m$, parties divide $\com(m)$ by
$g_2^m$ and run a {\bf Schnorr proof} for $r$ used in the commitment ($P$
sends $t=g_1^\rho$, $V$ sends $e$, $P$ sends $s=\rho+e\cdot{}r$, and $V$
accepts if $g_1^s\sr\frac{\com(m)^e}{g_2^m}\cdot{}t$.)

\subsubsection{$\fzk{pop}$: Proof for Knowledge of Plaintext }
\label{pokop}
For $\com(m)=g_1^r\cdot{}g_2^m$,  prover $P$ can prove that they know
$m$.

\begin{enumerate}[leftmargin=*]
\item $P$ sends $t=g_1^{\rho_1}\cdot{}g_2^{\rho_2}$ for
  $\rho_1,\rho_2\getr\Z_p$ to $V$.
\item $V$ sends $e\getr\Z_p$ to $P$.
  \item $P$ sends $s_1=\rho_1+e\cdot{}r$ and $s_2=\rho_2+e\cdot{}m$ to
    $V$.
    \item $V$ checks whether $g_1^{s_1}\cdot{}g_2^{s_2}\sr\com(m)^e\cdot{}t$.
\end{enumerate}

\subsubsection{$\fzk{bit}$: Proof of Plaintext Bit }
\label{pobit}
For a commitment $\com(x_i)$, prover $P$ can prove that $x_i$ is a
bit, i.e., $x_i\in\{0,1\}$. This is an application of the
\emph{one-out-of-two} (OR) technique~\cite{ooot}. Essentially, $P$
proves that either $x_i=1$ which implies proving that ${\com(x_i)}$
equals ${g_1^{r_1}\cdot{}g_2}$ for some $r_1$, or $x_i=0$ which implies
proving that $\com(x_i)$ equals $g_1^{r_2}$ for some $r_2$. Proving
that ${\com(x_i)}$ equals ${g_1^{r_1}\cdot{}g_2}$ is equivalent to
proving that $\frac{{\com(x_i)}}{g_2}$ equals ${g_1^{r_1}}$.

$P$ will prove that they know (I) an $r$ such that
$g_1^{r}=\frac{{\com(x_i)}}{g_2}$ or (II) an $r$ such that
$g_1^{r}=\com(x_i)$. These are essentially two standard Schnorr
proofs.  The trick is that $P$ chooses $e_1$ and $e_2$ such that, for
the verifier's challenge $e$, we have $e=e_1+e_2$. Prover $P$ proves
knowledge of $r_1$ for (I) using challenge $e_1$ and knowledge of
$r_2$ for (II) using challenge $e_2$. Thus, $P$ can choose either
$e_1$ or $e_2$ before sending their first message of the ZK proof and
cheat in one proof. Without loss of generality, let $x_i=1$, so $P$
will cheat in proof (II). This works as follows.

\begin{enumerate}[leftmargin=*]
\item $P$ sends $t_1=g_1^{\rho_1}$ and
  $t_2=\com(x_i)^{-e_2}\cdot{}g_1^{s_2}$, where $\rho,s_2\getr\Z_p$, to
  $V$.
\item $V$ sends $e\getr\Z_p$ to $P$.
  \item $P$ calculates $e_1 = e - e_2$, sends $e_1,e_2,s_1=\rho_1+e_1\cdot{}r$, and $s_2$ to $V$.

\item $V$ checks $e\sr{}e_1+e_2$, $g_1^{s_1}\sr{}\left(\frac{\com(x_i)}{g_2}\right)^{e_1}\cdot{}t_1$ and $g_1^{s_2}\sr{}\com(x_i)^{e_2}\cdot{}t_2$.
\end{enumerate}

\ignore{If $x_i=0$ then $P$ will modify steps 1 and 3 so that they ``cheat''
on the other side of the proof:

\begin{enumerate}[leftmargin=*]
\item [(1)] $P$ sends $t_1=c_1^{-e_1}\cdot{}g_1^{\rho}\cdot{}g$ and $t_2=g_1^{\rho'}$.
\item [(3)] $P$ calculates $e_2 = e - e_1$, sends $e_1, e_2, s_1=\rho, s_2=\rho'+e_2\cdot{}r$.
\end{enumerate}
}

\subsubsection{$\fzk{sum}$: Proof of Sum of Plaintexts equals $1$ }
\label{pkseo}
For commitments $\com(x)=g_1^r\cdot{}g_2^x$ and
$\com(1-x)=g_1^{r'}\cdot{}g_2^{1-x}$, $P$ shows that the sum of
plaintexts equals $1$.

  \begin{enumerate}[leftmargin=*]
  \item $P$ and $V$ compute
    $\com(1)=\com(x)\cdot{}\com(1-x)=g_1^{r+r'}\cdot{}g_2$.
\item $P$ proves that $\com(1)$ is a commitment to $1$ (see
  \S\ref{poe}).
  \end{enumerate}


\ignore{
\subsubsection{$\fzk{mul}$: Proof of Scalar Multiplication with Group Elements }
\label{pomult}
Let a party commit to $x$ with commitment
$\com(x) =g_1^r\cdot{}g_2^x$.  Given two elements $(A,B)$ of DDH group
$\myG$, such as an Elgamal ciphertext tuple, this party can then prove
in ZK that $(C=A^x,D=B^x)$ are the result of exponentiation with $x$,
i.e., scalar multiplication of $x$ with underlying plaintexts.


\begin{enumerate}[leftmargin=*]
      \item $P$ sends $t_1=A^{\rho_1},t_2=B^{\rho_1},
        t_3=g_1^{\rho_2}\cdot{}g_2^{\rho_1}$, for randomly chosen
        $\rho_i\getr\Z_p$, to $V$.

      \item $V$ sends challenge $e\getr\Z_p$.

      \item $P$ sends $s_1=\rho_1+e\cdot{}x,s_2=\rho_2+e\cdot{}r$.
        \item $V$ checks $A^{s_1}\sr{}C^e\cdot{}t_1$,
          $B^{s_1}\sr{}D^e\cdot{}t_2$, and
          $g_1^{s_2}\cdot{}g_2^{s_1}\sr{}\com(x)^e\cdot{}t_3$.
          
      \end{enumerate}   
}%ignore
\subsubsection{$\fzk{ExR}$: Proof of Exponentiation and Re-Encryption }
\label{pexr}
  One can
      efficiently prove correctness of combinations of linear operations  in one step. 
      We present the  
      example for the correctness of exponentiation of
      two elements $(A,B)$ from group $\myG$ with a committed value $x$
      and then multiplying $A^x$ by $g_1^{r'}$ and $B^x$ by $pk^{r'}$ from our protocol. So, this
      can be used to prove correct exponentiation (homomorphic scalar multiplication) of an Elgamal ciphertext
      by a previously committed scalar  $x$ and subsequent re-randomization of
      the result (homomorphic addition of Elgamal encryption of $0$).

      Specifically, given two group elements $(A,B)$ and commitment
      $\com(x) =g_1^{r}\cdot{}g_2^x$, prove correctness that
      $(C=g_1^{r'}\cdot{}A^x,D=pk^{r'}\cdot{}B^x)$ are the result of
      exponentiation with $x$ and multiplying with $g_1^{r'}$ and
      $pk^{r'}$, $r'\getr\Z_p$, known to $P$.

\begin{enumerate}[leftmargin=*]
  \item $P$ sends $t_1=g_1^{\rho_1}\cdot{}A^{\rho_2},t_2=pk^{\rho_1}\cdot{}B^{\rho_2},t_3=g_1^{\rho_3}\cdot{}g_2^{\rho_2}$ to $V$.
  \item $V$ sends $e\getr\Z_p$ to $P$.
    \item $P$ sends $s_1=\rho_1+e\cdot{}r'$, $s_2=\rho_2+e\cdot{}x$,
      and $s_3=\rho_3+e\cdot{}r$ to $V$.
\item $V$ checks whether $g_1^{s_1}\cdot{}A^{s_2}\sr{}C^e\cdot{}t_1$,
  $pk^{s_1}\cdot{}B^{s_2}\sr{}D^e\cdot{}t_2$, and
  $g_1^{s_3}\cdot{}g_2^{s_2}\sr{}\com(x)^e\cdot{}t_3$.
\end{enumerate}

      
\ignore{
\section{Old ZK Tools}
\subsubsection{ZK Proofs for Exponents}
\subsubsection{Proof of Plaintext Equivalence}
Let $c_1=(c_1[0],c_1[1],)=(g_1^{r_1},pk^{r_1}\cdot{}g^m)$ and
$c_2=(c_2[0],c_2[1],)=(g_1^{r_2},pk^{r_2}\cdot{}g^m)$ be two
encryptions from $\enc_{pk}(m)$. To prove plaintext equivalence of
these two ciphertexts, the prover shows that
$(g_1,\frac{c_1[0]}{c_2[0]},pk,\frac{c_1[1]}{c_2[1]})$ is a DDH tuple.

To prove that some ciphertext $c_1$ encrypts a plaintext $m$ with
respect to base $g$, a simple trick for the prover is to compute
another encryption $c_2$ of $m$ with respect to base $g$, show
plaintext equivalence, and then open randomness of $c_2$.

\subsubsection{Proofs for Arithmetic with Pedersen Commitments}
We can do simple arithmetic on Pedersen Commitments.
\begin{itemize}
\item Addition: given $\com_g(a)$ and $\com_g(b)$, everybody can
  compute and thus verify commitment
  $\com_g(c)=\com_g(a)\cdot{}\com_g(b)$ which commits to
  $c=a+b$. Obviously, no other party than the one originally computing
  $\com_g(a)$ and $\com_g(b)$ can open $\com_g(c)$, but all parties
  know that $\com_g(c)$ is a commitment to $c=a+b$
  
\item Multiplication: a party committing
  \begin{align*}
  \com_g(a)=g_1^{r_a}\cdot{}g^a, \com_g(b)=g_1^{r_b}\cdot{}g^b,
  \com_g(c)=g_1^{r_c}\cdot{}g^{a\cdot{}b}
  \end{align*}
  can prove in ZK that
  $\com_g(c)$ commits to the product of the messages committed in
  $\com_g(a)$ and $\com_g(b)$.

  The trick is to rewrite
  $\com_g(c)=g_1^{r_c-a\cdot{}r_b}\cdot\com_g(b)^{a}$ and then prove
  that all commitments are well formed, and $\com_g(c)$ uses the same
  exponent $a$ as $\com_g(a)$, but with basis $\com_g(b)$ instead of
  $g$. Specifically,
  \begin{enumerate}
  \item $P$ computes and sends
    \begin{align*}
      t_1=g_1^{\rho_1}\cdot{}g^{\rho_2},
      t_2=g_1^{\rho_3}\cdot{}g^{\rho_4},
      t_3=g_1^{\rho_5}\cdot{}\com_g(b)^{\rho_2}
      \end{align*}
      for $\rho_i\getr\Z_p$. Observe that the same randomness $\rho_2$
      is used for the same witness $a$.
    \item $V$ replies by sending challenge $e\getr\Z_p$.
    \item $P$ sends
      \begin{align*}
        s_1&=\rho_1+e\cdot{}r_a,s_2=\rho_2+e\cdot{}a,s_3=\rho_3+e\cdot{}r_b,s_4=\rho_4+e\cdot{}b,\\s_5&=\rho_5+e\cdot{}(r_c-a\cdot{}r_b).
        \end{align*}
    \item $V$ checks
      \begin{align*}
        g_1^{s_1}\cdot{}g^{s_2}\sr{}\com_g(a)^e, g_1^{s_3}\cdot{}g^{s_4}\sr{}\com_g(b)^e, g_1^{s_5}\cdot{}\com_g(b)^{s_2}\sr{}\com_g(c)^e\cdot{}t_3.
        \end{align*}
\end{enumerate}

\end{itemize}
}%ignore


\subsubsection{Proof of Construction~\ref{const:ioprf}}
\label{mainproof}
We now turn to our main proof, showing that
Construction~\ref{const:ioprf} is a secure $\ioprf$. We prove in the
hybrid model, using ZK hybrids with their abbrevations as introduced
in the previous section.  Recall that, in the hybrid model, ZK hybrids
are run by separate trusted third parties. Yet, during simulation, it
is the simulator who takes the role of the TTP and thus automatically
gets the adversary's inputs and can also cheat, see \citet{howto} for
details.

\begin{theorem}
  Assume that Construction~\ref{const:newprf} is an iterative
  pseudo-random function family $\iprf_K(\cdot)$.  Then,
  Construction~\ref{const:ioprf} is an $\ioprf$, realizing
  functionality $\fioprf$ in the
  $\left(\fzk{enc},\fzk{pop},\fzk{bit},\fzk{sum},\fzk{ExR}\right)$
  hybrid-model.
\end{theorem}

\begin{proof}
  First, observe that Construction~\ref{const:ioprf} is correct. Let
  $x$ be the receiver's input, and $K$ the key chosen by the
  sender. If both sender and receiver are honest, then the sender
  outputs nothing, and the receiver outputs
  $(v_1,\ldots,v_\ell)=\iprf_K(x)$. Thus, we focus on proving security
  and build simulators for two cases: one where $S$ is compromised,
  and one where $R$ is compromised.

  We will show that a simulator $\myS$ can be constructed from both
  the perspective of $S$ and $R$ such that the adversary $\A$'s view
  is indistinguishable from real executions of the protocol.  Thus we
  show that neither a compromised $S$ nor a compromised $R$ learn
  anything from the real execution of Construction~\ref{const:ioprf}
  beyond what is specified by the ideal functionality in
  Figure~\ref{idealioprf}.

  In our presentation below, we will use the term
  ``$\myS$ \emph{aborts''} as a shorthand for $\myS$ sending \abort to
  the TTP, simulating its party aborting to $\A$, and then outputting
  whatever $\A$ outputs.

In both cases below, the simulator will faithfully act as a verifier
  for ZKPs when interacting with $\A$ as necessary, aborting if the
  proof does not verify correctly. We omit these messages for
  readability since they require no special knowledge or behavior from
  the simulator. Our strategy will broadly be to:

\begin{itemize}[leftmargin=*]
  \item Replace Elgamal ciphertexts sent by $R$ with encryptions of
  zero (arbitrarily chosen).  Due to Elgamal's IND-CPA property, these
  ciphertexts will be indistinguishable from the real protocol for
  $\A$.  Since $S$ reveives no output from the real execution of the
  protocol, ciphertexts do not have to conform to any
  expectations.

\item Replace computation of $X_i$ and $Y_i$ by $S$
  in the real protocol with an encryption of the output of the
  $\ioprf$ received from the TTP.  $\myS$ does not know
  $K_i=(\alpha_i,\beta_i)$ and so cannot faithfully compute $X_i$ or
  $Y_i$, but it knows from the TTP what output $v_i$
  should. Consequently, $\myS$ crafts these values accordingly to
  simulate the real protocol and ``cheat'' in ZKPs where $\myS$ acts
  as the prover (see, e.g., \S~\ref{sec:extraction}).
\end{itemize}

Together, this will allow the simulator to generate a view which is
indistinguishable from a real execution, \ignore{\fixme{we said that
in the first paragraphs}while having now knowledge beyond that given
by the TTP,} thus proving that our construction is secure according to
Definition~\ref{def:ioprf}.

Note that also for all ZKPs with $\myS$ as a prover, $\myS$ acts as
the TTP and ``cheats'' to convince $\A$.  In many instances, $\myS$
could honestly prove to $\A$, so ``cheating'' is not really. Yet, for
ease of exposition, we assume that all proofs are simulated this way.
  
\vskip 1eX\noindent{\bf Case 1:} We assume that $\A$ has compromised
$S$ and build simulator $\myS$ taking the role of $S$ in the ideal
world, internally simulating a receiver to $\A$ which it only has
black box access to.

$\myS$ starts $\A$ and receives $2\ell$ commitments
$(\com(\alpha_i),\com{}(\beta_i))$ from $\A$. $\myS$ also receives
corresponding $(\alpha_i,\beta_i)$ together with random coins from
$\fzk{pop}$ sent from $\A$ to $\fzk{pop}$. If these do not match the
commitments, $\myS$ \emph{aborts}.
 

$\myS$ also
generates an Elgamal key pair $(sk, pk)$, sends
$pk$ to $\A$, and simulates $\fzk{enc}$.  Also, $\myS$ generates
$V_0=\enc_{pk}(0)$ and $D_0=\enc_{pk}(0)$, sends them to $\A$, and
simulates $\fzk{enc}$.
    
\noindent{}During the $i^\text{th}$ round,
  \begin{enumerate}[leftmargin=*]
  \item $\myS$ sends two independent commitments of zero and simulates
  $\fzk{bit}$ and $\fzk{sum}$.

  \item $\myS$ also computes and sends $(c_i,c'_i,d_i,d'_i)$, all
    encryptions of zero, to $\A$ and simulates
    $\fzk{ExR}$.

  \item $\myS$ receives $(X_i, Y_i)$ from $\A$ as well as
    $(\alpha'_i,\beta'_i)$ and random coins from $\fzk{ExR}$. If
    $\alpha_i\neq\alpha'_i$ or $\beta_i\neq\beta'_i$ or if random
    coins do not match computations specified in
    Construction~\ref{const:ioprf}, then $\myS$ \emph{aborts}. If they
    match, $\myS$ forwards $K_i=(\alpha_i,\beta_i)$ to the TTP.
   
  \item $\myS$ sends $P_i,P'_i,Q_i,Q'_i$, encryptions of zero, to $\A$
    and simulates $\fzk{ExR}$.
    
  \end{enumerate}
  $\myS$ outputs what $\A$ outputs.
During simulation, whenever $\A$ aborts, $\myS$ also \emph{aborts}.

\paragraph{Indistinguishable views} In the protocol, there are three types
of messages that $\myS$ sends to $\A$: Pedersen commitments, Elgamal
ciphertexts, and ZKP messages.  All of the Elgamal ciphertexts are
freshly encrypted (or re-encrypted) using fresh randomness.  They are
thus indistinguishable from any other Elgamal encryption, regardless
of any a priori knowledge that $\A$ might have.  As stated above, the
ZKPs are simulated and are thus also indistinguishable from a real
execution.  Finally, the commitments are perfectly hiding and are
never revealed during the protocol, so they are also indistinguishable
from the commitments of a real execution.

\vskip
1eX\noindent{\bf Case 2:} We assume that $\A$ has compromised $R$ and
build simulator $\myS$ as follows.

$\myS$ starts $\A$.
 $\myS$ randomly selects $\ell$ pairs $(\alpha'_i,\beta'_i)\getr(\Z_p)^2$,
  commits to them, sends commitments to $\A$, and proves knowledge of
  $(\alpha'_i,\beta'_i)$ using $\fzk{pop}$.

$\myS$ receives $pk$ from $\A$ and $(sk',pk')$ from
$\fzk{enc}$ which $\A$ has sent. If $pk\neq{}pk'$ or
$g_1^{sk'}\neq{}pk$, $\myS$ \emph{aborts}.  Also, $\myS$ receives
$(V_0,D_0)$ from $\A$ and $\A$'s random coins from $\fzk{enc}$. If
random coins do not match encryptions of $1$ ($V_0$) or $0$ ($D_0$),
$\myS$ \emph{aborts}.

\noindent{}During the $i^\text{th}$ round, 
\begin{enumerate}[leftmargin=*]
\item $\myS$ receives $(\com(x_i)$, $\com(1-x_i))$ from $\A$ and
  $(x'_i,1-y'_i)$ with the commitments' random coins from
  $\fzk{bit}$. If $x'_i$ or $1-y'_i$ and random coins do not match
  commitments, $\myS$ \emph{aborts}. In the same way, $\myS$ receives
  $z$ and a random coin for the commitment from sum hybrid
  $\fzk{sum}$. If $z\neq{}1$ or $z\neq{}x'_i+1-y'_i$ or the random
  coin does not match the commitment, $\myS$ \emph{aborts}. If
  everything matches, $\myS$ knows $\A$'s input $(x_i,1-x_i)$.

  $\myS$ receives $(c_i,c'_i,d_i,d'_i)$ from $\A$ and random coins and
  $(x'_i,1-y'_i)$ from $\fzk{ExR}$. If $(x'_i,1-y'_i)$ do not match
  the ones from the previous step or if any of the computations do not
  match $(c_i,c'_i,d_i,d'_i)$, $\myS$ \emph{aborts}.

  $\myS$ computes $(T_i,U_i)$ as in Construction~\ref{const:ioprf}.
  
\ignore{   Crucially, as $\myS$ knows $x_i$,
  they also know which of $T_i$ or $U_i$ contains the encryption of
  previous $\iprf$ output $v_{i-1}$ and which contains an encryption
  of $0$. If $x_i=1$, then $T_i$ contains the encryption of $v_{i-1}$
  (encryption of $1$ for $v_0$), and $U_i$ contains an encryption of
  $0$. If $x=0$, it is the other way around.}

\item $\myS$ queries the TTP for $x'_i$ and gets back $v_i$. If
  $x_i=1$, $\myS$ sets $X_i\leftarrow\enc_{pk}(v_i)$ and
  $Y_i\leftarrow\enc_{pk}(0)$.  If $x_i=0$, $\myS$ sets
  $X_i\leftarrow\enc_{pk}(0)$ and $Y_i\leftarrow\enc_{pk}(v_i)$.
  $\myS$ sends $(X_i,Y_i)$ to $\A$ and \emph{cheats} in $\fzk{ExR}$,
  convincing $\A$ that $(X_i,Y_i)$ are the result of raising $T_i$ and
  $U_i$ to $\alpha'_i$ and $\beta'_i$ and then re-encrypting.
  
\ignore{
  If $x_i = 1$,
meaning that $T_i$ contains the input from $\A$ that should be
included in the PRF, then $\myS$ computes $X_i = \enc_pk(y)$ and $Y_i = \enc_pk(0)$. If $x_i = 0$ it computes $Y_i
= \enc_pk(y)$ and $X_i = \enc_pk(0)$.  In either case, it also
``cheats'' the proofs $\fzk{pop}$ and $\fzk{ExR}$ by rewinding after
receiving the challenge and producing a correct commitment to match
the challenge.  This is necessary because $\myS$ does not know
$\alpha$ or $\beta$, only the final output of the $\ioprf$.
}

\item Finally, $\myS$ receives $(P_i,P'_i,Q_i,Q'_i)$ from $\A$ and
  random coins and $(x'_i,1-y'_i)$ from $\fzk{ExR}$. Again, $\myS$
  verifies correct computation of $(P_i,P'_i,Q_i,Q'_i)$ and whether
  $(x'_i,1-y'_i)$ match previously received values. If anything does
  not match, $\myS$ \emph{aborts}.

  $\myS$ computes $(V_i,D_i)$ as in Construction~\ref{const:ioprf}.
  
\end{enumerate}
  $\myS$ outputs what $\A$ outputs.
During simulation, whenever $\A$ aborts, also $\myS$ \emph{aborts}.

\paragraph{Indistinguishable views} As before, the commitments are
perfectly hiding and are not revealed and so are indistinguishable
from commitments of a real protocol execution.  ZKPs are also
simulated as before and are indistinguishable for the same reason.

The only part that is different in this case is the returned values of
$X_i$ and $Y_i$, which have to decrypt to the correct output of the
$\ioprf$ in order to match the real protocol.  Fortunately, $\myS$ can
query the TTP for the correct output and generate encryptions that
match that output.  In the real protocol, $S$ reencrypts $X_i$ and
$Y_i$ before returning them to $R$, and so they are indistinguishable
from the fresh encryptions generated by $\myS$.
\end{proof}
As $R$ verifies whether $S$ sends the same commitments to
$(\alpha_i,\beta_i)$ during multiple executions of
Construction~\ref{const:ioprf}, we trivially achieve verifiability.


\section{Implementation}
\label{sec:implementation}
\begin{figure}[tb]\centering
  \includegraphics[width=\columnwidth]{plot.pdf}
  %\vspace*{-5mm}
  \caption{\label{figure}Total runtime of Construction~\ref{const:ioprf}}
\end{figure}
  \begin{table}[bb]\sisetup{detect-weight,mode=text}\small
    \centering\caption{\label{tab:imp}Cost breakdown}
    \sisetup{detect-weight,mode=text}
  \begin{tabular}{|c|cc|S[table-format=2.1]S[table-format=3.1]|cc|}
  \hline\multirow{2}{*}{$\ell$}& \multicolumn{2}{c|}{CPU (ms)}&\multicolumn{2}{c|}{Communication (kB)}&\multicolumn{2}{c|}{Total runtime (ms)}
  \\& Sender& Receiver & \multicolumn{1}{c}{Sender} & \multicolumn{1}{c|}{Receiver}& LAN1& WAN1
  \\\hline 5&44&41&11.7&26.1&171&2425
  \\10&88&81&23.2&51.4&325&4571
  \\15&126&123&34.6&76.6&512&6707
  \\20&174&162&46.1&101.9&679&8873
  \\25&218&202&57.5&127.1&836&10987
  \\30&267&248&69.0&152.4&968&13128
  \\\hline
\end{tabular}
\end{table}

To show practicality of Construction~\ref{const:ioprf} including its
ZK proofs, we have implemented and benchmarked its performance in
several realistic network settings. We stress that our implementation
is a full implementation of Construction~\ref{const:ioprf}, with all
Zero-Knowledge Proofs of Knowledge of Appendix~\ref{sec:sec-analysis},
i.e., including witness extractability and security against fully
malicious verifiers. Sender and receiver instances communicate via
standard TCP sockets.

Our implementation is done in $C$ and uses OpenSSL for elliptic curve
operations on NIST curve {\texttt{secp224r1}}. The source code is
available for download~\cite{srcode}.  We benchmark our implementation
on a 4.1~GHz Core i5-10600k.  As network latency is typically the
bottleneck in multi-round secure two-party computation protocols, we
benchmark Construction~\ref{const:ioprf} in different settings with
different network latencies.  To precisely control network latency
between sender and receiver instances, we use Linux' standard
{\texttt{tc-netem}} tool. Figure~\ref{figure} shows benchmark results
averaged over 50 executions, and Table~\ref{tab:imp} presents the cost
breakdown.

We measure total run time for values of $\ell$ ranging from $1$ to
$32$. Note that $\ell=32$ would support binary tree data structures
with $2^{32}$ different paths and $2^{33}-1$
(8.6~billion) nodes. We vary latency assuming LAN scenarios
with standard Gigabit Ethernet (0.5~ms RTT) or WiFi (2~ms RTT) and WAN
scenarios for intra-continental communication (30~ms RTT) and
inter-continental communication (70~ms RTT)~\cite{verizon}. We also
show an evaluation with 0~ms RTT, however even this number is still
dominated by the TCP communication overhead.  We found that the
computation alone in our protocol, including all EC computation and
ZKPs, is approximately 3~ms per $\ioprf$ iteration.

Each iOPRF evaluation for a tree data structure with $2^{20}$ nodes
needs about 170~ms of CPU time per party with our (unoptimized)
implementation.  As soon as we introduce higher latency, CPU time
contributes little to total runtime, and communication latency
becomes the main performance obstacle. For example, in the WAN1
scenario with intra-continental communication between sender and
receiver, total runtime is about 9~s of which less than $4\%$ is spent
with computation, and the remainder is consumed by network
latency.

We conclude from Figure~\ref{figure} that even for large values of
$\ell$ and for high latency network connections,
Construction~\ref{const:ioprf} has only a few seconds of runtime which
is practical for many scenarios.

In Appendix~\ref{sec:perf-related}, we discuss why alternative
approaches to realize the $\ioprf$ functionality perform worse than Construction~\ref{const:ioprf}.

\section{Applications}
\subsection{RFID}
Radio Frequency IDentification (RFID) applications comprise a large
quantity of RFID tags attached to precious goods and RFID readers
which are connected to a central backend database. The goal is that
readers can properly identify tags using wireless communication in the
presence of adversaries.

An adversary observing wireless tag-reader interaction or being able
to interact with tags themselves should not be able to identify or
trace tags or even fabricate new tags or clone tags to counterfeit
goods. The main technical challenge is that RFID tags are extremely
resource restricted and can merely compute a cryptographic hash
function. While readers can perform more powerful operations, they
typically feature low storage (no state), but have network
connectivity, e.g., to connect to a central database.  RFID security
has been a very active area of research, see \citet{juels} for an
overview.

In a typical scenario, the reader wants to know whether a tag and
therewith the good it is attached to is valid, by interacting first
with the tag and then with the database. Typically, the tag stores a
unique key, and the reader performs a challenge-response type of
authentication, using the database which knows all tags' keys.
However, previous work has assumed that database and readers are
within the same trust domain, as the database learns which tag the
reader is querying for.  This is an unnecessary strong and often
unrealistic requirement.  To protect tag privacy and internal details
of supply or distribution chains, the database should not learn which
tag the reader is querying for. For example, if several readers
successively query for the same tag, the database knows that a
specific tag has traveled between these readers. At the same time, the
database does not want to give unrestricted access to the reader or
allow queries which leak more information than necessary for the
identification of a single tag per query. If the reader would receive
more information, the danger would be that a reader fabricates tags.
To mitigate these problems, we show how we can extend a prominent RFID
security protocol from the literature, the one by \citet{molnar}, by a
simple application of our $\ioprf$.

\paragraph{High-Level Idea} In the original work by \citeauthor{molnar}, the
database prepares a binary key tree of height $\ell$ storing random
keys in nodes. Leaves in the tree are enumerated by their path from
the root to the leaf. For example, the left most leaf is represented
by the bit string of $\ell$ zero bits. A tag is uniquely identified by
its ID, a bit string $x=(x_1\ldots{}x_\ell)$. During initialization, a
tag with ID $x$ receives all keys from the root to the leaf
represented by $x$. During tag identification, the tag chooses a
random $r$, ``encrypts'' $r$ with each of their keys, and sends the
resulting sequence of ciphertexts to the reader. The reader can access
the database and query keys. The reader checks which path in the tree
decrypts and ends up with a specific ID (leaf). As you can see, this
protocol does not protect the tag from a prying database. A simple
solution of just sending the whole key tree to the reader might
overburden the reader's storage capabilities and also impose a
security risk: having access to all keys, the reader could fabricate 
an arbitrary number of tags.

Our modification to the \citeauthor{molnar} protocol simply consists
of exchanging the way keys in the tree are computed. In our case, the
keys are outputs of the $\ioprf$ which will allow the reader to query
the database for exactly one contiguous path.  As a result, we hide
from the database which tag the reader is querying for, and the
database knows that the reader only gets one path of secrets from the
tree and will be able to identify exactly one tag with it.

\subsubsection{Technical Details}
Let $N=2^\ell$ be the total number of tags in the system. Each tag
uniquely corresponds to a leaf of a height $\ell$ binary \emph{key
tree}. To identify a tag, a reader can communicate with the database
which stores all keys of the key tree.

\paragraph{Preliminaries}
The database knows a secret key $K$ and populates binary key tree
$T$ as follows. First, nodes in this key tree are indexed by bit
strings following the intuitive notation that the left child (``0'')
of some node indexed by bit string $\gamma_1\ldots\gamma_i$ is index by
$\gamma_1\ldots\gamma_i0$, and the right child (``1'') is indexed by
$\gamma_1\ldots\gamma_i1$. By convention, the root is indexed by
empty bit string $\epsilon$.

\paragraph{Database Initialization}
Root node $\epsilon$ stores random key
$K_{\epsilon}\getr\{0,1\}^\lambda$.  The left child of the root stores
key $K_0=\ioprf_K(0)$, and the right child stores key
$K_1=\ioprf_K(1)$. For a node $\gamma_1\ldots\gamma_i$,
the left child stores key
$K_{\gamma_1\ldots\gamma_i0}=\ioprf_K(\gamma_1\ldots\gamma_i0)$, and its
right child stores key
$K_{\gamma_1\ldots\gamma_i1}=\ioprf_K(\gamma_1\ldots\gamma_i1)$.

During authentication of tag $x$, the database will run
$\ioprf_K(\cdot)$ as the sender, and the reader will be the receiver
with input bit strings $x=(x_1\ldots{}x_\ell)$ as follows.

\paragraph{Tag Initialization}
During initialization of a new tag $x$, the database stores a sequence
of $(\ell+1)$ keys $K$ on the tag: one for each node on the path from
the root $K_\epsilon$ of tree $T$ to leaf
$K_x=K_{x_1\ldots{}x_\ell}$. The tag also stores its own ID $x$.

\paragraph{Secure Tag Identification}
 Each tag identifies itself to a reader using a variation of
  the \citeauthor{molnar} protocol:

  \begin{itemize}[leftmargin=*]
  \item Tag $x$ chooses
    $r\getr\{0,1\}^\lambda$ and sends $r$ together with a hash of $r$
    and each of their $(\ell+1)$ keys and the next bit,
    respectively. More formally, the tag sends 
$    \trace=(r,T_0=H(r,
    K_\epsilon,x_1),\ldots,T_\ell=H(r,K_{x_1\ldots{}x_{\ell-1}},x_{\ell}),
    H(r,K_{x_1\ldots{}x_\ell})).
    $

The difference to the original protocol is that we also include
next bit $x_i$ into each hash. This allows the reader to check which
node to query for during the next iteration. Otherwise, the reader
would have to retrieve both children of the current node, revealing 
``one more key'' per level of the tree to the reader.

\item The reader uses the $\ioprf$ as the receiver and the database as
  the sender to identify the tag as follows.

 \begin{itemize}
 \item The database begins by sending $K_\epsilon$ to the reader.
   
  \item The reader checks whether either $H(r,K_\epsilon,0)$ or $H(r,K_\epsilon,1)$  matches
    $T_0$.

  \item Depending on the outcome, the reader iteratively continues and
    queries either the left child ($H(r,K_\epsilon,0)$ matches) or the
    right child ($H(r,K_\epsilon,1)$ matches) of the root with the
    $\ioprf$, compute keys, checks which matches etc.
\end{itemize}
  \end{itemize}

As you can see, the security we are aiming for asks only for a
(delegatable) OPRF. Our $\ioprf$ supports delegation, but can do more. We
could also ask as an additional security requirement that the reader
should only learn ``one path'', i.e., one tag per interaction with the
database. 

 Due to space limitations,
we have moved the {\bf security analysis} to Appendix~\ref{sec:rf-proof}.

\section{RFID Security Analysis}
\label{sec:rf-proof}
To summarize security requirements, we briefly describe a reactive,
ideal functionality $\myF$.  The database sends their input, keys
$K_\epsilon,K_0,K_1,\ldots,K_{1\ldots{}1}$, all $(2N-1)$ keys of the
key tree, to a TTP, and the reader sends an empty bit string. Then,
the TTP sends $K_\epsilon$ to the reader, and nothing to the database.
The internal state $s$ of TTP is initialized to the empty bit string.
Then, the RFID reader and TTP additionally interact in a total of
$\ell$ rounds. In round $i$, let the internal state be bit string
$s=\gamma_1\ldots{}\gamma_{i-1}$. The reader sends bit $\gamma_i$, and
TTP responds with $K_{\gamma_1\ldots\gamma_{i}}$ and updates its
state to $s=\gamma_1\ldots\gamma_{i}$.

\begin{lemma}\label{rfidproof} In the random oracle model, the modified \citeauthor{molnar} protocol securely realizes 
ideal functionality $\myF$.
\end{lemma}

As the proof of Lemma~\ref{rfidproof} is straightforward, we only
summarize it in a draft.
\begin{proof}[Sketch] We build a simulator for the case of a compromised reader. The simulator for the case of a compromised database works accordingly.
  \begin{enumerate}

  \item Simulator $\myS$ begins by preparing an initially empty
    key-value table $\mathsf{RO}$ to implement a standard random
    oracle functionality $H(\cdot)$. During simulation, whenever any
    party calls $H(k)$ for some input $k$, this functionality will
    check whether pair $(k,v)$ is already in table $\mathsf{RO}$ and
    responds with $v$ in that case. Otherwise, $H$ generates a random
    string $v$ of length $\lambda$, sends $v$ back to the caller, and
    places $(k,v)$ in $\mathsf{RO}$.

  \item Also, $\myS$ generates a random key
    $K=((\alpha_1,\beta_1),\ldots,(\alpha_\ell,\beta_\ell))$ for $\ioprf$.
 $\myS$ sends $\epsilon$ to TTP and receives $K_\epsilon$ which it
    forwards to $\A$.
    
  \item $\myS$ and $\A$ run Construction~\ref{const:ioprf} with $\myS$
    as the sender and $\A$ as the receiver.

    During the $i^\text{th}$ iteration of Construction~\ref{const:ioprf}:
    \begin{enumerate}
    \item $\myS$
      extracts $\A$'s input $x_i$ from the Pedersen commitment,
      forwards it to TTP, and receives back $K_{x_1\ldots{}x_i}$.

    \item $\myS$ adds key-value pair $(g_2^{ \prod_{j=1}^{i}
      \alpha_j^{x_j}\beta_j^{1-x_j}},K_{x_1\ldots{}x_i})$ to table
      $\mathsf{RO}$.   
    \end{enumerate}
\end{enumerate}
Observe that $\A$'s view in the simulation is indistinguishable from their view in a real protocol execution.
\qed\end{proof}
Note that $\A$ can perform an input-substitution attack, i.e., query
for some path which does not match the tag they are currently
interacting with. Without the ability to perform public key
cryptography on the tag, the strongest security for the database one
can guarantee is that the reader can get one path, identifying one tag
and thus can fabricate or clone at most one tag per interaction.


\section{Conclusion}

In this paper, we have introduced the concept of an iteratable
oblivious pseudo-random function and presented a construction which is
provably secure in the standard model under the DDH assumption.  We
have fully implemented and evaluated this construction and shown that
it is efficient in practice, comparable to similar protocols.  We have
also presented several applications for $\ioprf$ protocols that
demonstrate their usefulness, particularly in applications where
(two-sided) malicious security is necessary.


\ignore{
\input{related}

  \section{Difference to structured encryption}
\begin{itemize}
\item Different adversary model
\item Matrix queries and labeled data queries, neighbor queries and adjacency queries on graphs, are trivial.
\item Token length?!
\item the original PRF is mentioned by Naor and Reingold (Section 6.3), but details on how to use OT is mentioned by \url{https://www.iacr.org/archive/tcc2005/3378_304/3378_304.pdf}.  
\end{itemize}
}


\ignore{
* Note that our iOPRF can be evaluated ``interactively'',  i.e., the receiver runs OTs adaptively

Motivation:

* One could just replace the PRF in structured encryption (Figure 2 /
Section 5) with an OPRF, but this is not sufficient: the adversary
could ``flip-flop'' inside the graph, but we want that they can only
follow paths.


Apps:
* Graphs: https://robobees.seas.harvard.edu/files/privacytools/files/grecs.pdf
and https://par.nsf.gov/servlets/purl/10042572 and http://www.vldb.org/pvldb/vol11/p420-sahu.pdf

* Similar as with structured encryption (web graphs, graphs, matrices)

* What about running SQL queries https://eprint.iacr.org/2016/453.pdf

* We also allow for ``controlled disclosure'', e.g., the server
reveals one internal node, the root of some subtree, and the client
can then go on and make queries on that subtree. 

https://www.cis.upenn.edu/~mkearns/papers/nwlocal.pdf
Jump and crawl algorithms for analyzing social networks

* Microsoft Azure Marketplace: allow a cloud application to analyze your data.
** Data provider does not want to reveal whole data set, but only ``subtree''
** Cloud Application does not want to leak details about their techniques 
*** Compromise between no security and fully-homomorphic encryption or MPC

* HITS and PageRank: algorithms to analyze properties of an intranet, local sub-tree of the intranet

}



\bibliographystyle{plainnat}\balance
\begin{spacing}{0.9}
  \bibliography{main}
\end{spacing}

\appendix
\section{Decision Trees}
\label{app:dtrees}
As an alternative to their paper, we summarize here the changes
necessary to convert the semi-honest secure protocol from
\citeauthor{wu2016privately} to a fully-malicious-secure version using
$\ioprf$.  We will reference our modifications in contrast with their
protocol (Figure 1 in~\cite{wu2016privately}).

\begin{enumerate}
    \item In step 1, the client proves that their input encryptions
    are bits.  This also happens in the maliciously-secure version
    from the original paper.
    
    \item In step 2, the negation of the intended DGK
      comparisons~\cite{dgk} are also computed.  This way the client
      has a ``successful'' comparison one way or the other to use in
      their proof to the server that they are behaving correctly.
    
    \item In step 4, the server additionally encrypts each node of the
    tree with a symmetric key derived from an $\ioprf$.  The keys are
    chosen such that each node can be decrypted by an $\ioprf$
    evaluation that corresponds to that node's location in the binary
    tree, adjusted for the randomly flipped comparisons.  The goal
    here is to restrict the client to only being able to decrypt the
    nodes corresponding to the contiguous path in the tree resulting
    from its comparisons.

  \item In step 5, the client uses PIR~\cite{chang2004single} to
    retrieve the target leaf node instead of conditional OT.  The
    client additionally runs an $\ioprf$ protocol to retrieve the
    symmetric key necessary to decrypt their chosen leaf node.  In
    execution of this protocol, they also prove in ZK that
    the input to the $\ioprf$ corresponds to the correct results of
    the comparison protocol (see Appendix~\ref{app:zkp}).
\end{enumerate}

\subsection{Binding Homomorphic Comparisons to $\ioprf$ Input}
\label{app:zkp}
Since the client now executes two DGK comparisons per level of the
tree, the original intended one and its negation, they now always have
a ``successful'' comparison at every level, which tells them which
direction to go in the tree.  The main idea behind the proofs that
will bind the client to the correct path is that they can use the
encryption of zero that results from a successful comparison as
evidence to the server that they are going in the direction they are
supposed~to.

At each level of the tree $k\in[d]$, the client creates a ciphertext
$c \leftarrow \mathsf{Enc}_{pk}(0)$ and generates a commitment  $\com$ to
$x_k=1$ if the comparison at that node was true and $x_k=0$ if its
negation was true.  This corresponds to the direction their comparison
at level $k$ in the shuffled tree tells them to go on the next level.
They then must prove that there exists an $i$ such that $c$ (the
encryption of zero) is plaintext-equivalent to either
$\mathsf{ct}_{k,i}$ or $\mathsf{ct}'_{k,i}$ (the result of the negated
comparison), and that if it is $\mathsf{ct}_{k,i}$ then $\com$ is 1, or
if it is $\mathsf{ct}'_{k,i}$ then $\com$ is 0.  Then, $comm$ is used
as the commitment in the $\ioprf$ protocol.  This binds the output of
the comparison to the input of the $\ioprf$, completing the proof.

Let $a \equiv b$ signify that $a$ and $b$ are encryptions of the same
value and $a \equiv {0,1}$ signify that $a$ is an encryption of 0 or
1.  The statement being proven can then be writen as follows

\begin{gather*}
\Bigg [ (c \equiv \mathsf{ct}_{k,1} \lor  \ldots \lor c \equiv
\mathsf{ct}_{k,t}) \land \com \equiv 1 \Bigg ] \\\lor\\ \Bigg [ (c \equiv
\mathsf{ct}'_{k,1} \lor  \ldots \lor c \equiv \mathsf{ct}'_{k,t})
\land \com \equiv 0 \Bigg ] 
\end{gather*}
  
We do not produce a full ZK proof for this statement, as it can be
efficiently designed in the same way we design ZK proofs in
Section~\ref{sec:sec-analysis} (plaintext equivalence is equivalent to
a proof of DDH tuple, one-out-of two technique for the $\lor$,
parallel proofs for $\land$ etc.).  For more details on efficient
composition of ZK proofs, see also~\citet{camenisch1997proof}.

\begin{comment}
\begin{figure*}
\begin{framed}
\begin{flushleft} Let $(\mathsf{pk}, \mathsf{sk})$ be a public-secret
key-pair for an additively homomorphic encryption scheme over
$\mathbb{Z}_p$.  We assume the client holds the secret key.  Fix a
precision $t \leq \lfloor \log_2 p \rfloor$.
\end{flushleft}


\begin{itemize}
    \item {\bf Client input:} A feature vector $x \in \mathbb{Z}_p^n$
    where each $x_i$ is at most $t$ bits.  Let $x_{i,j}$ denote the
    $j^\text{th}$ bit of $x_i$.
    
    \item {\bf Server input:} \ul{An $\ioprf$ key $K = ((\alpha_1,
    \beta_1), \ldots, (\alpha_\ell, \beta_\ell))$}.  A complete, binary
    decision tree $\mathcal{T}$ with $m$ internal nodes.  Let $q_1,
    \ldots, q_\ell$ be the indices of the non-dummy nodes, and let
    $f_{q_k}(x) = 1\{x_{i_k} \leq t_k\}$, where $i_k \in [n]$ and $t_k
    \in \mathbb{Z}_p$.  For the dummy nodes $v$, set $f_v(x) = 0$.
    Let $z_0, \ldots, \in \{0,1\}^*$ be the values of the leaves of
    $\mathcal{T}$.
\end{itemize}

\begin{enumerate}
    \item {\bf Client:} For each $i \in [n]$ and $j \in [t]$, compute
    and send $\mathsf{Enc}_{\mathsf{pk}}(x_{i,j})$ to the server.
    \ul{The client proves to the server in zero-knowledge that
    $\forall i : x_i \in \{0,1\}$.}
    
    \setul{6pt}{.4pt}
    \item {\bf Server:} The server chooses $b \getr
    \{0,1\}^\ell$.  Then, for each $k \in [\ell]$, set $\gamma_k = 1-2
    \cdot b_k$.  For each $k \in [\ell]$ and $j \in [t]$, choose
    $r_{k,j}$,\ul{$r'_{k,j}$}$\getr \mathbb{Z}_p^{*}$ and
    homomorphically compute the ciphertext

    \vskip 2mm

    \hfil $\mathsf{ct}_{k,j} = \mathsf{Enc}_{\mathsf{pk}}  \Bigg
    [r_{k,j} \Bigg ( x_{i_k,w} - t_{k,w} + \gamma_k + 3 \cdot
    \sum_{w<j} (x_{i_k,w} \oplus t_{k,w}) \Bigg ) \Bigg ]$ \hfill

    \vskip 2mm

\ul{and also}
    \vskip 2mm
    \setul{12pt}{.4pt}
    \hfil \ul{$\mathsf{ct}'_{k,j} = \mathsf{Enc}_{\mathsf{pk}}  \Bigg
    [r'_{k,j} \Bigg ( x_{i_k,w} - t_{k,w} - \gamma_k + 3 \cdot
    \sum_{w<j} (x_{i_k,w} \oplus t_{k,w}) \Bigg ) \Bigg ]$} \hfill
    \resetul
    \vskip 2mm

    For each $k\in[\ell]$, the server sends the client the ciphertexts
    $(\mathsf{ct}_{k,1}, \ldots, \mathsf{ct}_{k,t})$ \ul{and
    $(\mathsf{ct'}_{k,1}, \ldots, \mathsf{ct'}_{k,t})$} in
    \emph{random} order.

    \item {\bf Client:} The client obtains a set of $\ell$ tuples of
    the form $(\widetilde{\mathsf{ct}}_{k,1}, \ldots,
    \widetilde{\mathsf{ct}}_{k,t})$ from the server.  For each $k \in
    [\ell]$, it sets $b'_k = 1$ if there exists $j \in [t]$ such that
    $\widetilde{\mathsf{ct}}_{k,j}$ is an encryption of 0.  Otherwise,
    it sets $b'_k = 0$.  The client replies with
    $\mathsf{Enc}_{pk}(b'_1), \ldots, \mathsf{Enc}_pk(b'_\ell)$.
    
    \item {\bf Server:} The server chooses $s \xleftarrow{R}
    \{0,1\}^m$ and constructs the permuted tree $\mathcal{T}' =
    \pi_s(\mathcal{T})$, where $\pi_s$ is the permutation associated
    with the bit-string $s$.  \ul{Let $\mu(m)={\mu_1, \ldots,
    \mu_\ell}$ be the index of a node $m$ such that $\mu_i = 0$ if at
    level $i$ of the tree the path to $m$ goes left and $\mu_i = 1$ if
    it goes right.  Each node of $\mathcal{T}'$ is symmetrically
    encrypted with the key $\ioprf_K(\mu(m) \oplus b)$, representing the
    comparisons necessary from the client to reach this node in the
    tree.} Initialize $\rho = 0^m$.  For $k \in [\ell]$, update
    $\sigma_{i_k} = b_k \oplus b'_k$.  Let $\mathcal{T}_s$ be the
    permutation on the node indices of $\mathcal{T}$ effected by
    $\pi_s$, and compute $\rho' \leftarrow \tau_s(\sigma \oplus s)$.
    The server homomorphically computes $\mathsf{Enc}_{pk}(\rho')$ and
    sends the result to the client.

    \item {\bf Client and Server}: The client decrypts the server's
    message to obtain $\rho'$ and then computes the index $i$ of the
    leaf node containing the response.  \ul{The client uses Private
    Information Retrieval to retrieve this leaf node from the server.
    The client then executes an $\ioprf$ protocol with the server with
    input $b'$ to obtain the key to decrypt this node.  For each $i
    \in [\ell]$, the client proves in zero knowledge to the server
    that $b'_i$ corresponds to the correct evaluation of the
    homomorphic comparison computed at that level (see Appendix~A.1).}
    %Some mystery of latex here, if I try to do \ref on the line above
    %it won't compile, so I "hard coded" the section
\end{enumerate}
\end{framed}
\caption{\label{fig:dectrees} \citeauthor{wu2016privately}'s protocol using an $\ioprf$}
\end{figure*}
\end{comment}

\section{RFID Security Analysis}
\label{sec:rf-proof}
To summarize security requirements, we briefly describe a reactive,
ideal functionality $\myF$.  The database sends their input, keys
$K_\epsilon,K_0,K_1,\ldots,K_{1\ldots{}1}$, all $(2N-1)$ keys of the
key tree, to a TTP, and the reader sends an empty bit string. Then,
the TTP sends $K_\epsilon$ to the reader, and nothing to the database.
The internal state $s$ of TTP is initialized to the empty bit string.
Then, the RFID reader and TTP additionally interact in a total of
$\ell$ rounds. In round $i$, let the internal state be bit string
$s=\gamma_1\ldots{}\gamma_{i-1}$. The reader sends bit $\gamma_i$, and
TTP responds with $K_{\gamma_1\ldots\gamma_{i}}$ and updates its
state to $s=\gamma_1\ldots\gamma_{i}$.

\begin{lemma}\label{rfidproof} In the random oracle model, the modified \citeauthor{molnar} protocol securely realizes 
ideal functionality $\myF$.
\end{lemma}

As the proof of Lemma~\ref{rfidproof} is straightforward, we only
summarize it in a draft.
\begin{proof}[Sketch] We build a simulator for the case of a compromised reader. The simulator for the case of a compromised database works accordingly.
  \begin{enumerate}

  \item Simulator $\myS$ begins by preparing an initially empty
    key-value table $\mathsf{RO}$ to implement a standard random
    oracle functionality $H(\cdot)$. During simulation, whenever any
    party calls $H(k)$ for some input $k$, this functionality will
    check whether pair $(k,v)$ is already in table $\mathsf{RO}$ and
    responds with $v$ in that case. Otherwise, $H$ generates a random
    string $v$ of length $\lambda$, sends $v$ back to the caller, and
    places $(k,v)$ in $\mathsf{RO}$.

  \item Also, $\myS$ generates a random key
    $K=((\alpha_1,\beta_1),\ldots,(\alpha_\ell,\beta_\ell))$ for $\ioprf$.
 $\myS$ sends $\epsilon$ to TTP and receives $K_\epsilon$ which it
    forwards to $\A$.
    
  \item $\myS$ and $\A$ run Construction~\ref{const:ioprf} with $\myS$
    as the sender and $\A$ as the receiver.

    During the $i^\text{th}$ iteration of Construction~\ref{const:ioprf}:
    \begin{enumerate}
    \item $\myS$
      extracts $\A$'s input $x_i$ from the Pedersen commitment,
      forwards it to TTP, and receives back $K_{x_1\ldots{}x_i}$.

    \item $\myS$ adds key-value pair $(g_2^{ \prod_{j=1}^{i}
      \alpha_j^{x_j}\beta_j^{1-x_j}},K_{x_1\ldots{}x_i})$ to table
      $\mathsf{RO}$.   
    \end{enumerate}
\end{enumerate}
Observe that $\A$'s view in the simulation is indistinguishable from their view in a real protocol execution.
\qed\end{proof}
Note that $\A$ can perform an input-substitution attack, i.e., query
for some path which does not match the tag they are currently
interacting with. Without the ability to perform public key
cryptography on the tag, the strongest security for the database one
can guarantee is that the reader can get one path, identifying one tag
and thus can fabricate or clone at most one tag per interaction.

\section{Discussion: Performance of Related Approaches}
\label{sec:perf-related}
$\ioprf$s must be interactive, requiring an interaction per
iteration, and interactivity turns out to be the runtime bottleneck.
Yet, we argue that such interaction is still more efficient than
alternatives.

For example, we could construct a single round $\ioprf$ protocol using
fully homomorphic encryption (FHE).  However, we would then have to
evaluate $\ell$ one-way functions inside the FHE circuit and prove
their correct computation.  We expect such computations would be too
long to be practical even on very powerful hardware.  Another
alternative would be general cryptographic primitives which allow
iterative one-way functions.  Recent Multi-Linear Maps could be used
for this purpose.  However, there exist no secure multi-linear map for
generic constructions, let alone efficient ones.  Lastly, the sender
could compute the $\iprf$ for all possible inputs by the receiver and
the receiver could select one using oblivious transfer.  Another
example of obliviously evaluating such a function are distributed
point functions \cite{fss} which would avoid oblivious transfer.
However, in both cases the server would need to evaluate $2^\ell$
functions rendering this approach quickly infeasible.  In conclusion,
our $\ioprf$ avoids the pitfalls of non-interactive design
alternatives providing practical performance.


Finally, one could envision realizing an $\ioprf$ using general
maliciously MPC frameworks such as MP-SPDZ~\cite{mpspdz} or
efficient maliciously secure 2PC~\cite{empag2pc}. However, it
is sender-receiver interactivity which turns out to be the main
challenge. Evaluation of an arithmetic (SPDZ) or Boolean (2PC) circuit
cannot be stopped, its output revealed, and then continued with new
input. Instead, sender and receiver would need to securely evaluate
$\ell$ different circuits. After evaluating circuit $i$, the receiver
learns the $i^{\text{th}}$ output, and specifies the $(i+1)^\text{st}$
input, and both parties evaluate another circuit. Inside the circuit,
the sender and receiver would need to somehow prove to each other that
they continue the evaluation with correct data which is not
trivial. For example, the circuit would need to output an (encrypted)
state to the sender after each iteration which the circuit then
verifies in the next round based on additional information output to
the receiver. The sender would also need to prove that they are using
the same key as one they have committed to, previously. Recall that
evaluation of cryptographic primitives inside a circuit is very
expensive, even using fast maliciously secure 2PC. For example,
\citet{empag2pc} report $85$~ms for the evaluation of a single SHA2
circuit (amortized over 1024 circuits) in a scenario with latency
comparable to LAN1. This is already more expensive than one full round
of Construction~\ref{const:ioprf}.


\end{document}
