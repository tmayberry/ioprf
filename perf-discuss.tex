\section{Discussion: Performance of Related Approaches}
\label{sec:perf-related}
$\ioprf$s must be interactive, requiring an interaction per
iteration, and interactivity turns out to be the runtime bottleneck.
Yet, we argue that such interaction is still more efficient than
alternatives.

For example, we could construct a single round $\ioprf$ protocol using
fully homomorphic encryption (FHE).  However, we would then have to
evaluate $\ell$ one-way functions inside the FHE circuit and prove
their correct computation.  We expect such computations would be too
long to be practical even on very powerful hardware.  Another
alternative would be general cryptographic primitives which allow
iterative one-way functions.  Recent Multi-Linear Maps could be used
for this purpose.  However, there exist no secure multi-linear map for
generic constructions, let alone efficient ones.  Lastly, the sender
could compute the $\iprf$ for all possible inputs by the receiver and
the receiver could select one using oblivious transfer.  Another
example of obliviously evaluating such a function are distributed
point functions \cite{fss} which would avoid oblivious transfer.
However, in both cases the server would need to evaluate $2^\ell$
functions rendering this approach quickly infeasible.  In conclusion,
our $\ioprf$ avoids the pitfalls of non-interactive design
alternatives providing practical performance.


Finally, one could envision realizing an $\ioprf$ using general
maliciously MPC frameworks such as MP-SPDZ~\cite{mpspdz} or
efficient maliciously secure 2PC~\cite{empag2pc}. However, it
is sender-receiver interactivity which turns out to be the main
challenge. Evaluation of an arithmetic (SPDZ) or Boolean (2PC) circuit
cannot be stopped, its output revealed, and then continued with new
input. Instead, sender and receiver would need to securely evaluate
$\ell$ different circuits. After evaluating circuit $i$, the receiver
learns the $i^{\text{th}}$ output, and specifies the $(i+1)^\text{st}$
input, and both parties evaluate another circuit. Inside the circuit,
the sender and receiver would need to somehow prove to each other that
they continue the evaluation with correct data which is not
trivial. For example, the circuit would need to output an (encrypted)
state to the sender after each iteration which the circuit then
verifies in the next round based on additional information output to
the receiver. The sender would also need to prove that they are using
the same key as one they have committed to, previously. Recall that
evaluation of cryptographic primitives inside a circuit is very
expensive, even using fast maliciously secure 2PC. For example,
\citet{empag2pc} report $85$~ms for the evaluation of a single SHA2
circuit (amortized over 1024 circuits) in a scenario with latency
comparable to LAN1. This is already more expensive than one full round
of Construction~\ref{const:ioprf}.
