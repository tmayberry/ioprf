\section{RFID Security Analysis}
\label{sec:rf-proof}
To summarize security requirements, we briefly describe a reactive,
ideal functionality $\myF$.  The database sends their input, keys
$K_\epsilon,K_0,K_1,\ldots,K_{1\ldots{}1}$, all $(2N-1)$ keys of the
key tree, to a TTP, and the reader sends an empty bit string. Then,
the TTP sends $K_\epsilon$ to the reader, and nothing to the database.
The internal state $s$ of TTP is initialized to the empty bit string.
Then, the RFID reader and TTP additionally interact in a total of
$\ell$ rounds. In round $i$, let the internal state be bit string
$s=\gamma_1\ldots{}\gamma_{i-1}$. The reader sends bit $\gamma_i$, and
TTP responds with $K_{\gamma_1\ldots\gamma_{i}}$ and updates its
state to $s=\gamma_1\ldots\gamma_{i}$.

\begin{lemma}\label{rfidproof} In the random oracle model, the modified \citeauthor{molnar} protocol securely realizes 
ideal functionality $\myF$.
\end{lemma}

As the proof of Lemma~\ref{rfidproof} is straightforward, we only
summarize it in a draft.
\begin{proof}[Sketch] We build a simulator for the case of a compromised reader. The simulator for the case of a compromised database works accordingly.
  \begin{enumerate}

  \item Simulator $\myS$ begins by preparing an initially empty
    key-value table $\mathsf{RO}$ to implement a standard random
    oracle functionality $H(\cdot)$. During simulation, whenever any
    party calls $H(k)$ for some input $k$, this functionality will
    check whether pair $(k,v)$ is already in table $\mathsf{RO}$ and
    responds with $v$ in that case. Otherwise, $H$ generates a random
    string $v$ of length $\lambda$, sends $v$ back to the caller, and
    places $(k,v)$ in $\mathsf{RO}$.

  \item Also, $\myS$ generates a random key
    $K=((\alpha_1,\beta_1),\ldots,(\alpha_\ell,\beta_\ell))$ for $\ioprf$.
 $\myS$ sends $\epsilon$ to TTP and receives $K_\epsilon$ which it
    forwards to $\A$.
    
  \item $\myS$ and $\A$ run Construction~\ref{const:ioprf} with $\myS$
    as the sender and $\A$ as the receiver.

    During the $i^\text{th}$ iteration of Construction~\ref{const:ioprf}:
    \begin{enumerate}
    \item $\myS$
      extracts $\A$'s input $x_i$ from the Pedersen commitment,
      forwards it to TTP, and receives back $K_{x_1\ldots{}x_i}$.

    \item $\myS$ adds key-value pair $(g_2^{ \prod_{j=1}^{i}
      \alpha_j^{x_j}\beta_j^{1-x_j}},K_{x_1\ldots{}x_i})$ to table
      $\mathsf{RO}$.   
    \end{enumerate}
\end{enumerate}
Observe that $\A$'s view in the simulation is indistinguishable from their view in a real protocol execution.
\qed\end{proof}
Note that $\A$ can perform an input-substitution attack, i.e., query
for some path which does not match the tag they are currently
interacting with. Without the ability to perform public key
cryptography on the tag, the strongest security for the database one
can guarantee is that the reader can get one path, identifying one tag
and thus can fabricate or clone at most one tag per interaction.
