\subsubsection{Security Analysis}
{\bf Ideal Functionality $\myF$:}
We describe a reactive ideal functionality $\myF$.  The database sends
their input, keys $K_\myroot,K_0,K_1,\ldots,K_{1\ldots{}1}$, all $N$
keys of the key tree, to $\myF$, and the reader sends empty bit string
$\epsilon$. In return, $\myF$ sends $K_\myroot$ to the reader, and
nothing to the database.  The internal state $s$ of $\myF$ is
initialized to the empty bit string $\epsilon$.

Then, the RFID reader and $\myF$ additionally interact in a total of
$\ell$ rounds. In round $i$, let the internal state be bit string
$s=\beta_1\ldots{}\beta_{i-1}$. The reader sends bit $\beta_i$, and
$\myF$ responds with $K_{\beta_1\ldots\beta_{i}}$ and updates its
state to $s=\beta_1\ldots\beta_{i}$.


\paragraph{Proof}
\fixme{This seems trivial.}

\begin{proof}[Proof Draft]
  %For the proof, we assume that the RFID reader knows
  %$\trace=(r,H(r,
  %K_\myroot,x_1),\ldots,H(r,K_{x_1\ldots{}x_{\ell-1}},x_{\ell}),H(r,K_{x_1\ldots{}x_\ell}))$
  %from a valid tag $x$.

Public information: the total number $N$ of tags in the system (leaves
in tree $T$, $\lambda$ security parameter.

  \begin{enumerate}

  \item Simulator $\myS$ begins by preparing an initially empty
    key-value table $\mathsf{RO}$ to implement a standard random
    oracle functionality $H(\cdot)$. During simulation, whenever any
    party calls $H(k)$ for some input $k$, this functionality will
    check whether pair $(k,v)$ is already in table $\mathsf{RO}$ and
    responds with $v$ in that case. Otherwise, $H$ generates a random
    string $v$ of length $\lambda$, sends $v$ back to the caller, and
    places $(k,v)$ in $\mathsf{RO}$.

  \item Also, $\myS$ generates a random key
    $\kappa=((a_1,b_1),\ldots,(a_\ell,b_\ell))$ for $\ioprf$ and generates
    a common reference string including parameters which will allow
    $\myF$ to extract commitments from $\A$ (\fixme{be more formal,
    do some kind of GEN algorithm for iOPRF}).

  \item $\myS$ sends $\epsilon$ to $\myF$ and receives $K_\epsilon$
    which it forwards to $\A$.
    
  \item $\myS$ and $\A$ run $\proto$ with $\myS$ as the sender and
    $\A$ as the receiver.

    Recall that protocol $\proto$ runs iteratively with two messages
    exchanged between sender and receiver during each
    iteration. Consequently, $\A$ can adaptively choose their input
    bit $x_i$ for the $i^\text{th}$ iteration. The input of $\myS$ to
    $\proto$ is key $\kappa$.

    During the $i^\text{th}$ iteration\fixme{add check all ZK proofs
      etc}:
    \begin{enumerate}
    \item After receiving the first message in $\proto$, $\myS$
      extracts $\A$'s input $x_i$ from the Elgamal commitment,
      forwards it to $\myF$, and receives back $K_{x_1\ldots{}x_i}$.

    \item $\myS$ adds key-value pair
      $(( \prod_{j=1}^{i}
      a_j^{x_j}b_j^{1-x_j})\cdot{}G,K_{x_1\ldots{}x_i})$ to table
      $\mathsf{RO}$.
   
    \end{enumerate}
\end{enumerate}
\end{proof}
