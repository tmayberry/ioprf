\subsection{RFID}
We support a total of $N$ tags in the system. Each tag uniquely
corresponds to a leaf of a height $\ell$ (binary) key tree. To
identify a tag, a reader (a simple device which can query the tag and
has Internet connectivity) can talk to a database which knows all keys
of the key tree.

The reader wants to know whether a tag is valid by interacting first
with the tag and then with the database. To protect tag privacy,
internal details of a supply/distribution chain etc, the database
should not learn which tag the reader is querying for. An adversary
observing tag-reader interaction or being able to query tags
themselves should not be able to identify or trace/follow tags or
even fabricate new tags, too.


The database knows a secret key $K$ and populates a binary key tree
$T$ as follows. First, nodes in this key tree are indexed by bit
strings following the intuitive notation that the left child (``0'')
of some node indexed by bit string $\gamma_1\ldots\gamma_i$ is index by
$\gamma_1\ldots\gamma_i0$, and the right child (``1'') is indexed by
$\gamma_1\ldots\gamma_i1$. By convention, the root is indexed by
empty bit string $\epsilon$.

Root node $\epsilon$ stores random key
$K_{\epsilon}\getr\{0,1\}^\lambda$.  The left child of the root stores
key $K_0=\ioprf_K(0)$, and the right child stores key
$K_1=\ioprf_K(1)$. For a node $\gamma_1\ldots\gamma_i$,
the left child stores key
$K_{\gamma_1\ldots\gamma_i0}=\ioprf_K(\gamma_1\ldots\gamma_i0)$, and its
right child stores key
$K_{\gamma_1\ldots\gamma_i1}=\ioprf_K(\gamma_1\ldots\gamma_i1)$.

During authentication of tag $x$, the database will run
$\ioprf_K(\cdot)$ as the sender, and the reader will be the receiver
with input bit strings $x=x_1\ldots{}x_\ell$.

Here are protocol details.
\begin{itemize}
  
\item During initialization of a new tag $x$, the database stores a
  sequence of $\ell+1$ keys $K$ on the tag: one for each node on the
  path from the root of the database's tree $T$ to leaf
  $x=x_1\ldots{}x_\ell$. The tag also stores its own ID, i.e., $x$.

\item Each tag now identifies itself to a reader using a variation of
  the Molnar (\fixme{Difference: we send the next bit by appending
    it}) protocol:

  \begin{itemize}
  \item Tag $x$ chooses a random $r$ and sends $r$ together with a
    hash of $r$ and each of their $\ell+1$ keys and the next bit,
    respectively:
    $\trace=(r,T_0=H(r,
    K_\myroot,x_1),\ldots,T_\ell=H(r,K_{x_1\ldots{}x_{\ell-1}},x_{\ell}),H(r,K_{x_1\ldots{}x_\ell}))$.

\item The reader uses our $\ioprf$ to identify the tag as follows (I am now sticking to our running example of $x=1011$):

 \begin{itemize}

 \item The database beings by sending $K_\epsilon$ to the reader.
   
  \item The reader checks whether either $H(r,K_\epsilon,0)$ or $H(r,K_\epsilon,1)$  matches
    $T_0$.

  \item If yes, the reader would continue and query either the left or
    right child of the root with the $\ioprf$, compute keys, check
    which matches etc.
\end{itemize}
  \end{itemize}
\end{itemize}

As you can see, the security we are aiming for asks only for a
(delegatable) OPRF. Our $\ioprf$ supports delegation, but can do more. We
could also ask as an additional security requirement that the reader
should only learn ``one path'', i.e., one tag per interaction with the
database. 


\newpage
section{This is the old protocol:}

Here are protocol details.
\begin{itemize}

\item During initialization of a new tag $x$, the database stores a
  sequence of $\ell$ keys $K$ on the tag, one for each node on the path from
  the root of the database's tree to leaf $x$. We keep the intuitive $0$
  and $1$ notation also for keys $K$, so for example tag $x = 1011$ stores
  keys $K_1$, $K_{10}$, $K_{101}$, $K_{1011}$, a total of $\ell=4$ keys.

\item The database computes each key as follows:  
\begin{itemize}
\item for a key corresponding to node $i$, it computes $\seed =
  \ioprf_K(\mathsf{PARENT}(i))$. For example, for $K_{1011}$, it would
  compute $\seed = \ioprf_K(101)$.

\item the database then computes
  $K_{\mathsf{leftChild}}||K_{\mathsf{rightChild}} = \prg(\seed)$. In
  our example, it would compute $K_{1010}||K_{1011} = \prg(\seed)$.

\item the database stores one of the two keys on the tag, the one corresponding to the node. So, $K_{1011}$ in our case.
\end{itemize}

\item Each tag can now identify itself to a reader using the Molnar protocol:

  \begin{itemize}
\item The tag chooses a random $r$ and sends $r$ together with a hash (or a PRF) of $r$ and each of their $\ell$ keys. So, it would send $r, H(r||K_1), H(r||K_{10}), H(r||K_{101}), H(r||K_{1011})$ to the reader.

\item The reader uses our $\ioprf$ to identify the tag as follows (I am now sticking to our running example of $x=1011$):
  \begin{itemize}
    
\item The reader would query the database for $\seed =
  \ioprf_K(\myroot)$, for some special input symbol $\myroot$.

\item The reader would derive $K_0$ and $K_1$ and would then check which of them matches the tag's first hash evaluation. 

\item If yes, the reader would continue and query either the left or right child of the root with the $\ioprf$, compute keys, check which hash matches etc.
\end{itemize}
  \end{itemize}
\end{itemize}

As you can see, the security we are aiming for asks only for a
delegatable OPRF. Our $\ioprf$ supports delegation, but can do more. We
could also ask as an additional security requirement that the reader
should only learn ``one path'', i.e., one tag per interaction with the
database. This is not 100\% true, because we currently also leak the
sibling of each ``key node'' in the above protocol. I don't think this
is a major problem though.
