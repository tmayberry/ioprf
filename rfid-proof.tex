\documentclass{article}

\usepackage{xfrac,amsmath,amsthm,amssymb,parskip,enumitem,url}
\usepackage[numbers,sort&compress]{natbib}

\newtheorem{definition}{Definition}
\newtheorem{construction}{Construction}
\newtheorem{theorem}{Theorem}

\newcommand{\oprf}[0]{\mathsf{OPRF}}
\newcommand{\getr}[0]{\stackrel{\$}{\leftarrow}}
\newcommand{\enc}[0]{{\mathsf{Enc}}}
\newcommand{\dec}[0]{{\mathsf{Dec}}}
\newcommand{\fixme}[1]{{\bf{}\ [FIXME:} {\emph{#1}} {\bf{}]}}
\newcommand{\todo}[1]{{\bf ToDo:} {{\bf #1}}}
\newcommand{\dash}[0]{{\text -}}
\newcommand{\A}[0]{{\mathcal{A}}}


\newcommand{\myO}[0]{\mathcal{O}}
\newcommand{\ioprf}[0]{\mathsf{i}\mathsf{OPRF}}
\newcommand{\iprf}[0]{\mathsf{i}\mathsf{PRF}}
\newcommand{\ot}[0]{\mathsf{OT}}
\newcommand{\proto}[0]{{\pi_{\ioprf}}}
\newcommand{\myS}[0]{{\mathcal{S}}}
\newcommand{\myF}[0]{{\mathcal{F}}}
\newcommand{\Z}[0]{\mathbb{Z}}
\newcommand{\G}[0]{\mathbb{G}}
\newcommand{\Hide}[1]{}

\newcommand{\prg}[0]{\mathsf{PRG}}
\newcommand{\seed}[0]{\mathsf{seed}}
\newcommand{\myroot}[0]{\mathsf{ROOT}}

\newcommand{\trace}[0]{\mathsf{Trace}}
\newcommand{\crs}[0]{\mathsf{CRS}}
\newcommand{\com}[0]{\mathsf{com}}
\newcommand{\sr}[0]{{\stackrel{?}{=}}}

\let\ignore\Hide

\newcommand{\N}[0]{{\mathbb{N}}}

\makeatletter
\def\old@comma{,}
\catcode`\,=13
\def,{%
  \ifmmode%
    \old@comma\discretionary{}{}{}%
  \else%
    \old@comma%
  \fi%
}
\makeatother


\begin{document}
\section{Ideal Functionality $\myF$}
We describe a reactive ideal functionality $\myF$.  The database sends
their input, keys $K_\myroot,K_0,K_1,\ldots,K_{1\ldots{}1}$, all $N$
keys of the key tree, to $\myF$, and the reader sends empty bit string
$\epsilon$. In return, $\myF$ sends $K_\myroot$ to the reader, and
nothing to the database.  The internal state $s$ of $\myF$ is
initialized to the empty bit string $\epsilon$.

Then, the RFID reader and $\myF$ additionally interact in a total of
$\ell$ rounds. In round $i$, let the internal state be bit string
$s=\beta_1\ldots{}\beta_{i-1}$. The reader sends bit $\beta_i$, and
$\myF$ responds with $K_{\beta_1\ldots\beta_{i}}$ and updates its
state to $s=\beta_1\ldots\beta_{i}$.


\section{Proof}
N.B.: For the proof, we assume that the RFID reader knows
$\trace=(r,H(r,
K_\myroot,x_1),\ldots,H(r,K_{x_1\ldots{}x_{\ell-1}},x_{\ell}),H(r,K_{x_1\ldots{}x_\ell}))$
from a valid tag $x$.

Public information: the total number $N$ of tags in the system (leaves
in tree $T$, $\lambda$ security parameter.

\begin{proof}[Draft]
  \begin{itemize}

  \item Simulator $\myS$ begins by preparing an initially empty
    key-value table to implement a standard random oracle
    functionality $H$. Whenever any party calls $H(\cdot)$ with some
    input $k$, this functionality will check whether $k$ is already in
    the table and responds with $v$. Otherwise, $H$ generates a random
    string $v$ of length $\lambda$, sends $v$ back to the caller, and
    places $(k,v)$ in the table.

  \item Adversary $\A$ sends $v'$ to $\myS$ (\fixme{or extractable
      commitment}). $\myS$ looks up $v'$ in its internal table and
    thus extracts $k'=\trace'$ such that $v'=H(\trace)$.

  \item $\myS$ forwards $\trace$ to $\myF$ and gets back reply
    $\beta_1\ldots\beta_\ell=\mathsf{ValidPrefix(\trace')}$.
    
  \item  $\myS$ begins by building their own binary key tree
    $T'$ with $n$ leaves.  For each node, $\myS$ randomly chooses a
    bit string of length $\lambda$ and writes into the node.  Using
    the same notation as with the database's tree $T$, each node
    $\beta_1\ldots{}\beta_i$ in $T'$ thus stores a random bit string
    $K'_{\beta_1\ldots{}\beta_i}$, and the root stores random bit
    string $K'_\myroot$.



    
    \end{itemize}
\end{proof}


\end{document}